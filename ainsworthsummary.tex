\documentclass[]{article}

\begin{document}
\section{Summary of article}

It is by Robert Ainsworth and JK Slingerland from the National University of
Ireland, and titled ``Topological qubit design and leakage''. Published June
2011 in the New Journal of Physics.

They study 2 questions:

\begin{enumerate}
\item What is the optimal number of anyons to use to model a qubit?
\item To what extent can leakage be avoided?
\end{enumerate}

All of their proofs are completely general and rely only on theorems about
representations of the braid group.


\subsection{Qubit design}

They prove that if we use $n > 4$ anyons to model a qubit, then either:

\begin{itemize}
\item the representation of $B_n$ is abelian
\item there is leakage into the noncomputational space in single qubit operations
\end{itemize}

In general for modelling qudits, the same holds if we use $n > d+2$ anyons.

The mathematical problem that this reduces to is finding the maximum $n$ for
which there is a nonabelian $d$-dimensional representation of $B_n$. A
(nontrivial) theorem shows that the answer is always $d+2$.

\subsection{Leakage for multiqubit gates}

They prove that if there is a system of 2 qubits where every 2-qubit gate does
not cause leakage, then the anyons are Ising anyons.

Further, they show that there are no nonabelian anyons where all the gates on 2
qutrits or 1 qubit + 1 qutrit are leakage-free.

However, if each anyon carries a representation of a quantum group, then their
tensor product does as well and thus all gates are leakage-free. However, it is
conjectured that the braid group representations that occur in this way have
finite image.


\end{document}
