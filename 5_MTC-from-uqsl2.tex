% from Bakalov & Kirillov, section 3.3
We will denote the category of representations of $U_q(\mathfrak{g})$ at a
$\ell^\text{th}$ root of unity $q$ by $\mathcal{C}(\mathfrak{g}, \ell)$. 

This is a rigid monoidal category because of the Hopf algebra structure on
$U_q(\mathfrak{g})$ described above. 

As discussed in the previous section, $\mathcal{C}(\sll(2),\ell)$ is not
semisimple, and further has an infinite number of nonisomorphic
$\ell$-dimensional simple objects.  

The same is true for the category $\mathcal{C}(\mathfrak{g}, \ell)$ for a
general semisimple Lie algebra $\mathfrak{g}$.

Our goal is to find an appropriate semisimple part of this category to obtain a
modular tensor category as a quotient.




\section{Quantum trace}

The Hopf algebra structure on $U_q(\mathfrak{g})$, coupled with the $R$-matrix,
gives the category of representations $\mathcal{C}(\mathfrak{g}, \ell)$ the
structure of a ribbon category. There is therefore a well-defined trace of an
endomorphism in the category. This is distinct from the usual trace of an
vector space endomorphism. 

We will denote the trace of a morphism $f$ by $\tr_q(f)$.

In the case $U_q(\sll(2))$, the quantum trace is given by 

\begin{equation}
    \tr_q(f) = \tr(Kf)
\end{equation}

where $\tr$ is the usual trace of a vector space endomorphism.

\section{Negligible morphisms}
\todo{ sort out which things should be theorems and which things should be lemmas}
To make this subcategory into a modular tensor categories, we will need to
quotient this category by the modules with quantum dimension zero. This is a
fairly natural thing to do.  \todo{ why, really?}

\begin{defn}
    A morphism $f: T_1 \to T_2$ is \emph{negligible} if $\tr_q(fg) = 0$ for any $g: T_2 \to T_1$
\end{defn}


\begin{defn}
    A module $T$ is \emph{negligible} if all of its endomorphisms are negligible.
\end{defn}
\section{Tilting modules} 
% when citing, say something like: originally done by Anderson & P, there's
% something more accessible by Sawin

The first step is to restrict our attention to the subcategory of \emph{tilting
modules}. We will define tilting modules, describe some results classifying the
tilting modules in $U_q(\mathfrak{g})$, and finally list the tilting modules in
the category $\mathcal{C}(\sll(2), \ell)$.

\todo{ define Weyl modules}
\begin{defn}
    Suppose $M$ is a $U_q(\mathfrak{g})$-module. A \emph{Weyl filtration} $M$
    is a sequence of submodules 

    \begin{equation}
        \left\{ 0 \right\} = J_0 \subset \cdots \subset J_n = M
    \end{equation}

    such each $J_k$ is a maximal submodule of $J_{k+1}$ and each quotient $J_{k+1}/J_k$ is a Weyl module. 
\end{defn}

\begin{defn}
    A $U_q(\sll(2))$-module $M$ is called \emph{tilting} if both $M$ and $M^*$ have Weyl filtrations.
\end{defn}


\begin{theorem}
    Every tilting module is a direct sum of indecomposable tilting modules.
    Every indecomposable tilting module is isomorphic to some $T_\lambda$, the
    unique indecomposable tilting module with a maximal vector of weight
    $\lambda$.
\end{theorem}



\begin{lemma}
    A tilting module $T$ is negligible if and only if its quantum dimension is 0.
\end{lemma}

\todo{ actually define this region}
Define the region $\Lambda^\ell$.

The following two theorems provide a characterization of the tilting modules
sufficient for our purposes.

\begin{theorem}
For any $\lambda \notin \Lambda^\ell$, $T^\lambda$ is negligible. 
\end{theorem}

\begin{theorem}
For any $\lambda \in \Lambda^\ell$, $T^\lambda = W^\lambda$.
\end{theorem}



\section{Construction of the MTC}
\label{MTC-construction}

We construct the modular tensor category that we will use to do topological
computing by restricting our attention to the subcategory of tilting modules of
$U_q(\mathfrak{g})$, and quotienting by the negligible morphisms. We can do
this because the hom sets $Hom(V,W)$ are vector spaces:

\todo{ change the notation here, I guess}
\begin{defn}
    Define the category $\mathcal{C}^\text{int}$ be the category with objects tilting modules and morphisms 

    \begin{equation*}
        \Hom(V,W) = \Hom_T(V,W) / \text{negligible morphisms}
    \end{equation*}

\end{defn}

The category $\mathcal{C}^\text{int}$ has the following properties:
\begin{enumerate}
    \item An object $T$ is negligible if and only if it is isomorphic to 0 in $\mathcal{C}^\text{int}$
    \item  $\mathcal{C}^\text{int}$ is a ribbon category
    \item Any object $T$ in $\mathcal{C}^\text{int}$ is isomorphic to a direct sum of Weyl modules.
    \item $\mathcal{C}^\text{int}$ is a semisimple abelian category. 
    \item $\dim_{\mathcal{C}^\text{int}} T > 0$ for every $T \not\simeq 0$
\end{enumerate}

$\mathcal{C}^\text{int}$ is in fact a modular tensor category. A proof of this
can be found in \cite{Kirillov} 
 
\section{Fusion rules}

The fusion rules for $U_q(\sll(2))$ are given by 

\todo{ notation is inconsistent here}
\begin{equation}
    V_m \otimes V_n \simeq \sum_i N_{mn}^i V_i
\end{equation}

where 

\begin{equation}
    N_{mn}^i = \begin{cases} 1 \text{ for } |m-n| \leq i \leq m+n, i \leq 2k - (m+n), i + m + n \in 2 \mathbf{Z} \\
                             0 \text{ else } 
               \end{cases}
\end{equation}

% Bakalov & Kirillov


