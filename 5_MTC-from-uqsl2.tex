\label{chapter:MTC}
% from Bakalov & Kirillov, section 3.3
The category of finite dimensional representations of $U_z^{res}(\sll(2))$
which are direct sum of their weight spaces is a braided monoidal category: it
is monoidal because $U_z^{res}(\sll(2))$ is a Hopf algebra, and we saw that it
is braided in \ref{section:braiding}. Call this category
$\Rep_w(U_z^{res}(\sll(2)))$. This category is certainly not a modular tensor
category: it is not semisimple, and it has infinitely many nonisomorphic simple
objects. The main goal of this chapter is to describe the passage from
representations of quantum groups to modular tensor categories.  The simple
objects in the modular tensor category we construct will be $\left\{
W_z^{res}(0), \ldots, W_z^{res}(\ell-2) \right\}$. We construct this category
by first restricting to the subcategory of ``tilting modules'' and then
quotienting this subcategory. Basically the intuition is that we take our
simple objects to be $\left\{ W_z^{res}(0), \ldots, W_z^{res}(\ell-2) \right\}$
and then quotient by the objects with quantum dimension zero.  The simple
objects in this modular tensor category will correspond to anyon types.



\section{Quantum trace and dimension}

The objects in $\Rep_w(U_q(\sll(2)))$ are vector spaces, and so any
endomorphism $f$ has a trace $\tr(f)$. $\Rep_w(U_q(\sll(2)))$ is a ribbon
category, and we saw in section \ref{section:RibbonCategories} that there is a
trace in any ribbon category. In $\Rep_w(U_q(\sll(2)))$, this trace does
\emph{not} correspond to the usual trace of a linear map, and we will denote it
by $\tr_q$, the ``quantum trace''. It is a useful exercise to show from the
definition that for any $f: V \to V$,

\begin{equation}
\tr_q(f) = \tr(Kf)
\end{equation}

where $\tr$ is the usual trace of a linear map. We also define the quantum
dimension $\dim_q$,
\begin{equation}
\dim_q(V) = \tr_q(\id_V)
\end{equation}

For example, we can see from the definition of the Weyl module $W_z^{res}(n)$
in \ref{section:RepTheoryofResSL2} that $\dim_q(W_z^{res}(n)) = [n+1]_z$. In
particular we see that $\dim_q(W_z^{res}(\ell - 1)) = 0$, so objects other than
the zero object can have quantum dimension zero. 

\section{Tilting modules} 
\label{section:Tilting}

We saw earlier that tensor products of irreducible $U_q(\sll(2))$-modules
$V_q(n)$ decompose as a direct sum of irreducible modules. This will not be the
case for irreducible $U_z^{res}(\sll(2))$-modules. However,
$W_z^{res}(n)\otimes W_z^{res}(m)$ decomposes as a direct sum of indecomposable
so-called \emph{tilting modules}. In fact the indecomposable tilting modules are exactly
the direct summands of the tensor products of the Weyl modules. We will
quotient the subcategory of tilting modules to obtain a modular tensor
category. 

In this section we will give a technical definition for tilting modules, define
what they are explicitly in certain cases, and give an explicit decomposition
for $W^{res}_z(n) \otimes W_z^{res}(m)$ for $0 \leq n,m \leq \ell - 1$.

\begin{defn}
    Suppose $M$ is a $U_q(\sll(2))$-module. A \emph{Weyl filtration} for $M$
    is a sequence of submodules 

    \begin{equation}
        \left\{ 0 \right\} = J_0 \subsetneq \cdots \subsetneq J_n = M
    \end{equation}

    such that each $J_k$ is a maximal submodule of $J_{k+1}$ and each quotient $J_{k+1}/J_k$ is a Weyl module. 
\end{defn}

\begin{defn}
    A $U_q(\sll(2))$-module $M$ is called \emph{tilting} if both $M$ and $M^*$ have Weyl filtrations.
\end{defn}

This definition by itself is not very illuminating, but again the point is that
tilting modules are the direct summands of tensor products of Weyl modules.
Let $\tilt$ be the full subcategory of $\mathcal{C}(\sll(2), \ell)$ whose
objects are the tilting modules modules for $U_z^{res}(\sll(2))$.  

\begin{prop}
    \label{prop:tiltingclosed}
    $\tilt$ is closed under direct sums, tensor products, duals, and taking direct summands.
\end{prop}
\begin{proof}
    It is easy to see that $\tilt$ is closed under direct sums, duals, and
    taking direct summands. It is much harder to show that the tensor product
    of tilting modules is tilting, and the proof can be found in \cite{Andersen1992}.
\end{proof}

We can therefore restrict our attention to the indecomposable tilting modules.
The indecomposable tilting modules for $U_z^{res}(\sll(2))$ are indexed by integers $n
\geq 0$ (corresponding to their maximal weights), and the tilting modules
$T_z(n)$ for $0 \leq n < 2\ell - 2$ can be described explicitly as follows
(\cite{CP}). 

$T_z(n)$ has basis $\left\{ t_0, \ldots, t_n \right\} \bigcup \left\{ t'_0,
    \ldots, t'_{r} \right\}$, where $r = 2\ell - 2 - n$, and has the following
    action:

\begin{align}
    Kt_i &= z^{n-2i} t_i \\
    X^+ t_i &= [n-i+1]_z t_{i-1} \\
    (X^+)^{(\ell)} t_i &= ((n-i)_1) + 1) t_{i-\ell} \\
    X^-t_i &= [i+1]_z t_{i+1} \\
    (X^-)^{(\ell)} t_i &= (i_1 + 1) t_{i+\ell} \\
    Kt'_i &= z^{r-2i} t'_i \\
    X^+ t'_i &= [r-i+1]_z t'_{i-1}  + \dbinom{n + i - \ell}{i}_z t_{n+i-\ell} &\text{ if $0 < i \leq r$}\\
    X^+ t'_0 &= [n - \ell + 1]_z t_{n-\ell} \\
    X^-t'_i &= [i+1]_z t'_{i+1} &\text{ if $0 \leq i < r$}\\
    X^- t'_r &= \dbinom{\ell - 1}{n - \ell + 1}_z t_\ell \\
    (X^\pm)^{(\ell)} t'_i  &= 0
\end{align}

The essential aspect of these equations can be captured diagrammatically. For
example, when $\ell = 5, n = 5$, $T_z(5)$ looks as follows:

\begin{center}
    \begin{figure}
        \caption{The upwards pointing arrows indicate the action of $X^+$ and
        the downwards pointing arrows indicate that of $X^-$.  }
\scalebox{1} % Change this value to rescale the drawing.
{
\begin{pspicture}(0,-1.57125)(2.905,1.57125)
\psbezier[linewidth=0.01,arrowsize=0.05291667cm 2.0,arrowlength=1.4,arrowinset=0.4]{->}(0.48,0.88875)(0.62,1.02875)(0.64,1.12875)(0.48,1.26875)
\rput(0.3021875,1.37875) {$t_0$}
\rput(0.3021875,0.81875) {$t_1$}
\rput(0.3021875,0.25875) {$t_2$}
\rput(0.3021875,-0.30125){$t_3$}
\rput(0.3021875,-0.86125){$t_4$}
\rput(0.3021875,-1.42125){$t_5$}

\rput(2.5421875,0.81875) {$t'_0$}
\rput(2.5421875,0.25875) {$t'_1$}
\rput(2.5421875,-0.30125){$t'_2$}
\rput(2.5421875,-0.86125){$t'_3$}
\psline[linewidth=0.01cm,arrowsize=0.05291667cm 2.0,arrowlength=1.4,arrowinset=0.4]{->}(2.3,0.28875)(0.62,0.70875)
\psline[linewidth=0.01cm,arrowsize=0.05291667cm 2.0,arrowlength=1.4,arrowinset=0.4]{->}(2.3,-0.97125)(0.66,-1.31125)
\psbezier[linewidth=0.01,arrowsize=0.05291667cm 2.0,arrowlength=1.4,arrowinset=0.4]{->}(2.74,-0.79125)(2.88,-0.65125)(2.9,-0.55125)(2.74,-0.41125)
\psbezier[linewidth=0.01,arrowsize=0.05291667cm 2.0,arrowlength=1.4,arrowinset=0.4]{->}(2.72,-0.27125)(2.86,-0.13125)(2.88,-0.03125)(2.72,0.10875)
\psbezier[linewidth=0.01,arrowsize=0.05291667cm 2.0,arrowlength=1.4,arrowinset=0.4]{->}(2.72,0.28875)(2.86,0.42875)(2.88,0.52875)(2.72,0.66875)
\psbezier[linewidth=0.01,arrowsize=0.05291667cm 2.0,arrowlength=1.4,arrowinset=0.4]{->}(0.48,0.30875)(0.62,0.44875)(0.64,0.54875)(0.48,0.68875)
\psbezier[linewidth=0.01,arrowsize=0.05291667cm 2.0,arrowlength=1.4,arrowinset=0.4]{->}(0.48,-0.25125)(0.62,-0.11125)(0.64,-0.01125)(0.48,0.12875)
\psbezier[linewidth=0.01,arrowsize=0.05291667cm 2.0,arrowlength=1.4,arrowinset=0.4]{->}(0.46,-0.83125)(0.6,-0.69125)(0.62,-0.59125)(0.46,-0.45125)
\psbezier[linewidth=0.01,arrowsize=0.05291667cm 2.0,arrowlength=1.4,arrowinset=0.4]{->}(0.48,-1.33125)(0.62,-1.19125)(0.64,-1.09125)(0.48,-0.95125)
\psbezier[linewidth=0.01,arrowsize=0.05291667cm 2.0,arrowlength=1.4,arrowinset=0.4]{<-}(0.14,-1.35125)(0.02,-1.21125)(0.06,-1.09125)(0.14,-0.97125)
\psbezier[linewidth=0.01,arrowsize=0.05291667cm 2.0,arrowlength=1.4,arrowinset=0.4]{<-}(0.14,-0.77125)(0.02,-0.63125)(0.06,-0.51125)(0.14,-0.39125)
\psbezier[linewidth=0.01,arrowsize=0.05291667cm 2.0,arrowlength=1.4,arrowinset=0.4]{<-}(0.14,-0.23125)(0.02,-0.09125)(0.06,0.02875)(0.14,0.14875)
\psbezier[linewidth=0.01,arrowsize=0.05291667cm 2.0,arrowlength=1.4,arrowinset=0.4]{<-}(2.38,-0.79125)(2.26,-0.65125)(2.3,-0.53125)(2.38,-0.41125)
\psbezier[linewidth=0.01,arrowsize=0.05291667cm 2.0,arrowlength=1.4,arrowinset=0.4]{<-}(2.36,-0.23125)(2.24,-0.09125)(2.28,0.02875)(2.36,0.14875)
\psbezier[linewidth=0.01,arrowsize=0.05291667cm 2.0,arrowlength=1.4,arrowinset=0.4]{<-}(2.36,0.32875)(2.24,0.46875)(2.28,0.58875)(2.36,0.70875)
\psbezier[linewidth=0.01,arrowsize=0.05291667cm 2.0,arrowlength=1.4,arrowinset=0.4]{<-}(0.12,0.32875)(0.0,0.46875)(0.04,0.58875)(0.12,0.70875)
\psbezier[linewidth=0.01,arrowsize=0.05291667cm 2.0,arrowlength=1.4,arrowinset=0.4]{<-}(0.12,0.98875)(0.0,1.12875)(0.04,1.24875)(0.12,1.36875)
\psline[linewidth=0.01cm,arrowsize=0.05291667cm 2.0,arrowlength=1.4,arrowinset=0.4]{->}(2.3,0.84875)(0.62,1.26875)
\psline[linewidth=0.01cm,arrowsize=0.05291667cm 2.0,arrowlength=1.4,arrowinset=0.4]{->}(2.3,-0.27125)(0.62,0.14875)
\psline[linewidth=0.01cm,arrowsize=0.05291667cm 2.0,arrowlength=1.4,arrowinset=0.4]{->}(2.3,-0.83125)(0.62,-0.41125)
\end{pspicture} 
}
    \end{figure}
\end{center}


It is easy to see from the figure that the $T_z(n)$ are indecomposable, and that $\left\{ t_0,
\ldots, t_n \right\}$ span a submodule isomorphic to $W_z^{res}(n)$. The
quotient is spanned by $\left\{ t'_0, \ldots, t'_r \right\}$ and is
isomorphic to $W_z^{res}(r)$. It is therefore clear that $T_z(n)$ has a Weyl
filtration. Note that if $0 \leq n \leq \ell -1$, $T_z(n) = W_z^{res}(n) = W_{z}^{res}(n)$.

Given this definition, we can compute the quantum dimension of $T_z(n)$: 

\begin{equation}
    \dim_q(T_z(n)) = \begin{cases} [n+1]_z & \text{ if $0 \leq n \leq \ell-1$} \\
                                   [n+1]_z + [2\ell-n-1]_z = 0 &\text{ if $\ell \leq n < 2\ell - 2$}
                     \end{cases}
\end{equation}
so we have that $\dim_q(T_z(n)) = 0$ for $n < \ell-1$. In fact, \cite{Andersen1992} shows that 
\begin{prop}
$\dim_q(T_z(n)) \neq 0$ if and only if $n < \ell - 1$
\end{prop}

which gives us $\dim_q(T_z(n))$ for all $n$.

A useful corollary of the above proposition is the following. It gives a
general form for the decomposition of any two tilting modules.

\begin{corollary}
\label{theorem:tensortilting}
    For any tilting modules $T_1, T_2$, there is a tilting module $Z$ such that 

    \begin{equation}
            T_1 \otimes T_2 \simeq \left(\bigoplus_{k=0}^{\ell - 2} W_z^{res}(k)^{\otimes m_k}\right) \oplus Z
    \end{equation}
    where $m_k \in \mathbb{Z}$, and $\dim_q(Z) = 0$
\end{corollary}
\begin{proof}
\begin{equation}
    T_1 \otimes T_2 \simeq \bigoplus_{k=0}^{\infty} T_z(k)^{\otimes n_k} =
    \bigoplus_{k=0}^{\ell-2} W_z^{res}(k)^{\otimes n_k} \oplus
    \bigoplus_{k=\ell-1}^{\infty} T_z(k)^{\otimes n_k}
\end{equation}

Take $Z = \displaystyle\bigoplus_{k=\ell-1}^{\infty} T_z(k)^{\otimes n_k}$.
\end{proof}

We can now give a decomposition of $W_z^{res}(n) \otimes W_z^{res}(m)$ into
indecomposable modules for $0 \leq n,m \leq \ell-1$.  This tensor product will
not decompose as a direct sum of irreducible modules in general as with the
irreducible representations of $\sll(2)$. However, the crucial point is that it
\emph{does} decompose as a direct sum of irreducible modules and tilting
modules $T$ with $\dim_q(T) = 0$. This means if we can ``quotient out'' by the
tilting modules with quantum dimension zero, then any tensor product will
decompose as a direct sum of irreducible modules as with representations of
$\sll(2)$.

\begin{prop}
\label{theorem:decomposition}
Suppose $0 \leq m,n \leq \ell - 1$. Then

\begin{equation}
W_z^{res}(n) \otimes W_{z}^{res}(m) \simeq \bigoplus_{i=|n-m| \atop i + n + m \text{ even}}^{n+m} W_z^{res}(i) 
\end{equation}

if $n+m \leq \ell - 2$, and 

\begin{equation}
W_z^{res}(n) \otimes W_{z}^{res}(m) \simeq \bigoplus_{i=|n-m| \atop i + n + m \text{ even}}^{2\ell - 4 - n - m} W_z^{res}(i) 
                                    \oplus \bigoplus_{i = \ell - 1 \atop i + n + m \text{ even}} ^{n + m} T_z(i)
\end{equation}

for $\ell - 1 \leq n + m < 2\ell - 2$
\end{prop}


\section{Construction of the MTC}
\label{MTC-construction}

We will now quotient the category $\tilt$ by those tilting modules with quantum
dimension zero to obtain an MTC. With this aim, we make the following
definitions:

\begin{defn}
    A morphism $f: V \to W$ is \emph{negligible} if $\tr_q(fg) = \tr_q(gf) = 0$ for any $g: W \to V$
\end{defn}

\begin{defn}
A module $T$ is \emph{negligible} if all its endomorphisms are negligible
\end{defn}

The following proposition characterizing negligible tilting modules follows
from \ref{prop:tiltingclosed} and \ref{theorem:tensortilting}.
\begin{prop}
A tilting module $T$ is negligible if and only if it has quantum dimension
zero. 
\end{prop}

Define the category $\mathcal{C}^\text{int}$ be the category with objects tilting modules and morphisms 

    \begin{equation}
        \Hom(V,W) = \Hom_{\tilt}(V,W) / \text{negligible morphisms}
    \end{equation}

Prop \ref{theorem:decomposition} gives us that $\tilt$ is semisimple,
with simple objects 
\begin{equation}
\left\{ W_z^{res}(0), \ldots, W_z^{res}(\ell-2) \right\}
\end{equation}

The $\left\{ W_z^{res}(0), \ldots, W_z^{res}(\ell-2) \right\}$ are all
nonisomorphic because they all have different quantum dimensions.

The fusion rules in $\mathcal{C}^{\text{int}}$ are given by

\begin{equation}
    W_z^{res}(m) \otimes W_z^{res}(n) \simeq \sum_i N_{mn}^i W_z^{res}(i)
\end{equation}

where 

\begin{equation}
    N_{mn}^i = \begin{cases} 1 \qquad \text{ if } |m-n| \leq i \leq m+n, i \leq 2k - (m+n), i + m + n \in 2 \mathbf{Z} \\
                             0 \qquad \text{ else } 
               \end{cases}
\end{equation}

$\mathcal{C}^\text{int}$ is also a ribbon category, since $\tilt$ is.

To summarize, 
\begin{prop}
\begin{enumerate}
    \item  $\mathcal{C}^\text{int}$ is a ribbon category
    \item Any object $T$ in $\mathcal{C}^\text{int}$ is isomorphic to a direct sum of Weyl modules.
    \item $\mathcal{C}^\text{int}$ is a semisimple abelian category. 
    \item $\dim_{\mathcal{C}^\text{int}} T > 0$ for every $T \not\simeq 0$
\end{enumerate}
\end{prop}

All that remains to show that it is a modular tensor category is to check that
the $S$-matrix (defined in Equation \ref{equation:s-matrix}) is invertible. A
proof of this can be found in \cite{Kirillov2001}.
