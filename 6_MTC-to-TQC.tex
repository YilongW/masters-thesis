\section{The Fibonacci anyon}
The simplest universal model of TQC, and so the most likely candidate for
implementing TQC, is the Fibonacci anyon. We will introduce the Fibonacci
anyon, and show how this model comes from the modular tensor category obtained
from $U_q(\sll(2))$ for $q=e^{2\pi i/5}$ in some detail. We will also explain a
fairly simple way to simulate quantum circuits with Fibonacci anyons.

In the Fibonacci anyon model there are 2 types, labelled $\mathbf{1}$, $\tau$.
$\mathbf{1}$ is the trivial type. The fusion rules are as follows: 


\begin{itemize}
    \item $\tau \odot \tau = I \oplus \tau$
    \item $\tau \odot I = \tau$
    \item $I \odot I = I$
\end{itemize}

As with all TQC theories, this takes place in a MTC which we call $\Fib$. The
MTC $\Fib$ has two isomorphism classes of simple objects labelled by
$\mathbf{1}$, $\tau$. $\mathbf{1}$ is the tensor unit. We will denote the
tensor product by $\odot$. The $\oplus$ in the first fusion rule above should
be taken to mean that if two anyons of type $\tau$ fuse, then the result is an
anyon of type either $\mathbf{1}$ or $\tau$ (or a superposition). If we compute
$\tau \odot (\tau \odot \tau)$, we get
\begin{equation}
    (\tau \odot \tau) \odot \tau = (\tau \oplus I) \odot \tau = (\tau \odot \tau) \oplus (I \odot \tau) = (\tau \oplus I) \oplus \tau = 2\tau \oplus I
\end{equation}

This means that when $\tau,\tau,\tau$ fuse, the result is either $\tau$ in one of
two possible ways, or $\mathbf{1}$ (in one way). What do we mean by ``in one of two
possible ways?'' Well, there are 3 fusion possibilities for 3 anyons, which we
can draw as follows: 

\newcommand{\fusiondiagram}[2]
{
\begin{pspicture}(0,-0.99125)(2.841875,0.99125)
\psline[linewidth=0.01cm]{->}(0.2209375,0.63)(0.65,0.2)
\psline[linewidth=0.01cm]{->}(1.4209375,0.63)(1.05,0.2)
\psline[linewidth=0.01cm]{->}(0.8209375, -0.12)(1.2209375,-0.57125)
\psline[linewidth=0.01cm]{->}(2.6209376,0.63)(1.5,-0.57125)
\rput(1.531875,0.79875) {$\tau$}
\rput(0.111875,0.79875) {$\tau$}
\rput(2.691875,0.79875) {$\tau$}
\rput(0.85, 0.1)    {$#1$}
\rput(1.35,-0.7)        {$#2$}
\end{pspicture} 
}
\begin{center}
\fusiondiagram{\mathbf{1}}{\tau}
\fusiondiagram{\tau}{\mathbf{1}}
\fusiondiagram{\tau}{\tau}
\end{center}

So in the first and third case the end result is $\tau$, and in the second case
$\mathbf{1}$. There are therefore two different ways of obtaining a final
result of $\tau$: the first two anyons can fuse with a result of either $\tau$
or $\mathbf{1}$.


\subsection{Simulating quantum circuits with Fibonacci anyons}

In the previous section we saw that three anyons of type $\tau$ are fused,
there are 3 possibilities for the result: 

\begin{center}
\fusiondiagram{\mathbf{1}}{\tau}
\fusiondiagram{\tau}{\mathbf{1}}
\fusiondiagram{\tau}{\tau}
\end{center}

Two of them have the final result $\tau$. This is equivalent to saying that
$(\tau \odot \tau) \odot \tau \simeq 2\tau \oplus \mathbf{1}$. We therefore
have that 

\begin{equation}
    \Hom(\tau, (\tau \odot \tau) \odot \tau) = \mathbb{C}^2
\end{equation}

Our qubits will correspond to elements of $\Hom(\tau, (\tau \odot \tau) \odot
\tau)$, which we will write as fusion diagrams: let 

\begin{equation}
\ket{0} = \fusiondiagram{\mathbf{1}}{\tau}
\ket{1} = \fusiondiagram{\tau}{\tau}
\end{equation}
\todo{fix formatting}

Intuitively speaking, $\ket{i}$ is a triplet of $\tau$ anyons that fuses in the
way indicated in the diagram above.

The central operation in TQC is that of braiding: computations correspond to
braids. Since \Fib is a modular tensor category, for each pair of objects $V,W$
there is a braiding isomorphism $\sigma_{V,W}: V \odot W \to W \odot V$.

In particular, $\sigma_{\tau,\tau}: \tau \odot \tau \to \tau \odot \tau$ is an
automorphism of $\tau \odot \tau$. We can depict it as a braid as follows:

\begin{center}
\todo{make a picture of the braid here}
\end{center}

This graphical notation can be formalized in a graphical calculus for ribbon
categories, but we do not have time to discuss this here. See \cite{Kirillov}. 

$\sigma_{\tau,\tau} \odot \id_\tau$ acts on $\Hom(\tau, (\tau \odot \tau) \odot \tau)$ by composition: 

\begin{equation}
f \mapsto (\sigma_{\tau,\tau} \odot \id_\tau) \circ f
\end{equation}

Since $\Hom(\tau, (\tau \odot \tau) \odot \tau) \simeq \mathbb{C}^2$ , we can
write this as a $2 \times 2$ matrix called the \emph{$R$-matrix}. This can be
either solved for using the fusion rules and pentagon equation or calculated
using the $R$-matrix from \ref{section:braiding}. As in any monoidal category, 

As in any monoidal category, $(\tau \odot \tau) \odot \tau \simeq \tau \odot
(\tau \odot \tau)$ via the associativity isomorphism $\alpha_{\tau,\tau,\tau}$.
This gives an isomophism between $\Hom(\tau, (\tau \odot \tau) \odot \tau)$ and
$\Hom(\tau, \tau \odot (\tau \odot \tau))$
\section{Universality}
