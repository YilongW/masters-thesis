In this chapter, we explain the relationship between TQC and modular tensor
categories arising from quantum groups. We will focus mainly on the category
$\Fib$ of Fibonacci anyons, which is the simplest MTC which is universal for
quantum computing. 

We will introduce $\Fib$ and discuss in some detail how to simulate the quantum
circuit model in $\Fib$. We will then discuss Hormozi's constructions of a
universal gate set for quantum computing in $\Fib$. We will also discuss
Rowell's classification of MTCs of small order and state a conjecture of
Rowell's about universality in MTCs.

\section{The Fibonacci anyon}
The simplest universal model of TQC, and so a good candidate for implementing TQC, is the Fibonacci anyon. Here we introduce the Fibonacci anyon
and explain how to simulate quantum circuits using Fibonacci anyons.

The category $\Fib$ of Fibonacci anyons is a modular tensor category which is a
subcategory of $\mathcal{C}^{int}(\sll(2),5)$: its objects are representations
of $U_z^{res}(\sll(2))$ for $z$ a $5^{th}$ root of unity. The simple objects in 
$\Fib$ are $V_0, V_2$. We will refer to them as $\one$ and $\tau$
respectively, $\one$ being the tensor unit and trivial anyon type. The
fusion rules in $\Fib$ can be computed easily and are as follows:

\begin{itemize}
    \item $\tau \odot \tau = I \oplus \tau$
    \item $\tau \odot I = \tau$
    \item $I \odot I = I$
\end{itemize}
Recall that the $\oplus$ here means that when $\tau$, $\tau$ fuse, the result
is a superposition of anyons of type $1$ and $\tau$.  The reason they are
called Fibonacci anyons is that $\tau^{\odot n} \simeq F_{n-1} \one \oplus F_{n+2}
\tau$, where $F_n$ is the $n^{th}$ Fibonacci number.

We will now explain a simple way to simulate quantum circuits using Fibonacci
anyons. To simulate quantum circuits, we need

\begin{itemize}
\item Qubits
\item Systems of multiple qubits
\item A way to do measurement
\item Unitary gates
\item A universal gate set
\end{itemize}

First we construct qubits. Consider a triplet $(\tau,\tau,\tau)$ of Fibonacci
anyons. A simple computation shows that 

\begin{equation}
(\tau \odot \tau) \odot \tau \simeq 2\tau \oplus 1
\end{equation}

This means that there are $3$ fusion possibilities for $3$ anyons with type $\tau$:

\newcommand{\fusiondiagram}[2]
{
\begin{pspicture}(0,-0.99125)(2.841875,0.99125)
\psline[linewidth=0.01cm]{->}(0.2209375,0.63)(0.65,0.2)
\psline[linewidth=0.01cm]{->}(1.4209375,0.63)(1.05,0.2)
\psline[linewidth=0.01cm]{->}(0.8209375, -0.12)(1.2209375,-0.57125)
\psline[linewidth=0.01cm]{->}(2.6209376,0.63)(1.5,-0.57125)
\rput(1.531875,0.79875) {$\tau$}
\rput(0.111875,0.79875) {$\tau$}
\rput(2.691875,0.79875) {$\tau$}
\rput(0.85, 0.1)    {$#1$}
\rput(1.35,-0.7)        {$#2$}
\end{pspicture} 
}

\newcommand{\fusiondiagramright}[2]
{
\begin{pspicture}(0,-1)(2.84,1)
\psline[linewidth=0.01cm]{->}(2.62,0.63)(2.19,0.2)
\psline[linewidth=0.01cm]{->}(1.42,0.63)(1.79,0.2)
\psline[linewidth=0.01cm]{->}(2.02,-0.12)(1.62,-0.57)
\psline[linewidth=0.01cm]{->}(0.22,0.63)(1.34,-0.57)
\rput(1.31,0.79875) {$\tau$}
\rput(2.73,0.79875) {$\tau$}
\rput(0.15,0.79875) {$\tau$}
\rput(1.99, 0.1)    {$#1$}
\rput(1.49,-0.7)        {$#2$}
\end{pspicture} 
}

\begin{center}
\label{fusionleft}
\fusiondiagram{\one}{\tau}
\fusiondiagram{\tau}{\one}
\fusiondiagram{\tau}{\tau}
\end{center}

Two of them have the final result $\tau$. We use this fact to define our qubits
to be triplets of $\tau$ anyons with total charge $\tau$. We saw in
\ref{fusionleft} that the first two qubits can fuse to be either $\tau$ or
$\one$: denote these two possibilities by $((\tau,\tau)_\one,\tau)_\tau$ and
$((\tau,\tau)_\tau,\tau)_\tau$.

In $\Fib$, these are represented by eleemnts of $\Hom(\tau, ((\tau \odot
\tau)\odot \tau)) \simeq \Hom(\tau, 2\tau \oplus \one) \simeq \mathbb{C}^2$. 
Define 

\begin{align}
\ket{0} = ((\tau,\tau)_\one , \tau)_\tau \\
\ket{1} = ((\tau,\tau)_\tau , \tau)_\tau 
\end{align}

$\ket{0}$ and $\ket{1}$ are elements of $\Hom(\tau, ((\tau \odot \tau)\odot
\tau))$. We will call $\operatorname{span}\left\{ \ket{0}, \ket{1} \right\}$
the \emph{computational space}. There is also a \emph{noncomputational space}
spanned by $\ket{\text{NC}} = ((\tau,\tau)_\tau,
\tau)_\one$. One of the principal issues with this simulation is that of
``leakage'': some operations take us out of the computational space. 

Performing measurements in the $\left\{ \ket{0}, \ket{1} \right\}$ basis is
straightforward: just fuse the first two anyons and measure the result, then
fuse the remaining two and measure again.

We have explained how 1-qubit systems are simulated. Multiqubit systems are
similar: for $n$ qubits, we take $n$ sets of $3$ anyons: $\tau^{\odot 3n}$.
For $\ket{\varphi},\ket{\psi}\in \Hom(\tau,(\tau \odot \tau) \odot \tau)$,
their tensor product $\ket{\varphi} \odot \ket{\psi} \in \Hom(\tau \odot
\tau, (\tau)^{\odot 6}) \simeq \mathbb{C}^{13}$. 

The vectors $\left\{ \ket{i}\odot \ket{j} : 0 \leq i,j\leq 1 \right\}$
therefore span a 4-dimensional subspace of $\Hom(\tau \odot \tau, (\tau)^{\odot
6}) \simeq \mathbb{C}^{13}$. We again refer to this space as the
\emph{computational space} $V$. As $n$ gets larger, more and more of the space
is the noncomputational space. 

We now turn to the problem of defining gates. $\Fib$ is braided, so for any $n$
and object $V$ in $\Fib$, $B_n$ acts on $\Hom(V, \tau^{\odot n})$ by
composition: 

\begin{equation}
f \mapsto \sigma \circ f
\end{equation}

This gives a representation of the braid group on the hom sets $\Hom(\one,
\tau^{\odot n})$. In fact, it turns out that this representation is unitary.
Braids therefore act unitarily on our qubits, and we will use them as our
quantum gates. How do these braids act? The generator $\sigma_1$ of $B_3$,

\begin{center}
    % drawing of a braid
    \scalebox{1} % Change this value to rescale the drawing.
    {
    \begin{pspicture}(0,-0.9615033)(2.421875,0.9815033)
    \pscustom[linewidth=0.02cm]
    {
    \newpath
    \moveto(0.1609375,0.558031)
    \lineto(0.1309375,0.43743896)
    \curveto(0.1159375,0.37714294)(0.1159375,0.2486859)(0.1309375,0.1805255)
    \curveto(0.1459375,0.112365104)(0.2159375,-0.00822694)(0.2709375,-0.06065797)
    \curveto(0.3259375,-0.113089)(0.4759375,-0.21008728)(0.5709375,-0.25465393)
    \curveto(0.6659375,-0.2992206)(0.8459375,-0.40670472)(0.9309375,-0.4696222)
    \curveto(1.0159374,-0.53253967)(1.1359375,-0.6609961)(1.1709375,-0.726535)
    \curveto(1.2059375,-0.7920746)(1.2459375,-0.87334293)(1.2609375,-0.9205308)
    }
    \pscustom[linewidth=0.02cm]
    {
    \newpath
    \moveto(0.5809375,-0.1340619)
    \lineto(0.7909375,-0.05541505)
    \curveto(0.8959375,-0.01609193)(1.0309376,0.10187865)(1.0609375,0.1805255)
    \curveto(1.0909375,0.25917235)(1.1359375,0.39811522)(1.1809375,0.5790033)
    }
    \pscustom[linewidth=0.02cm]
    {
    \newpath
    \moveto(0.4409375,-0.25989687)
    \lineto(0.2909375,-0.40146118)
    \curveto(0.2159375,-0.47224304)(0.1159375,-0.6216724)(0.0909375,-0.70031923)
    \curveto(0.0659375,-0.77896667)(0.0509375,-0.8785858)(0.0809375,-0.9415033)
    }
    \rput(0.111875,0.7890033){$\tau$}
    \rput(1.151875,0.7890033){$\tau$}
    \rput(2.271875,0.7890033){$\tau$}
    \psline[linewidth=0.02cm](2.2609375,0.59900326)(2.2609375,-0.90099674)
    \end{pspicture} 
    }
\end{center}

acts on $\Hom(\tau, ((\tau \odot \tau) \odot \tau))$ via the unitary matrix

\begin{equation}
\begin{pmatrix}
e^{\frac{-4\pi i}{5}} & 0 \\
0 & e^{\frac{-2\pi i}{5}}  \\
\end{pmatrix}
\end{equation}

with respect to the basis $\left\{ \ket{0}, \ket{1} \right\}$. This can be
computed either using the ``universal $R$-matrix'' in \ref{section:braiding} or
by solving for it using the fusion rules and the axioms for a braided monoidal
category (see \cite{Panangaden}).

$\Hom(\tau, \tau^{\odot 3})$ also has the basis \{ $(\tau,
(\tau,\tau)_\one)_\tau$, $(\tau, (\tau,\tau)_\tau)_\tau$ \}. It is obtained
from the basis $\left\{ \ket{0}, \ket{1} \right\}$ by composing with the
associativity isomorphism $\alpha_{\tau,\tau,\tau}$. The change of basis matrix
is called the \emph{$F$-matrix}. It is

\begin{equation}
\begin{pmatrix}
\phi^{-1} & \sqrt(\phi^{-1}) \\
\sqrt(\phi^{-1}) & -\phi^{-1} \\
\end{pmatrix}
\end{equation}

where $\phi$ is the golden ratio. The $F$-matrix is also unitary.

The $R$ and $F$-matrices allow us to compute the $2 \times 2$ matrix
corresponding to each element of $B_3$ in this representation: the other
generator $\sigma_2$ of $B_3$ acts as $F^{-1}RF$. In fact, the $R$ and $F$
matrices are enough to compute the matrix associated to any element of the
$B_n$ acting on $\Hom(\one, \tau^{\odot n})$ or $\Hom(\tau, \tau^{\odot n})$.


\section{Universality and implementations}

We have explained the standard way to simulate the circuit model using
Fibonacci anyons. Freedman et al show in \cite{Freedman2000} that the image of
$B_6$ in its representation of $Hom(\tau \oplus \tau, \tau^{\odot 6}) \simeq
\mathbb{C}^5 \oplus \mathbb{C}^8$ is dense in $SU(5) \oplus SU(8)$. 

There are two issues: firstly it is necessary to find braids $\sigma$ such that
$\rho(\sigma)$ (at least approximately) takes vectors in the computational
space to vectors in the computational space. Once this is solved, we need to be
implement arbitrary unitary gates on the computational space.

This theorem does not however give a means of, given a quantum gate, finding a
braid whose image approximately realizes this gate.

This problem has been tackled by Hormozi et al in \cite{Hormozi2007} and \expand{whoever}.  In this section we give an account of how these problems have been handled.

the most \todo{naive} solution to the problem of ``topological quantum
compiling'' is brute force search: try braids until you find one where the
image is close enough to the gate required.

In \cite{Hormozi2007}, Hormozi et al

\begin{enumerate}
\item reduce the search space by only considering a limited subset of braids
\item 
\end{enumerate}
