\section{Overview of TQC}

First we'll discuss the basic idea behind TQC.

One of the fundamental problems with implementations of quantum computer is
that it is very hard to implement quantum gates accurately. Kitaev in 1997
propoesd an implementation of QC which exploits topological properties to make
this easier.

In our basic setup, we have a set of identical anyons which have some joint ``state''.

\begin{center}
    (picture here?)
%    $\circ \hspace{1in} \circ\hspace{1in} \circ\hspace{1in} \circ$
\end{center}

A computation consists of exchanging the anyons in some way. If we draw such an
exchange in 2+1 dimensions, we obtain a braid. 

\begin{center}
    (picture here?)
\end{center}

The key point here is that in this model the change in state of the anyons
depends \emph{only} on the topological class of the braid, so minor
pertubations of a particle's path won't affect the result of the computation at
all. The process is thus inherently resistant to errors. 



\section{The Fibonacci anyon}

In this section we will describe the Fibonacci anyon, a simple universal model
of topological quantum computing, and explain how it arises from the category
of representations of $U_q(\sll(2))$ in some detail. 

This model consists of 2 types of anyons, with labels $1, \tau$. The permitted
operations are 

\begin{itemize}
    \item exchanging 2 anyons
    \item fusing 2 anyons
\end{itemize}

The category $\Fib$ is a modular tensor category, with two isomorphism classes
of simple objects labelled by $I,\tau$, $I$ being the tensor unit. We will
denote the tensor product by $\odot$, which represents fusion.

The fusion rules are as follows:

\begin{itemize}
    \item $\tau \odot \tau = I \oplus \tau$
    \item $\tau \odot I = \tau$
    \item $I \odot I = I$
\end{itemize}

The $\oplus$ in the first rule should be taken to mean ``if two anyons of type
$\tau$ fuse, then the result is either $I$ or $\tau$. $\odot$ distributes over
$\oplus$, so for example we have 

\begin{equation}
    (\tau \odot \tau) \odot \tau = (\tau \oplus I) \odot \tau = (\tau \odot \tau) \oplus (I \odot \tau) = (\tau \oplus I) \oplus \tau = 2\tau \oplus I
\end{equation}

This means ``when $\tau,\tau,\tau$ fuse, the result is either $\tau$ in one of
2 possible ways, or $I$ (in one way)''. What do we mean by ``in one of two
possible ways?'' Well, if we draw dragrams for the fusion possibilites, we
obtain:

\begin{center}
    (draw some fusion diagrams here)
    \todo{make these diagrams}
\end{center}

\subsection{Where are the qubits?}

When three anyons of type $\tau$ are fused, there are 3 possibilities for the result: 

\begin{center}
    (draw some fusion diagrams here)
    \todo{make these diagrams}
\end{center}

2 of them have the final result $\tau$. Another way of putting this is 

\begin{equation}
    \Hom(\tau, \tau^{\odot 3}) = \mathbb{C}^2
\end{equation}



\section{Universality}
