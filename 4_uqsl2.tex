Models of topological quantum computation are usually presented as modular
tensor categories for reasons explained in the introduction (or somewhere: I
will give you a few paragraphs for that later).  One important source
of modular tensor categories, and hence possible models of topological quantum
computing, is quantum groups.  The category of representations of a quantum
group does not form a modular tensor category directly, however, after suitable
constructions, a modular tensor category can be constructed from the category
of representations of a quantum group.  The main goal of this chapter is to
introduce quantum groups.

There is no agreed upon definition of quantum group, rather, there are a
number of examples that are called quantum groups.  The concept arose in
inverse scattering theory~\cite{Faddeev79} and also in earlier work by
Yang~\cite{Yang67} on the factorizability of certain transfer matrices in
statistical mechanics.  This was later expanded by Baxter~\cite{Baxter82}
and led to the Yang-Baxter equation.  The abstract presentation of these
ideas is due to M. Jimbo~\cite{Jimbo85} and V. Drinfeld~\cite{Drinfeld86}.
Essentially quantum groups are certain kinds of Hopf algebras and are, in
general, neither commutative nor cocommutative.  The ones of so-called
Drinfeld-Jimbo type arise as deformations of the universal enveloping
algebra of Lie algebras.

In this chapter we will define and describe in detail the representation theory
of the quantum group $U_q(\sll(2))$ at a root of unity, which arises as a
deformation of the Lie algebra $\sll(2)$. We will also give a braiding on the
category of representations and discuss briefly the general case, of
$U_q(\frakg)$ for any semisimple Lie algebra $\frakg$, an exposition of which
can be found in \cite{CP}.

\section{The quantum group $U_q(\sll(2))$}
\label{UqSL2}


The quantum group $U_q(\sll(2))$ is a Hopf algebra which is a deformation of the
universal enveloping algebra $U(\sll(2))$ defined in \ref{example:UEA}.  In this
section we will define the algebra $U_q(\sll(2))$, explain its relationship to 
$U(\sll(2))$, and give the Hopf algebra structure on $U_q(\sll(2))$.

Recall that the algebra $U(\sll(2))$ is generated by $X,Y,H$ with relations

\begin{align}
[X,Y] & = H \\ 
[H,X] &= 2X \\
[H,Y] &= -2Y \\
\end{align}

$X$ and $Y$ act as raising and lowering operators on the representations of
$U(\sll(2))$: in the representation $V_n = \operatorname{span}\left\{ v_0,
\ldots, v_n \right\}$:

\begin{align*}
    H v_i &= (n - 2i) v_i \\
    Y v_i &= (i+1)v_{i+1}   \text{ for $i < m$}  \\
    X v_i &= (n-i+1)v_{i-1} \text{ for $i > 0$} \\
\end{align*}

\begin{defn}
\label{Uqsl2Def}
Let $q$ be an indeterminate. $U_q(\sll(2))$ is the associative algebra over
$\mathbb{C}(q)$ with generators $X^+,X^-,K, K^{-1}$ and relations 

\begin{align}
    KK^{-1} &= 1 = K^{-1}K \\
    K X^+ K^{-1} &= q^2  X^+  \\
    KX^-K^{-1} &= q^{-2} X^- \\
    [ X^+ ,X^-] &= \frac{K - K^{-1}}{q - q^{-1}} \label{equation:hadamard}
\end{align}
\end{defn}

In general $ X^+ $ and $X^-$ should be thought of as the analogs of $X$ and $Y$
in $U(\sll(2))$. This last equation can be thought of as replacing 

\begin{equation}
[X,Y] = H
\end{equation}

with the ``quantum version''

\begin{equation}
[X^+,X^-] = [H]_q = \frac{q^H - q^{-H}}{q - q^{-1}}
\end{equation}

Of course $q^H$ does not make sense here, so we introduce a new element $K=q^H$
and obtain \ref{equation:hadamard}. $X^+$ and $X^-$ act as raising and lowering
operators in representations of $U_q(\sll(2))$ the same way as $X,Y$ in $U(\sll(2))$. To
see the relationship between $U_q(\sll(2))$ and $U(\sll(2))$, we will make use
of an alternate presentation of $U_q(\sll(2))$.  Define $U'_q(\sll(2))$ to be
the associative algebra with generators $ X^+ ,X^-,L,K,K^{-1}$ and relations

\begin{align}
    KK^{-1} &= 1 &  K^{-1}K  &=1 \\
    KEK^{-1} &= q^2 E & KFK^{-1} &= q^{-2} F \\
    [E,F] &= L  & (q - q^{-1})L &= K-K^{-1} \\
    [L,E] &= q(EK + K^{-1}E) & [L,F] &= -q^{-1}(FK + K^{-1}F)
\end{align}

A straightforward calculation shows that $U_q(\sll(2)) \simeq U_q'(\sll(2)$. If
we define the algebra $U_1'(\sll(2))$ over $\mathbb{C}$ having the same
generators and relations as $U_q'(\sll(2))$ with $q$ replaced by 1, then it is
easily seen that $U_1'(\sll(2)) /(K-1) \simeq U(\sll(2))$ by sending $X^+
\mapsto X, X^- \mapsto Y, L \mapsto H, K \mapsto 1$. This gives some
justification for calling $U_q(\sll(2))$ a deformation of $U(\sll(2))$.

We can define a Hopf algebra structure on $U_q(\sll(2))$ as follows:

% from Kassel, p. 140
The comultiplication and counit are given by
\begin{align}
    \Delta(X^+) &= 1 \otimes  X^+  +  X^+  \otimes K &  \Delta(X^-) &= K^{-1} \otimes X^- + X^- \otimes 1 \\
    \Delta(K) &= K \otimes K &  \Delta(K^{-1}) &= K^{-1} \otimes K^{-1}\\ 
    \varepsilon( X^+ ) &= \varepsilon(X^-) = 0 &  \varepsilon(K) &= \varepsilon(K^{-1}) = 1
\end{align}

and the antipode by
\begin{align}
    S(X^+) &= - X^+ K^{-1} & S(X^-)      &= -KX^- \\
    S(K) &= K^{-1}   & S(K^{-1}) &= K 
\end{align}
It is straightforward to check that this defines a Hopf algebra structure. This
looks quite similar to the Hopf algebra structure on $U(\sll(2))$ defined in
\ref{example:UEA}, and in fact the isomorphism $U_1'(\sll(2)) /(K-1)
\simeq U(\sll(2))$ is an isomorphism of Hopf algebras.

\section{Representations of $U_q(\sll(2))$}
\label{section:RepsOfUq}

The representation theory of $U_q(\sll(2))$ bears a striking resemblance to
that of $\sll(2)$. In particular, in $U_q(\sll(2))$:

\begin{enumerate}
    \renewcommand{\labelenumi}{\roman{enumi})}
    \item Every finite dimensional representation is completely reducible
    \item The eigenvectors of $K$ form a basis for any irreducible representations. We call the eigenvalues of $K$ \emph{weights}.
    \item Each irreducible representation has a highest weight vector
    \item The irreducible representations are classified by highest weights much as in $U(\sll(2))$.
\end{enumerate}

The proofs of these facts are also more or less analogous to those in
$U(\sll(2))$, though not always in a straightforward way. For proofs of i),
ii), see \cite{Kassel1994}. We will sketch the classification of the
irreducible representations of $U_q(\sll(2))$ here.

\begin{claim}
\label{claim:HighestWeight}
    Let $M$ be a finite dimensional $U_q(\sll(2))$-module. Then there is an
    eigenvector $v_0 \in M$ of $K$ such that $ X^+ v_0 = 0$. We call $v_0$ a
    \emph{highest weight vector}.
\end{claim}

\begin{proof}

    Suppose not. Let $v$ be an eigenvector of $K$ with eigenvalue $\lambda$.
    $K X^+ = q^2  X^+ K$, so it follows that $(X^+)^i v$ is an
    eigenvector of $K$ with eigenvalue $q^{2i}\lambda$. Each value
    $q^{2i}\lambda$ is different, therefore $K$ has infinitely many
    eigenvectors. 

    This is impossible, so some $(X^+)^i v$ must be zero. Let $i$ be the
    smallest such, and choose $v_0 = (X^+)^{i-1} v$.
\end{proof}

We now have that any irreducible representation of $U_q$ is generated by a
highest weight vector. We can now show how to classify the irreducible
representations. 

For any $\lambda \in \mathbb{C}(q), \lambda \neq 0$ there is an infinite
dimensional $U_q(\sll(2))$-module $M(\lambda)$ called a \emph{Verma module}
with basis $v_0, v_1, v_2, \ldots$ such that for any $i$,

\begin{align}
    Kv_i &= \lambda q^{n-2i}v_i \\
    X^-v_i &= v_{i+1} \\
     X^+ v_i &= \begin{cases} 0 & \text{if $i = 0$} \\
                          [i]_q \frac{\lambda q^{1-i} - \lambda^{-1}q^{i-1}}{q - q^{-1}}v_{i-1} & \text{else}
            \end{cases}
\end{align}

To see that this a $U_q(\sll(2))$-module, we can note that 

\begin{equation}
    M(\lambda) \simeq U_q(\sll(2))/(U_q(\sll(2)) X^+  + U_q(\sll(2))(K-\lambda))
\end{equation}

with $v_i$ being the coset of $(X^-)^i$. It follows from this that if $M$ is a
$U_q(\sll(2))$-module and $v \in M$ is such that $X^+ v = 0$ and $Kv = \lambda v$, there
is a unique homomorphism $\varphi: M(\lambda) \to M$ where $\varphi(v_0) = v$.
If $M$ is simple, then this homomorphism must be  surjective and it follows
that $M$ is a quotient of $M(\lambda)$. 

We now have that 
\begin{claim}
    Any finite dimensional simple $U_q(\sll(2))$-module is a quotient of $M(\lambda)$.
\end{claim}

\begin{claim}
    $M(\lambda)$ is simple for $\lambda \neq \pm q^n$
\end{claim}

\begin{proof}
    See \cite{Jantzen1995}.
\end{proof}

For $\lambda = \pm q^{n}$, $M(\lambda)$ has one nontrivial submodule spanned by the $v_i$ with $i \geq n+1$.
This leads us to the following characterization (details can be found in \cite{Jantzen1995}):


For each integer $n \geq 0$, there are two unique (up to isomorphism)
representations of $U_q(\sll(2))$ of dimension $n+1$. 
The representations are $V_q(n)$ with basis $\left\{ v^+_0, \ldots, v^+_n
\right\}$, and $V_q^-(n)$ with basis $\left\{ v^-_0, \ldots, v^-_n \right\}$
with action defined by 

% from Kassel, p. 101
\begin{align*}
    &K v^+_i = q^{n-2i} v^+_i  &
    &K v^-_i = -q^{n-2i} v^-_i \\
    &X^- v^+_i = \begin{cases} [i+1]_qv_{i+1}& \text{ if $i < n$} \\ 0& \text{ if $i = n$} \end{cases} &
    &X^- v^-_i = \begin{cases} [i+1]_qv'_{i+1}& \text{ if $i < n$} \\ 0& \text{ if $i = n$} \end{cases} \\
    & X^+  v^+_i = \begin{cases} 
                     [n-i+1]_qv_{i-1}& \text{ if $i > 0$} \\ 
                    0& \text{ if $i = 0$} 
             \end{cases} &
    & X^+  v^-_i = \begin{cases} 
                    -[n-i+1]_q v'_{i-1}& \text{ if $i > 0$} \\ 
                    0& \text{ if $i = 0$} 
             \end{cases}
\end{align*}

By the proof sketched above, these are the only representations of
$U_q(\sll(2))$.  Note that the representations $V_q^-(n)$ can be obtained by
$V_q(n)$ by tensoring with the one-dimensional representation $V_q^-(0)$:
essentially the difference from $U(\sll(2))$ here is that there are 2
nonisomorphic 1-dimensional representations. We can (and will) therefore
restrict our attention to the representations $V_q(n)$.

If we consider tensor products of these representations $V_q(n)$, we find
another similarilty with the representation theory of $\sll(2)$. It is well
known that the tensor product of two irreducible representations $V(n)$, $V(m)$
($n \geq m$) decomposes as follows:

\begin{equation}
V(n) \otimes V(m) \simeq V(n-m) \oplus V(n-m+2)  \oplus \cdots \oplus V(n+m-2)\oplus V(n+m)
\end{equation}

Exactly the same formula holds for the irreducible representations $V_q(n)$: for any $n \geq m \geq 0$

\begin{equation}
V_q(n) \otimes V_q(m) \simeq V_q(n-m) \oplus V_q(n-m+2)  \oplus \cdots \oplus V_q(n+m-2)\oplus V_q(n+m)
\end{equation}

\section{Setting $q$ to be a root of unity}

We are interested in constructing modular tensor categories from quantum groups
at roots of unity. In this section, we explain how to set the indeterminate $q$
to be a complex root of unity in $U_q(\sll(2))$. There are several different ways to
set $q$ to be a complex number $z$, which result in different algebras when $z$
is a root of unity. We give two such ways below. We will take $z$ to be an
$\ell^{th}$ root of unity, $\ell \geq 3$ odd.

The first and most obvious way is, for any $z \in \mathbb{C}$ such that $z ^2
\neq 0,1$, to set $q = z$ in the definition (\ref{Uqsl2Def}) of $U_q(\sll(2)$.
Denote the algebra obtained in this way by $U_z(\sll(2))$. This algebra is of
interest in its own right when $z$ is a root of unity (\cite{Jantzen1995}
discusses it in detail), but does not lead us to a modular tensor category.

The second way is to let $\mathcal{A} = \mathbb{Z}[q,q^{-1}]$ and define a
`restricted' $\mathcal{A}$-subalgebra $U_\mathcal{A}^{res}(\sll(2))$ to be the
$\mathcal{A}$-subalgebra of $U_q(\sll(2))$ generated by $ X^+$
,$X^-$,$K$,$K^{-1}$, $(X^+) ^{(n)}$, $(X^-)^{(n)}$, where


\begin{align}
     (X^+)^{(n)} = \frac{ (X^+)^n}{[n]^!_{q}} \\
    (X^-)^{(n)} = \frac{(X^-)^n}{[n]^!_{q}} 
\end{align}


Define 
\begin{equation}
    U_z^{res}(\sll(2)) = U_\mathcal{A}^{res}(\sll(2)) \otimes_A \mathbb{C}
\end{equation}

This is the algebra whose representations we will be interested in.  If $z$ is
not a root of unity, then $U_z^{res}(\sll(2))$ and $U_z(\sll(2))$ are
isomorphic. When $z$ is an $\ell^{th}$ root of unity, $\ell \geq 3$,

\begin{lemma}
    In $U_z^{res}(\sll(2))$, $ (X^+)^\ell = (X^-)^\ell = 0$ for all $i$
\end{lemma}
\begin{proof}
    $ (X^+)^\ell = [\ell]_{q}^!  (X^+)^{(\ell)} = 0$ as $[\ell]_{q}^! = 0$. Similarly for $(X^-)^\ell$.
\end{proof}

$(X^+)^\ell \neq 0$ in $U_z(\sll(2))$, so it is clear that these algebras are
not isomorphic when $q$ is a root of unity. The representation theory of
$U_z^{res}(\sll(2))$, discussed in the next section, will give us modular
tensor categories and therefore models of topological quantum computing.

\section{The Representation theory of $U_z^{res}(\sll(2))$}
\label{section:RepTheoryofResSL2}


Throughout this section $z$ will be assumed to be an $\ell^{th}$ root of unity, where $\ell \geq 3$ is odd.

The objects of the MTC we are interested in are irreducible representations of
$U_z^{res}(\sll(2))$. Unlike in the case of $U_q(\sll(2))$, not every
representation of $U_z^{res}(\sll(2))$ is completely reducible, and in general
the representation theory is quite complicated. To get an idea of why this is
the case, note that the argument in \ref{claim:HighestWeight} for why every
representation has a highest weight breaks down when $q=z$ is a root of unity.
However, the modular tensor category we wish to construct comes from a
subcategory of the category of representations, so it will be enough to
understand a limited subset of the representations. The most important
representations for us to understand are the \emph{Weyl modules}, which we will
define in this section. The objects of our MTC will be irreducible Weyl
modules. 

Define a representation of $U_z^{res}$ to be of \emph{type $\mathbf{1}$} if
$K^\ell = 1$. As in $U_q(\sll(2))$, any representation of $U_z^{res}$ is
isomorphic to the tensor product of a representation of type $\mathbf{1}$ with
a one-dimensional representation. We can therefore restrict our attention to
representations of type $\mathbf{1}$. 

Recall that in $U_q(\sll(2))$ the irreducible representations $V_q(n)$ are
indexed by integers $n \geq 0$. We will use these representations to define
representations of $U_z^{res}(\sll(2))$ as follows.

Let $V_q(n)$ be the irreducible $U_q(\sll(2))$-module defined in section
\ref{section:RepsOfUq}.  Let $V_\mathcal{A}^{res}(n)$ be the
$U_\mathcal{A}^{res}$-submodule of $V_q(n)$ generated by $v_0$, and define the
\emph{Weyl module}

\begin{equation}
    W_z^{res}(n) = V_\mathcal{A}^{res}(n) \otimes_\mathcal{A} \mathbb{C}
\end{equation}


More concretly, the Weyl module $W_z^{res}(n)$ has basis $\{v_0, v_1, \ldots, v_n\}$, where
\begin{align}
    X^+ v_i &= [n-i+1]_z v_{i-1} \\
    (X^+)^{(\ell)} v_i &= ( (n-i)_1) + 1) v_{i-\ell} \\
    (X^-)^{(\ell)} v_i &= ( i_1 + 1) v_{i+\ell} \\
    X^-v_i &= [i+1]_z v_{i+1} \\
    Kv_i &= z^{n-2i} v_i \\
\end{align}

where for any integer $r$, $r_0,r_1$ are defined such that $r = r_0 + \ell r_1$
and $0 \leq r_0 < \ell$. 

\begin{claim} $W_z^{res}(n)$ is irreducible if and only if $n < \ell$ or $\ell \mid (n+1)$
\end{claim}
\begin{proof}
See \cite{CP}
\end{proof}

In the case that $W_z^{res}$ is not irreducible, it is not difficult to prove the following proposition:

\begin{prop}
Let $V'$ be the subspace of $W_z^{res}(n)$ spanned by the $v_r$ such that $n_0 < r_0 < \ell$ and $r_1 < m_1$
\begin{enumerate}
    \renewcommand{\labelenumi}{\roman{enumi})}
    \item $V'$ is the only proper $U_z^{res}$-submodule of $W_z^{res}(n)$
    \item $V' \simeq V_z^{res}(m)$, where $m = \ell - 2 - n_0 + \ell(n_1 - 1)$
    \item $W_z^{res}(n) / V'$ is irreducible
\end{enumerate}
\end{prop}

It follows that each $W_z^{res}(n)$ has a unique irreducible quotient module $V_z^{res}(n) = W_z^{res}(n) / V'$. 

The proofs of the following two theorems about the structure of irreducible
representations of $U_z^{res}(\sll(2))$ can be found in \cite{CP}. We record
these theorems here to give some more intution about the structure of the
category of representations of $U_z^{res}(\sll(2))$, but they are not necessary
to construct a MTC.

\begin{theorem}
Every finite-dimensional irreducible $U_z^{res}$-module of type $\mathbf{1}$ is isomorphic to some $V_z^{res}(n)$
\end{theorem}

\begin{theorem}
$V_z^{res}(n) \simeq V_z^{res}(n_0) \otimes V_{z}^{res}(\ell n_1)$
\end{theorem}

\section{A braiding for $U_z^{res}(\sll(2))$}
\label{section:braiding}

To construct a MTC from the representations of $U_z^{res}(\sll(2))$, we need a
braiding on the category of representations as defined in
\ref{section:Braiding}.  In this section we will give a
braiding $\sigma_{V,W}: V \otimes W \to W \otimes V$ for representations $V,W$
of $U_z^{res}(\sll(2))$ which are the direct sums of their weight spaces.
Throughout $V,W$ will be assumed to be such modules. This braiding does not
come from a universal $R$-matrix for $U_z^{res}(\sll(2))$ as discussed in
\ref{subsection:R-matrix}, but is derived from a universal $R$-matrix for a
larger algebra. Details of this construction can be found in \cite{CP}.


Define 

\begin{equation}
\tilde{R}_q = \sum_{t=0}^\infty q^{\frac{1}{2} t(t+1)} \frac{(1-q^{-2})^t}{[t]_q^{!}} ((X^+)^t \otimes (X^-)^t)
\end{equation}

This will be our ``universal $R$-matrix''. Of course, this is not an element of
$U_z^{res}(\sll(2)) \otimes U_z^{res}(\sll(2))$ as the sum is infinite; it lies in a
completion of $U_z^{res}(\sll(2)) \otimes U_z^{res}(\sll(2))$. However, each of the terms
in the sum is an element of $U_z^{res}(\sll(2)) \otimes U_z^{res}(\sll(2))$, and for any
irreducible finite dimensional representations $\rho_V : U_z^{res}(\sll(2)) \to
\End(V)$, $\rho_W : U_z^{res}(\sll(2)) \to \End(W)$, all but finitely many terms in the
sum act as zero on $V \otimes W$. In other words, $\rho_V \otimes \rho_W
(\tilde{R}_q) \in \End(V \otimes W)$. This will allow us to construct a
braiding.


Define an invertible operator $E_{V,W}$ on $V \otimes W$ which acts as
$\lambda\overline{\mu}$ on the weight subspace $V^\lambda \otimes W^\mu$. As remarked
earlier, $\rho_V \otimes \rho_W (\tilde{R}_q) $ is a well-defined element of
$\End(V \otimes W)$. Let 

\begin{equation}
R_{V,W} = E_{V,W} \rho_V \otimes \rho_W (\tilde{R}_q)
\end{equation}

\begin{claim}
For any $x \in U_z^{res}(\sll(2))$, $V,W$ irreducible representations of $U_z^{res}(\sll(2))$
\begin{enumerate}
\item $R_{V,W} (\rho_V \otimes \rho_W) (\Delta(x)) R^{-1}_{V,W} = (\rho_{V} \otimes \rho_W)(\Delta^{op}(x))$
\item $R_{V,W}$  satisfies the Yang-Baxter equation: 

\begin{equation}
R_{12} R_{13} R_{23} = R_{23} R_{13} R_{12}
\end{equation}
\end{enumerate}
\end{claim}
\begin{proof}
    See \cite{CP}.
\end{proof}

We have therefore that 

\begin{theorem}
\label{theorem:Braiding}
The category of finite dimensional $U_z^{res}(\sll(2))$-modules of type
$\mathbf{1}$ which are the direct sums of their weight spaces is a braided
monoidal category, with braiding isomorphisms given by 

\begin{equation}
\sigma_{V,W}(v \otimes w) = R_{V,W}(w \otimes v)
\end{equation}
\end{theorem}


\section{The general case: $U_q(\frakg)$}
\label{section:U_q(g)}


In general we can define a quantum group $U_q(\frakg)$ for any semisimple Lie
algebra $\frakg$. The representations of the algebra $U_z^{res}(\frakg)$
where $z$ is a root of unity give us a modular tensor category in the same way
as those of $U_z^{res}(\sll(2))$. This results in a rich array of examples of
modular tensor categories. $U_q(\frakg)$ is a deformation of the
universal enveloping algebra $U(\frakg)$ in the same way as $U_q(\sll(2))$ is a
deformation of $U(\sll(2))$; its representations are indexed by highest weights
$\lambda$ in the same way as the representations of $\frakg$. 

In this section we will give a definition of the quantum group $U_q(\frakg)$ in
general, and touch briefly on its representation theory. The construction of
the algebra $U_z^{res}(\frakg)$ where $z$ is a complex number and the braiding
on the category of representations of $U_z^{res}(\frakg)$ can be done almost
exactly as with $\sll(2)$. Chari and Pressley \cite{CP} discuss this general
case extensively. 

The definition of $U_q(\frakg)$ is modelled on the definition of $U_q(\sll(2))$
and Serre's presentation of a semisimple Lie algebra $\frakg$, which we give
briefly here. More about this can be found in Humphreys~\cite{Humphreys1973}.

Let $\frakg$ be an arbitrary semisimple algebra. Suppose $\Pi$ is
a basis of the root system, and denote the entries of the Cartan matrix
by 

    \begin{equation}
        a_{\alpha\beta} = 2(\alpha, \beta) / (\alpha, \alpha).
    \end{equation}

    Then the Lie algebra has a presentation with $3|\Pi|$ generators
    $x_\alpha, y_\alpha, h_\alpha$, $\alpha \in \Pi$ and the relations 

    \begin{align}
        \left[ h_\alpha, h_\beta \right] &= 0       &   \left[ x_\alpha, y_\beta \right] &= \delta_{\alpha\beta} h_\alpha \\
        \left[ h_\alpha, x_\beta \right] &= a_{\alpha\beta} x_\beta  &   \left[ h_\alpha, y_\beta \right] &= -a_{\alpha\beta} y_\beta \\
    \end{align}

    Let $d_\alpha = \frac{(\alpha,\alpha)}{2} \in \left\{ 1,2,3 \right\}$, and set 

    \begin{equation}
        q_\alpha = q^{d_\alpha}
    \end{equation}

    and for $n \in \mathbb{Z}$, define

    \begin{equation}
        [n]_\alpha = [n]_{q_\alpha} = \frac{q^{nd_\alpha} - q^{-nd_\alpha}}{q^{d_\alpha} - q^{-d_\alpha}}
    \end{equation}

    $[n]_\alpha^!$ and $\dbinom{n}{k}_\alpha$ are defined similarly: 
    
    \begin{equation}
        [n]_\alpha^! = [n]_{q_{\alpha}}^!
    \end{equation}
    \begin{equation}
        \dbinom{n}{k}_\alpha = \dbinom{n}{k}_{q_\alpha}
    \end{equation}

    \begin{defn}
    \label{UqgDef}
        The quantum group $U_q(\frakg)$ has generators
        $ X^+ _\alpha, X^-_\alpha, K_\alpha, K_\alpha^{-1}$ for each $\alpha \in
        \Pi$, and relations

        \begin{align}
            K_\alpha K_\alpha^{-1} =\ &1  = K_\alpha^{-1}K_\alpha \\
            K_\alpha K_\beta &= K_\beta K_\alpha \\
            K_\alpha  X^+ _\beta K_\alpha^{-1} &= q^{(\alpha, \beta)}  X^+ _\beta \\
            K_\alpha X^-_\beta K_\alpha^{-1} &= q^{-(\alpha, \beta)}  X^+ _\beta \\
            [ X^+ _\alpha, X^-_\beta] &= \delta_{\alpha\beta} \frac{K_\alpha - K_\alpha^{-1}}{ q_\alpha - q_\alpha^{-1}} 
        \end{align}

        for all $\alpha, \beta \in \Pi$, and for $a \neq \beta$

        \begin{align}
            \sum_{s=0}^{1-a_{\alpha\beta}} (-1)^s \dbinom{1-a_{\alpha\beta}}{s}_\alpha  (X^+)_\alpha^{1-a_{\alpha\beta} - s}  X^+_\beta  (X^+)_\alpha^s  &= 0 \\
            \sum_{s=0}^{1-a_{\alpha\beta}} (-1)^s \dbinom{1-a_{\alpha\beta}}{s}_\alpha (X^-)_\alpha^{1-a_{\alpha\beta} - s} (X^-)_\beta (X^-)_\alpha^s  &= 0 
        \end{align}
    \end{defn}

We can define a Hopf algebra structure $(\Delta, \varepsilon, S)$ such
that for all $\alpha \in \Pi$,

    \begin{align}
        \Delta( X^+ _\alpha) &=  X^+ _\alpha \otimes 1 + K_\alpha \otimes  X^+ _\alpha      & \varepsilon( X^+ _\alpha) &= 0  & S( X^+ _\alpha) &= -K_\alpha^{-1}  X^+ _\alpha \\
        \Delta(X^-_\alpha) &= X^-_\alpha \otimes K_\alpha^{-1} + 1 \otimes X^-_\alpha & \varepsilon(X^-_\alpha) &= 0  & S(X^-_\alpha) &= -X^-_\alpha K_\alpha \\
        \Delta(K_\alpha) &= K_\alpha \otimes K_\alpha                           & \varepsilon(K_\alpha) &= 1  & S(K_\alpha) &= K_\alpha^{-1}
    \end{align}


We can set $q$ to be a root of unity in essentially exactly the same way as
with $U_q(\sll(2))$: let $\mathcal{A} = \mathbb{Z}[q,q^{-1}]$ and define an 
$\mathcal{A}$-subalgebra $U_\mathcal{A}^{res}(\sll(2))$ to be the
$\mathcal{A}$-subalgebra of $U_q(\frakg)$ generated by $ X_\alpha^+$
,$X_\alpha^-$,$K_\alpha$,$K_\alpha^{-1}$, $(X_\alpha^+) ^{(n)}$,
$(X_\alpha^-)^{(n)}$, where


\begin{align}
     (X_\alpha^+)^{(n)} = \frac{ (X_\alpha^+)^n}{[n]^!_{q}} \\
    (X_\alpha^-)^{(n)} = \frac{(X_\alpha^-)^n}{[n]^!_{q}} 
\end{align}


We will now briefly discuss the representation theory of $U_q(\frakg)$ and $U_z^{res}(\frakg)$.

The irreducible representations $V_q(\lambda)$ of type $\mathbf{1}$ of
$U_q(\frakg)$ are indexed by highest weights $\lambda$ and every representation
is generated by a highest weight vector. The simultaneous eigenvectors of the
$K_\alpha$ form a basis for the representation. We can use these
representations to define representations of $U_z^{res}(\frakg)$ as follows:

Let $V_q(\lambda)$ be the irreducible $U_q(\frakg)$-module with highest weight
$\lambda$. Let $V_\mathcal{A}^{res}(\lambda)$ be the
$U_\mathcal{A}^{res}(g)$-submodule of $V_q(\lambda)$ generated by the highest
weight vector $v_\lambda$, and define the \emph{Weyl module}

\begin{equation}
    W_z^{res}(\lambda) = V_\mathcal{A}^{res}(\lambda) \otimes_\mathcal{A} \mathbb{C}
\end{equation}

via the homomorphism $A \to \mathbb{C}$, $q \mapsto z$.

As in the case of $\sll(2)$, we will be most interested in the irreducible Weyl
modules. 

\begin{claim}
    $W_z^{res}(\lambda)$ is irreducible if either 
    \begin{enumerate}
        \item $(\lambda + \rho, \hat{\alpha}) < \ell$ for all positive
            roots $\alpha$
        \item $\lambda = (\ell - 1) \rho + \ell \mu$ for some positive root $\mu$
    \end{enumerate}
\end{claim}
\begin{proof}
See Chari \& Pressley~\cite{CP}
\end{proof}

The objects in the MTC constructured from the category of representations of
$U_z^{res}(\frakg)$ are the irreducible Weyl modules $W_z^{res}(\lambda)$ such
that $(\lambda + \rho, \hat{\alpha}) < \ell$ for all positive roots $\alpha$.
