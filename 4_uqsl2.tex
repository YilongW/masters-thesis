\section{The quantized universal enveloping algebra $U_q(\sll(2))$}

$U_q(\sll(2))$ is a deformation of the universal enveloping algebra
$U(\sll(2))$ defined in \ref{UnivEnvAlg}, which in some sense approaches
$U(\sll(2))$ as $q \to 1$. In this section we will define $U_q(\sll(2))$ and
make explicit its relationship to $U(\sll(2))$.

% Kassel, p. 121
%% Jantzen, p. 9
Let $q$ be a complex number such that $q \neq 0$, $q^2 \neq 1$.  Then
$U_q(\sll(2))$ is the associative algebra with generators $E,F,K, K^{-1}$ and
relations 

\begin{align}
    KK^{-1} &= 1 = K^{-1}K \\
    KEK^{-1} &= q^2 E \\
    KFK^{-1} &= q^{-2} F \\
    [E,F] &= \frac{K - K^{-1}}{q - q^{-1}}
\end{align}


%% this presentation taken verbatim from Kassel, p. 125

This is not defined for $q = 1$. We can use an alternate presentation which is
defined for $q = 1$. $U_q(\sll(2))$ is isomorphic to the algebra
$U'_q(\sll(2))$ with generators $E,F,K,K^{-1},L$ and relations:

\begin{align}
    KK^{-1} &= 1 &  K^{-1}K  =1 \\
    KEK^{-1} &= q^2 E & KFK^{-1} &= q^{-2} F \\
    [E,F] &= L  & (q - q^{-1})L &= K-K^{-1} \\
    [L,E] &= q(EK + K^{-1}E) & [L,F] = -q^{-1}(FK + K^{-1}F)
\end{align}

For $q=1$, we can define a homomorphism $\varphi: U'_1(\sll(2)) \to
U(\sll(2))$ by sending $E$ to $X$, $F$ to $Y$, $K$ to 1, and $L$ to $H$. 
\begin{lemma}
This homomorphism is surjective and has kernel generated by $(K-1)$, so
$U'_1(\sll(2)) \stackrel{\sim}{=} U(\sll(2))$
\end{lemma}

%todo: should I explain why the representation theory looks this way? 
% from Kassel, p. 123
There is a unique algebra automorphism of $U_q(\sll(2))$ such that 

\begin{equation}
    \omega(E) = F, \omega(F) = E, \omega(K) = K^{-1}
\end{equation}

This is called the \emph{Cartan automorphism}. We will use this later when
constructing representations of $U_q(\sll(2))$.

Before proceeding, we will note some relations between the generators
$E,F,K,K^{-1}$. The following can be checked fairly easily by induction:

\begin{align}
    \label{EFrelations}
    EF^s = F^sE + [s]_q F^{s-1} \frac{Kq^{1-s} - K^{-1} q^{-(1-s)}}{q-q^{-1}} \\
    FE^s = E^sF - [s]_q E^{s-1} \frac{Kq^{s-1} - K^{-1} q^{-(s-1)}}{q-q^{-1}}
\end{align}

\section{A Hopf Algebra Structure on $U_q(\sll(2))$}
%todo: define such a structure on \mathfrak{g}, really
We can define a Hopf algebra structure on $U_q(\sll(2))$ as follows:

% from Kassel, p. 140
\begin{align}
    \Delta(E) &= 1 \otimes E + E \otimes K &  \Delta(F) &= K^{-1} \otimes F + F \otimes 1 \\
    \Delta(K) &= K \otimes K &  \Delta(K^{-1}) &= K^{-1} \otimes K^{-1}\\ 
    \varepsilon(E) &= \varepsilon(F) = 0 &  \varepsilon(K) &= \varepsilon(K^{-1}) = 1
\end{align}

and antipode defined by 

\begin{align}
    S(E) &= -EK^{-1} & S(F)      &= -KF \\
    S(K) &= K^{-1}   & S(K^{-1}) &= K 
\end{align}
It is straightforward to check that this defines a Hopf algebra structure. 

\section{Representations of $U_q(\sll(2))$, $q$ not a root of unity}

When $q$ is not a root of unity, the representation theory of $U_q(\sll(2))$
bears a striking resemblance to that of $\sll(2)$. We will sketch the
classification of the representations of $\sll(2)$ here for $q$ not a root of
unity.

\begin{claim}
    Let $M$ be a finite dimensional $U_q(\sll(2))$-module. Then there is a
    vector $v \in M$ such that $Ev = 0$. 
\end{claim}
%todo: do I need to prove that $K$ has eigenvectors?

\begin{proof}
    Let $v_0$ be an eigenvector of $K$ with eigenvalue $\lambda$. $KEK^{-1} =
    q^2 E$, so it follows that $E^i v_0$ is an eigenvector of $K$ with
    eigenvalue $q^{2i}\lambda$. Each value $q^{2i}\lambda$ is different as $q$
    is not a root of unity, therefore $K$ has infinitely many eigenvectors. 

    This is impossible, so some $E^i v_0$ must be zero.
\end{proof}

For any $\lambda \in k, \lambda \neq 0$ there is an infinite dimensional
$U_q(\sll(2))$-module $M(\lambda)$ with basis $v_0, v_1, v_2, \ldots$ such that
for any $i$,
\begin{align}
    Kv_i &= \lambda q^{-2i}v_i \\
    Fv_i &= v_{i+1} \\
    Ev_i &= \begin{cases} 0 & \text{if $i = 0$} \\
                          [i]_q \frac{\lambda q^{1-i} - \lambda^{-1}q^{i-1}}{q - q^{-1}}v_{i-1} & \text{else}
            \end{cases}
\end{align}

To see that this a $U_q(\sll(2))$-module, we can note that 

%todo: set $U = U_q(\sll(2))$ here
\begin{equation}
    M(\lambda) = U/(UE + U(K-\lambda))
\end{equation}

with $v_i$ being the coset of $F^i$. It follows from this that if $M$ is a
$U_q(\sll(2))$-module and $v \in M$ such that $Ev = 0$, $Kv = \lambda v$, there
is a unique homomorphism $\varphi: M(\lambda) \to M$ where $\varphi(v_0) = v$.
If $M$ is simple, then this homomorphism is surjective and it follows that $M$
is a quotient of $M(\lambda)$. 

We now have that 
\begin{claim}
    Any finite dimensional simple $U_q(\sll(2))$-module is a quotient of $M(\lambda)$.
\end{claim}

\begin{claim}
    $M(\lambda)$ is simple for $\lambda \neq \pm q^n$
\end{claim}

\begin{proof}
    See \cite{Jantzen1996}.
\end{proof}

For $\lambda = \pm q^{n}$, $M(\lambda)$ has one nontrivial submodule spanned by the $v_i$ with $i \geq n+1$.
This leads us to the following characterization (details can be found in \cite{Jantzen1996}):


For each integer $n \geq 0$, there are two unique (up to isomorphism)
representations of $U_q(\sll(2))$ of dimension $n+1$. 
The representations are $V(n,+)$ with basis $\left\{ v^+_0, \ldots, v^+_n
\right\}$ , and $V(n,-)$ with basis $\left\{ v^-_0, \ldots, v^-_n \right\}$
with action defined by 

% from Kassel, p. 101
\begin{align*}
    &K v^+_i = q^{n-2i} v^+_i  &
    &K v^-_i = -q^{n-2i} v^-_i \\
    &F v^+_i = \begin{cases} [i+1]_qv_{i+1}& \text{ if $i < n$} \\ 0& \text{ if $i = n$} \end{cases} &
    &F v^-_i = \begin{cases} [i+1]_qv'_{i+1}& \text{ if $i < n$} \\ 0& \text{ if $i = n$} \end{cases} \\
    &E v^+_i = \begin{cases} 
                     [n-i+1]_qv_{i-1}& \text{ if $i > 0$} \\ 
                    0& \text{ if $i = 0$} 
             \end{cases} &
    &E v^-_i = \begin{cases} 
                    -[n-i+1]_q v'_{i-1}& \text{ if $i > 0$} \\ 
                    0& \text{ if $i = 0$} 
             \end{cases}
\end{align*}

By the proof sketched above, these are the only representations of
$U_q(\sll(2))$. It is also possible to show that every finite dimensional
representation of $U_q(\sll(2))$ is semisimple. In other words, the category of
representations is semisimple.

\section{Representations of $U_q(\sll(2))$, $q$ a root of unity}
% Jantzen, 2.11 - 2.13. June 2010 in Clairefontaine book

%% Explain this explicitly
%The representation theory of the associative algebra $U(\sll(2))$ can
%be described in exactly the same way as that of the Lie algebra
%$\sll(2)$. 

Suppose now that $q$ is a root of unity. Here the situation is considerably
different. The category of finite dimensional representations is no longer
semisimple. It is still however possible to classify the irreducible representations,
and we will do so. 

First, we will show the following result:

\begin{theorem}
If $q$ is a root of unity, then there are no simple $U_q(\sll(2))$-modules
with dimension $\geq \ell+1$. 
\end{theorem}

First it will be helpful to prove a lemma about the center of $U_q(\sll(2))$.

\begin{lemma}
    If $q$ is a $\ell^\text{th}$ root of unity, then $E^\ell, F^\ell, K^\ell,
    K^{-\ell}$ are all central in $U_q(\sll(2))$.
\end{lemma}
\begin{proof}
    It is straightforward to check from the defining relations that 

    \begin{align}
        K^\ell E K^{-\ell} &= q^{2\ell}E = E  & K^{\ell}FK^{-\ell} &= q^{-2\ell}F = F\\
        K E^\ell K^{-1} &= q^{2\ell}E^\ell = E^\ell  & KF^\ell K^{-1} &= q^{-2\ell}F^\ell = F^\ell
    \end{align}
    
    and from \ref{EFrelations}, since $[\ell]_q = 0$, 
    \begin{equation}
        EF^\ell = F^\ell E \text{ and } E^\ell F = FE^\ell
    \end{equation}

\end{proof}
\begin{proof}
    %todo
\end{proof}

The irreducible representations of
$U_q(\sll(2))$ can be classified as follows: 

For each $n\geq 0$, there are 2 simple $U_q(\sll(2))$-modules of dimension $n+1$: 
% todo: fix formatting here
$V_+(n)$ with basis $v^+_0, \ldots, v^+_n$ \\
$V_-(n)$ with basis $v^-_0, \ldots, v^-_n$

such that: 
% todo: find a better way to motivate q-notation. (maybe that paper that
% prakash showed you?)
\begin{align*}
    &K v^+_i = q^{n-2i} v^+_i  &
    &K v^-_i = -q^{n-2i} v^-_i \\
    &F v^+_i = \begin{cases} [i+1]_qv_{i+1}& \text{ if $i < n$} \\ 0& \text{ if $i = n$} \end{cases} &
    &F v^-_i = \begin{cases} [i+1]_qv'_{i+1}& \text{ if $i < n$} \\ 0& \text{ if $i = n$} \end{cases} \\
    &E v^+_i = \begin{cases} 
                     [n-i+1]_qv_{i-1}& \text{ if $i > 0$} \\ 
                    0& \text{ if $i = 0$} 
             \end{cases} &
    &E v^-_i = \begin{cases} 
                    -[n-i+1]_q v'_{i-1}& \text{ if $i > 0$} \\ 
                    0& \text{ if $i = 0$} 
             \end{cases}
\end{align*}

% todo: why are there 2 representations here instead of 1? what extra freedom
% allows that to happen?

Suppose $\ell$ is odd, and $q$ is an $\ell^{\text{th}}$ root of unity. 

If $0 \leq n \leq \ell - 1$, the $(n+1)$-dimensional
$U_q(\sll(2))$-modules are the $V_{\pm}(n)$ described above. 

It remains to classify the $\ell$-dimensional $U_q(\sll(2))$-modules.
There are $3$ infinite classes of simple $\ell$-dimensional representations of  $U_q(\sll(2))$:
\begin{enumerate}
        % todo: say somewhere that the results and notation are taken from
        % Jantzen
        \item For any $b,\lambda \in \mathbf{C}, \lambda \neq 0$, define
            $Z_b(\lambda)$ to be the module with basis $v_0, \ldots, v_{\ell -
            1}$ with action of $U$ given by: 
\begin{align*}
    &K v_i = q^{-2i} \lambda v_i \\
    &F v_i = \begin{cases} v_{i+1}& \text{ if $i < n$} \\  b v_0& \text{ if $i = n$} \end{cases}  \\
    &E v_i = \begin{cases} 
        [i]_q \frac{\lambda q^{1-i} - \lambda^{-1} q^{i-1}}{q - q^{-1}} v_{i-1}& \text{ if $i > 0$} \\ 
                    0& \text{ if $i = 0$} 
             \end{cases} 
\end{align*}

If $b \neq 0$ or $\lambda^{2\ell}\neq 1$ then $Z_b(\lambda)$ is simple.
$Z_0(\pm q^k)$ is simple if and only if $k = \ell - 1$. 

\begin{remark}
    $U_q(\sll(2))$ is not semisimple: the modules $Z_0(\pm q_k)$ are not
    semisimple for $0 \leq k < \ell - 1$.
\end{remark}
\item For any $U_q(\sll(2))$-module $N$, define $^\omega N$ to be
    equal to $N$ as a vector space, and where $u$ acts on $^\omega N$ as
    $\omega(u)$ acts on $N$.

    The $^\omega Z_b(\lambda)$ are another class of modules.
\item Let $a,b,\lambda \in \mathbf{C}$, $a,b\neq 0$
    % todo: say something about them not all being the same.
\begin{align*}
    &K v_i = q^{-2i} \lambda v_i \\
    &F v_i = \begin{cases} 
                v_{i+1}& \text{ if $i < n$} \\  
                b v_0& \text{ if $i = n$} 
             \end{cases}  \\
    &E v_i = \begin{cases} 
                \left(ab + [i]_q \frac{(\lambda q^{1-i} - \lambda^{-1} q^{i-1})}{q - q^{-1}}\right) v_{i-1}& \text{ if $i > 0$} \\ 
                a v_{\ell - 1}& \text{ if $i = 0$} 
             \end{cases} 
\end{align*}
\end{enumerate}


\section{Representations of $U_q(\mathfrak{g})$}
    \subsection{$U_q(\mathfrak{g})$}

        % taken from Jantzen

        Let $\mathfrak{g}$ be an arbitrary semisimple algebra. Suppose $\Pi$ is
        a basis of the root system, and denote the entries of the Cartan matrix
        by 

        \begin{equation}
            a_{\alpha\beta} = 2(\alpha, \beta) / (\alpha, \alpha).
        \end{equation}

        Then the Lie algebra has a presentation with $3|\Pi|$ generators
        $x_\alpha, y_\alpha, h_\alpha$, $\alpha \in \Pi$ and the relations 

        \begin{align}
            \left[ h_\alpha, h_\beta \right] &= 0       &   \left[ x_\alpha, y_\beta \right] &= \delta_{\alpha\beta} h_\alpha \\
            \left[ h_\alpha, x_\beta \right] &= a_{\alpha\beta} x_\beta  &   \left[ h_\alpha, y_\beta \right] &= -a_{\alpha\beta} y_\beta \\
        \end{align}

        When constructing the algebra $U_q(\sll(2))$, we required that $q^2
        \neq 1$. We need a slightly stronger version of this in general: set

        \begin{equation}
            d_\alpha = \frac{(\alpha, \alpha)}{2}
        \end{equation}

        We require that $q^{2d_{\alpha}} \neq 0$ for all $\alpha$. This means
        that $q^2 \neq 0$, may also require that $q^4, q^6 \neq 0$, depending
        on the length of the longest root in $\mathfrak{g}$.

        Now set 

        \begin{equation}
            q_\alpha = q^{d_\alpha}
        \end{equation}

        and for $n \in \mathbf{Z}$,

        \begin{equation}
            [n]_\alpha = [n]_{q_\alpha} = \frac{q^{nd_\alpha} - q^{-nd_\alpha}}{q^{d_\alpha} - q^{-d_\alpha}}
        \end{equation}

        $[n]_\alpha^!$ and $\dbinom{n}{k}_\alpha$ are defined similarly.

        \begin{defn}
            The quantized enveloping algebra $U_q(\mathfrak{g})$ has generators
            $E_\alpha, F_\alpha, K_\alpha, K_\alpha^{-1}$ for each $\alpha \in
            \Pi$, and relations

            \begin{align}
                K_\alpha K_\alpha^{-1} =\ &1  = K_\alpha^{-1}K_\alpha \\
                K_\alpha K_\beta &= K_\beta K_\alpha \\
                K_\alpha E_\beta K_\alpha^{-1} &= q^{(\alpha, \beta)} E_\beta \\
                K_\alpha F_\beta K_\alpha^{-1} &= q^{-(\alpha, \beta)} E_\beta \\
                [E_\alpha, F_\beta] &= \delta_{\alpha\beta} \frac{K_\alpha - K_\alpha^{-1}}{ q_\alpha - q_\alpha^{-1}} 
            \end{align}

            for all $\alpha, \beta \in \Pi$, and for $a \neq \beta$

            \begin{align}
                \sum_{s=0}^{1-a_{\alpha\beta}} (-1)^s \dbinom{1-a_{\alpha\beta}}{s}_\alpha E_\alpha^{1-a_{\alpha\beta} - s} E_\beta E_\alpha^s  &= 0 \\
                \sum_{s=0}^{1-a_{\alpha\beta}} (-1)^s \dbinom{1-a_{\alpha\beta}}{s}_\alpha F_\alpha^{1-a_{\alpha\beta} - s} F_\beta F_\alpha^s  &= 0 
            \end{align}
        \end{defn}
        \subsection{A Hopf algebra structure on $U_q(\mathfrak{g})$}

        We can define a Hopf algebra structure $(\Delta, \varepsilon, S)$ such
        that for all $\alpha \in \Pi$,

        \begin{align}
            \Delta(E_\alpha) &= E_\alpha \otimes 1 + K_\alpha \otimes E_\alpha      & \varepsilon(E_\alpha) &= 0  & S(E_\alpha) &= -K_\alpha^{-1} E_\alpha \\
            \Delta(F_\alpha) &= F_\alpha \otimes K_\alpha^{-1} + 1 \otimes F_\alpha & \varepsilon(F_\alpha) &= 0  & S(F_\alpha) &= -F_\alpha K_\alpha \\
            \Delta(K_\alpha) &= K_\alpha \otimes K_\alpha                           & \varepsilon(K_\alpha) &= 1  & S(K_\alpha) &= K_\alpha^{-1}
        \end{align}
    \subsection{Representation theory of $U_q(\mathfrak{g})$}


