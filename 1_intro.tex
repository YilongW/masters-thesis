One of the major challenges in implementations of quantum computing is that of
implementing quantum gates accurately. Topological computing, first proposed by
Kitaev in 1997 \cite{Kitaev2003}, exploits topological properties of certain
quasiparticles called \emph{anyons} to give an implementation of quantum
comping which is inherently fault-tolerant.

First we give an introduction to anyons. the spin statistics theorem in 3
dimensions says that when 2 identical particles are inerchanged, there are
exactly 2 possibilities: either the wave is symmetric (in the case of bosons)
or it is antisymmetric (in the case of fermions).  This is because in 3
dimensions, 2 identical particles being interchanged twice is topologically
equivalent to one particle being transported around the other, which is in turn
topologically equivalent to neither particle moving at all.  In 2 dimensions
the situations is completely different: the process of moving one particle
around another is not topologically equivalent to neither particle moving
around another, and the system will not necessarily end up in the same state.
We refer to particles in 2 dimensions which are neither bosons nor fermions as
\emph{anyons}. 

Suppose we have a system of $n$ anyons, and rearrange them in some way. Each
topological class of such trajectories in 2+1 dimensions corresponds to an
element of the braid group $B_n$:

\begin{center}
\scalebox{1} % Change this value to rescale the drawing.
% picture of braiding worldlines
{
\begin{pspicture}(0,-1.4947689)(5.20875,1.5147688)
\psbezier[linewidth=0.02](4.04,0.7813313)(4.04,0.6240438)(4.22,0.23082528)(4.64,0.15218155)(5.06,0.073537834)(5.04,-0.10930765)(5.04,-0.30985886)
\psbezier[linewidth=0.02](5.04,0.7813313)(5.04,0.46675643)(4.96,0.27014711)(4.6,0.21509652)
\psbezier[linewidth=0.02](4.42,0.16004592)(4.16,0.104995325)(4.04,-0.16239332)(4.04,-0.30985886)
\psdots[dotsize=0.1](4.08,1.0213313)
\psdots[dotsize=0.1](5.08,1.0213313)
\rput(4.043125,1.31){$a$}
\rput(5.047969,1.35){$b$}
\psdots[dotsize=0.1](0.1,-0.018668715)
\psdots[dotsize=0.1](1.1,-0.018668715)
\rput(0.063125,0.27){$a$}
\rput(1.0679687,0.31){$b$}
\psarc[linewidth=0.02]{<-}(1.0,0.061331283){0.78}{199.98311}{174.28941}
\psbezier[linewidth=0.02](4.04,-0.30985886)(4.0,-0.6047751)(4.22,-0.845619)(4.64,-0.92426276)(5.06,-1.0029064)(5.04,-1.3332101)(5.06,-1.4747688)
\psbezier[linewidth=0.02](5.04,-0.30985886)(5.04,-0.6096879)(4.96,-0.8062972)(4.6,-0.8613478)
\psbezier[linewidth=0.02](4.42,-0.9163984)(4.16,-0.971449)(4.04,-1.2388376)(4.04,-1.4747688)
\rput(2.7798438,-0.028668715){$\iff$}
\end{pspicture} 
}
\end{center}

The state of the anyons therefore transforms via representations of the braid
group $B_n$. In the simplest case, the state transforms via a one-dimensional
representation of $B_n$:

\begin{equation}
\Psi \mapsto e^{i\theta}\Psi
\end{equation}

Anyons which transform in this way are called \emph{abelian anyons}. The other
possibility is that the state transforms according to some higher-dimensional
representation of $B_n$. In this case nontrivial operations can be performed
simply by rearranging the anyons in some suitable way. Anyons which transform
according to higher-dimensional representations are called \emph{nonabelian
anyons}, and it is possible to do quantum computation with certain species
of nonabelian anyons.

The key point here is that in this model the change in state of the anyons
depends \emph{only} on the topological class of the braid, so minor
perturbations of a particle's path won't affect the result of the computation
at all. The process is thus inherently resistant to errors. 

Do anyons actually exist? As we said earlier, all particle types are bosons or
fermions. However, \expand{Something about anyons arising as emergent
properties of some kinds of system\ldots fractional quantum Hall systems here}.

In general a system with anyons will have several different types of anyons.
Any two anyons can be brought together to form a new anyon. This process is
called \emph{fusion}. In general if two anyons fuse, then there might be
several possible results. 

The mathematical structure that describes anyons is that of modular tensor
categories. The goal in this thesis is to give an explanation of modular tensor
categories as they relate to topological quantum computing, only requiring some
basic knowledge of algebra and category theory. The modular tensor categories
that describe anyons' behavior are constructed from the representations of
certain algebraic objects called \emph{quantum groups}.

\todo{Prakash: Rowell says `only anyonic properties of bosonic systems can be
described fully by MTCs'. Can you make some sense of this?}


In Chapter 2 we give the algebraic background of braid groups, Lie algebras,
and Hopf algebras necessary to construct quantum groups. In Chapter 3 we describe the
correspondence between the features of anyonic systems and those of modular
tensor categories and define modular tensor categories.  In
Chapter 4 we introduce the quantum group $U_q(\sll(2))$ and discuss its
representation theory. In
Chapter 5 we explain how to construct a modular tensor category from the
quantum group $U_q(\sll(2))$. In Chapter 6 we finally get to the business of
discussing topological quantum computing. We review the work that has been done
on the subject, in particular that on the computational business of
constructing braids that correspond to particular unitaries. 

The hope is to give a concrete account of the theory of modular tensor
categories obtained from quantum groups at roots of unity for computer
scientists, from a category theoretic and algebraic point of view rather than a
physical point of view. 
