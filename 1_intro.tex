\section{Overview of TQC}

First we'll discuss the basic idea behind TQC.

One of the major problems with quantum computing implementations is that it is
very hard to implement quantum gates accurately. Even with the advent of
quantum error correction, a certain minimum level of accuracy is required.
Kitaev in 1997 proposed an implementation of QC which exploits topological
properties to make this easier.

The spin statistics theorem says that when 2 identical particles are
interchanged, the result on the system is either $+1$ or $-1$. Particles for
which $+1$ are called ``bosons'' and those for which $-1$ are called
``fermions''. Essentially this is because the \expand{which?} group of the
\expand{what?} is $\left\{ -1, 1 \right\} \simeq \mathbb{Z}_2$. 


In \expand{year}, \expand{who?} found that certain quasiparticles in fractional
quantum Hall states had statistics which were neither fermionic nor bosonic. 
As the \expand{what?} group of the circle is \expand{what?}, \emph{any} phase
can occur when exchanging two identical such quasiparticles. \expand{Who? (Wilzcek?)} 
termed these quasiparticles ``anyons''. 

These anyons live on a 2D surface, so their worldlines in 2+1 dimensions can be modelled as braids:

\scalebox{1} % Change this value to rescale the drawing.
% picture of braiding worldlines
{
\begin{pspicture}(0,-1.4947689)(5.20875,1.5147688)
\psbezier[linewidth=0.02](4.04,0.7813313)(4.04,0.6240438)(4.22,0.23082528)(4.64,0.15218155)(5.06,0.073537834)(5.04,-0.10930765)(5.04,-0.30985886)
\psbezier[linewidth=0.02](5.04,0.7813313)(5.04,0.46675643)(4.96,0.27014711)(4.6,0.21509652)
\psbezier[linewidth=0.02](4.42,0.16004592)(4.16,0.104995325)(4.04,-0.16239332)(4.04,-0.30985886)
\psdots[dotsize=0.1](4.08,1.0213313)
\psdots[dotsize=0.1](5.08,1.0213313)
\rput(4.043125,1.31){$a$}
\rput(5.047969,1.35){$b$}
\psdots[dotsize=0.1](0.1,-0.018668715)
\psdots[dotsize=0.1](1.1,-0.018668715)
\rput(0.063125,0.27){$a$}
\rput(1.0679687,0.31){$b$}
\psarc[linewidth=0.02]{<-}(1.0,0.061331283){0.78}{199.98311}{174.28941}
\psbezier[linewidth=0.02](4.04,-0.30985886)(4.0,-0.6047751)(4.22,-0.845619)(4.64,-0.92426276)(5.06,-1.0029064)(5.04,-1.3332101)(5.06,-1.4747688)
\psbezier[linewidth=0.02](5.04,-0.30985886)(5.04,-0.6096879)(4.96,-0.8062972)(4.6,-0.8613478)
\psbezier[linewidth=0.02](4.42,-0.9163984)(4.16,-0.971449)(4.04,-1.2388376)(4.04,-1.4747688)
\rput(2.7798438,-0.028668715){$\iff$}
\end{pspicture} 
}
\todo{want a different picture here}

It turns out that if $n$ anyons are rearranged in some way, then the change of
state of their system depends only on the element of the braid group $B_n$ that
the braid corresponds to! This gives an inherent resistance to errors: small
perturbations have absolutely no effect on the transformation performed by
rearranging the anyons.

\expand{something about abelian and nonabelian anyons: what's the difference?}

In particular two anyon models which are nonabelian have been proposed: the
Ising and Fibonacci anyons. There are modelled by the MTCs obtained from the
quantum groups $U_q(\sll(2))$ at a fourth and fifth root of unity respectively. 


\expand{Give some history of the subject. Some of the results that TQC has given us (Jones polynomial universal for QC)}

\expand{a couple of words about current experimental status}

\section{Anyons}
In our basic setup, we have a set of identical anyons which have some joint ``state''.

\expand{ Setup (anyons, exchanging, fusing)}
\expand{worldlines as braids, dependence only on topological class of braid}

\begin{center}
    (picture here?)
%    $\circ \hspace{1in} \circ\hspace{1in} \circ\hspace{1in} \circ$
\end{center}

A computation consists of exchanging the anyons in some way. If we draw such an
exchange in 2+1 dimensions, we obtain a braid. 

\begin{center}
    (picture here?)
\end{center}

The key point here is that in this model the change in state of the anyons
depends \emph{only} on the topological class of the braid, so minor
erturbations of a particle's path won't affect the result of the computation at
all. The process is thus inherently resistant to errors. 

\section{Connection to MTCs}

Anyons are essentially described by MTCs. 

\expand{Anyons are modelled by MTCs, and these MTCs are obtained from quantum
groups at roots of unity}

\todo{Prakash: Rowell says `only anyonic properties of bosonic systems can be
described fully by MTCs'. Can you make some sense of this?}

Modular tensor categories provide a framework for this braiding. In particular,
the MTCs that describe TQC are constructed from the representations of quantum groups. 

In Chapters 2 and 3 we go over the basic categorical and algebraic background
required to describe quantum groups and modular tensor categories. In Chapter 4
we introduce the quantum group $U_q(\sll(2))$ and its representation theory. In
Chapter 5 we explain how to construct a modular tensor category from the
quantum group $U_q(\sll(2))$. In Chapter 6 we finally get to the business of
discussing topological quantum computing. We review the work that has been done
on the subject, in particular that on the computational business of
constructing braids that correspond to particular unitaries. 

The hope is to give a concrete account of the theory of modular tensor
categories obtained from quantum groups at roots of unity for computer
scientists, from a category theoretic and algebraic point of view rather than a
physical point of view. 

In this thesis we give an account of topological quantum computing from a
computational and mathematical point of view. 
