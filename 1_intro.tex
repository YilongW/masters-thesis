%\expand{A summary of the notions behind TQC. Bosonic and fermionic statistics. Introduction of the term ``anyon''. Give some history of the subject. Some of the results that TQC has given us (Jones polynomial universal for QC)}
%
%In \expand{year}, \expand{who?} found that certain quasiparticles in fractional
%quantum Hall states had statistics which were neither fermionic nor bosonic. In
%particular two anyon models which are nonabelian have been proposed: the Ising
%and Fibonacci anyons. There are modelled by the MTCs obtained from the quantum
%groups $U_q(\sll(2))$ at a fourth and fifth root of unity respectively. 
%
%Anyons are essentially described by MTCs. 
%
%\expand{Anyons are modelled by MTCs, and these MTCs are obtained from quantum groups at roots of unity}
%
%\todo{Prakash: Rowell says `only anyonic properties of bosonic systems can be described fully by MTCs'. Can you make some sense of this?}
%
The hope is to give a concrete account of the theory of modular tensor
categories obtained from quantum groups at roots of unity for computer
scientists, from a category theoretic and algebraic point of view rather than a
physical point of view. 

