Any anyonic model corresponds to a modular tensor category. In this chapter we
give a series of definitions leading up to the definition of a modular tensor
category. Before delving into categorical definitions, however, we give a
sketch of the correspondence between an anyonic model and a modular tensor
category to motivate the upcoming definitions. The basic idea is that anyons
correspond to objects and anyon types correspond to isomorphism classes of
simple objects. So,

\begin{itemize}
\item Anyons correspond to objects
\item In any anyonic model, there are a finite number of \emph{charges} or
      \emph{types}. These correspond to isomorphism classes of simple objects
\item Anyons are the basic entities of the theory, and any object is
constructed from them. In categorical terms, the category must be
\emph{semisimple}
\item Systems of several anyons correspond to tensor products of anyons: the
      category is \emph{monoidal}. These tensor products decompose as direct sums of
      anyons, e.g. 
\begin{equation}
a \otimes b \simeq 2a \oplus 3c
\end{equation}

      The physical interpretation of this is that if an anyon of type $a$ fuses
      with an anyon with type $b$, the result is either an anyon of type $a$,
      in one of two possible ways, or an anyon of type $b$, in one of three
      possible ways. The trivial charge corresponds to the tensor unit.
\item Exchanging anyons corresponds to braiding their worldlines, which in the
      category corresponds to applying an appropriate braiding isomorphism to
      the system of anyons. In other words, the category is \emph{braided}.
\begin{center}
\scalebox{1} % Change this value to rescale the drawing.
% picture of braiding worldlines
{
\begin{pspicture}(0,-1.4947689)(5.20875,1.5147688)
\psbezier[linewidth=0.02](4.04,0.7813313)(4.04,0.6240438)(4.22,0.23082528)(4.64,0.15218155)(5.06,0.073537834)(5.04,-0.10930765)(5.04,-0.30985886)
\psbezier[linewidth=0.02](5.04,0.7813313)(5.04,0.46675643)(4.96,0.27014711)(4.6,0.21509652)
\psbezier[linewidth=0.02](4.42,0.16004592)(4.16,0.104995325)(4.04,-0.16239332)(4.04,-0.30985886)
\psdots[dotsize=0.1](4.08,1.0213313)
\psdots[dotsize=0.1](5.08,1.0213313)
\rput(4.043125,1.31){$a$}
\rput(5.047969,1.35){$b$}
\psdots[dotsize=0.1](0.1,-0.018668715)
\psdots[dotsize=0.1](1.1,-0.018668715)
\rput(0.063125,0.27){$a$}
\rput(1.0679687,0.31){$b$}
\psarc[linewidth=0.02]{<-}(1.0,0.061331283){0.78}{199.98311}{174.28941}
\psbezier[linewidth=0.02](4.04,-0.30985886)(4.0,-0.6047751)(4.22,-0.845619)(4.64,-0.92426276)(5.06,-1.0029064)(5.04,-1.3332101)(5.06,-1.4747688)
\psbezier[linewidth=0.02](5.04,-0.30985886)(5.04,-0.6096879)(4.96,-0.8062972)(4.6,-0.8613478)
\psbezier[linewidth=0.02](4.42,-0.9163984)(4.16,-0.971449)(4.04,-1.2388376)(4.04,-1.4747688)
\rput(2.7798438,-0.028668715){$\iff$}
\end{pspicture} 
}
\end{center}
\item The conjugate of a charge corresponds to its \emph{dual}: the category is
      \emph{rigid}.
\item Anyons can be rotated, so that their worldlines become twisted. The
    corresponding mathematical formalism is that of a braided \emph{ribbon
    category}
\end{itemize}

To summarize, the category must be semisimple, monoidal, braided, rigid, and
ribbon. A category with all these properties that satisfies one additional
condition is called a modular tensor category. We will explain this condition
and discuss modular tensor categories in more detail in section
\ref{section:MTCs}. In this chapter, we give abstract definitions for all the
properties a MTC needs to satisfy. We also show that the category
of representations of any Hopf algebra is a rigid monoidal category, since the
modular tensor categories we will be interested in are constructed from
categories of representations of Hopf algebras.

\section{Monoidal categories}
\begin{defn}
    A \emph{monoidal category} is a category $\mathcal{C}$ with 
    \begin{enumerate}
    \renewcommand{\labelenumi}{\roman{enumi})}
        \item a bifunctor $\otimes: \mathcal{C} \times \mathcal{C} \to
            \mathcal{C}$
        \item a unit object $\one$ and natural transformations
            \begin{align}
                \lambda_V &: \one \otimes V \isomto V \\
                \rho_V &: V \otimes \one \isomto V
            \end{align}
        \item a natural transformation 

            \begin{equation}
                \alpha_{UVW} (U \otimes V) \otimes W \isomto U \otimes (V \otimes W)
            \end{equation}

        \item if $X_1, X_2$ are two objects obtained from $V_1 \otimes V_2
            \otimes \cdots V_n$ by inserting 1s and brackets, then all
            isomorphisms $\varphi: X_1 \isomto X_2$ composed of $\alpha$'s,
            $\lambda$'s, and $\rho$'s are equal. 
        \item $\one$ is a simple object and $\End_\mathcal{C} \one = k$
    \end{enumerate}

\end{defn}

In the context of topological quantum computing, the unit object should be
thought of as the trivial anyon. The fact that $\one \otimes V \isomto V$
in a natural way should be taken to mean that fusing a trivial anyon with any
other anyon has no effect.

\begin{example}
    \begin{enumerate}
    \renewcommand{\labelenumi}{\roman{enumi})}
        \item The category of $k$-vector spaces $\operatorname{Vec}(k)$
        \item The category of finite dimensional representations of a group, algebra, or Lie algebra
    \end{enumerate}
\end{example}

\section{Rigid monoidal categories}

A rigid monoidal category is a monoidal category where there is a notion of a
dual. This corresponds to the notion of charge conjugation: every anyon type
$T$ has a unique conjugate type $T^*$ which fuses with it to give the trivial
charge.

\begin{defn}
    Let $\mathcal{C}$ be a monoidal category, $V$ an object in $\mathcal{C}$. A
    \emph{right dual} to $V$ is an object $V^*$ with two morphisms

    \begin{align}
        e_V: V^* \otimes V \to \one  \\
        i_V: \one \to V^* \otimes V
    \end{align}
\end{defn}

such that the composition

\begin{equation}
    V \stackrel{ \id_V \otimes i_V }{\xrightarrow{\hspace*{1cm}}} V \otimes V^*
    \otimes V  \stackrel{\id_V \otimes e_V}{\xrightarrow{\hspace*{1cm}}} V
\end{equation}

is equal to $\id_V$, and similarly the composition

\begin{equation}
    V^* \stackrel{i_V \otimes \id_{V^*} }{\xrightarrow{\hspace*{1cm}}} V^*
    \otimes V \otimes V^*  \stackrel{e_V \otimes
    \id_{V^*}}{\xrightarrow{\hspace*{1cm}}} V^*
\end{equation}

is equal to $\id_{V^*}$

The morphisms $i_V$ and $e_V$ can be thought of as encoding the way in which
$V^*$ and $V$ fuse to yield the trivial charge.
   
\section{Braided monoidal categories}
\label{section:Braiding}

The central operation in topological quantum computing is that of exchanging
two anyons. Mathematically, the structure we need to model this is that of a
braided monoidal category: any computation on a set of anyons corresponds to
an element of the braid group. A braided monoidal category is a monoidal
category $\mathcal{C}$ with a natural transformation

\begin{equation}
    \sigma_{V,W} : V \otimes W \to W \otimes V
\end{equation}

For $V_1, V_2, \dots, V_n \in \mathcal{C}$, consider expressions of the form 

\begin{equation}
    ((V_{i_1}\otimes V_{i_2}) \otimes (V_{i_3} \otimes 1)) \otimes \cdots \otimes V_{i_n}
\end{equation}

obtained from $V_1,V_2,\ldots,V_n$ by adding brackets and $1$'s.

To any composition of $\alpha's, \lambda's, \rho's$, assign an element of the
braid group $B_n$ as follows:

\begin{align}
    \alpha, \lambda, \rho &\mapsto 1 \\
    \sigma_{V_{i_k}, V_{i_{k+1}}} &\mapsto b_k
\end{align}

\begin{defn}
    A \emph{braided tensor category} is a monoidal category with natural
    isomorphisms $\sigma_{X,Y}$ as above such that any $\varphi: X_1 \to X_2$
    obtained by composing $\alpha's, \lambda's, \rho's, \sigma's$ depends only
    on its image in $B_n$.
\end{defn}

\begin{defn}
    A braided tensor category is called \emph{symmetric} if $\sigma_{X,Y} \circ
    \sigma_{Y,X} = \id_{X \otimes Y}$ for any objects $X,Y$.
\end{defn}

Some examples of symmetric tensor categories include:

\begin{itemize}
    \item The category $\mathbf{Vect}$ of finite dimensional vector spaces over a field $k$.
\end{itemize}

\section{Ribbon categories}
\label{section:RibbonCategories}

Anyons can be rotated so that their worldlines become twisted. This requires
more structure than in a braided monoidal category.  In ribbon categories, it
is possible to define a natural twisting isomorphism $\theta_V: V\to V$ for
each $V$. It is also possible to define the notion of a \emph{trace}, which
will be important later on.

\begin{defn}

    A \emph{ribbon category} is a rigid braided tensor category with a natural
    isomorphism
    \begin{equation}
        \delta_V: V \to V^{**}
    \end{equation}

such that 
\begin{enumerate}
    \renewcommand{\labelenumi}{\roman{enumi})}

    \item $\delta_{V \otimes W} = \delta_V \otimes \delta_W$
    \item $\delta_1 = \id$
    \item $\delta_{V^*} = (\delta_V^*)^{-1}$
\end{enumerate}

\end{defn}

In any rigid braided tensor category, we can construct natural transformations 

\begin{equation}
\psi_V: V^{**} \to V
\end{equation}

via the composition

\begin{equation}
\xymatrix{
    V^{**} \ar[r]^-{i \otimes \id} & V \otimes V^* \otimes V^{**} \ar[r]^-{id \otimes \sigma^{-1}} & V \otimes V^{**} \otimes V^* \ar[r]^-{id \otimes e} &  V
    }
\end{equation}

Define the twist maps $\theta_V$ by

\begin{equation}
\theta_V = \psi_V \delta_V
\end{equation}

If $V$ is an object in a ribbon category $\mathcal{C}$ and $f$ an endomorphism
of $V$, we can define the trace of $f$ by the composition

\begin{equation}
    \xymatrix{
    \one \ar[r]^-{i_V} & V \otimes V^* \ar[r]^-{f \otimes \id} & V^* \otimes V \ar[r]^-{\delta_V \otimes \id} & V^{**} \otimes V^* \ar[r]^-{e_{V^*}} & 
    \one
    }
\end{equation}

We define the dimension of an object $V$ to be $\dim V = \tr \id_V$.

\section{Semisimple categories}
\begin{defn}
    A category $\mathcal{C}$ is called \emph{abelian} if it satisfies the conditions:

    \begin{enumerate}
    \renewcommand{\labelenumi}{\roman{enumi})}
        \item All the hom sets $\Hom(A,B)$ are $k$-vector spaces, and the composition
            
            \begin{equation}
                (\varphi, \psi) \mapsto \varphi \circ \psi
            \end{equation}

            is $k$-bilinear.
        \item There is a zero object $\mathbf{0} \in \Ob \mathcal{C}$ such that
            $\Hom(0,V) = \Hom(V,0) = 0$ for every object $V$
        \item Finite direct sums exist in $\mathcal{C}$
        \item Every morphism $\varphi$ has a kernel $\ker \varphi$ and a
            cokernel $\coker \varphi$. Every morphism is a composition of an
            epimorphism followed by a monomorphism. If $\ker \varphi = 0$, then
            $\varphi = \ker(\coker \varphi)$. If $\coker \varphi = 0$, then
            $\varphi = \coker(\ker \varphi)$.
    \end{enumerate}

    Examples of abelian categories include the category of $k$-vector spaces,
    the category of finite dimensional $k$-vector spaces, and the category of
    representations of a group $G$ over $k$.

\end{defn}

\begin{defn}
    An object $U$ in an abelian category is called $\emph{simple}$ if any
    injection $V \hookrightarrow U$ is either $0$ or an isomorphism.
\end{defn}

\begin{defn}
    An abelian category $\mathcal{C}$ is \emph{semisimple} if any object $V$ is isomorphic to a direct sum of simple objects

    \begin{equation}
        V \simeq \bigoplus{i} N_i V_i
    \end{equation}

    where the $V_i$ are simple objects, $N_i \in \mathbf{N}$

\end{defn}

    Suppose that $\mathcal{C}$ is a semisimple ribbon category. Let $I$ be the
    set of equivalence classes of nonzero simple objects in $\mathcal{C}$ and
    choose a representative $V_i$ for each equivalence class  $i \in I$.
    
    We can define the \emph{fusion coefficients} $N_{ij}^k \in \mathbf{N}$.

    \begin{equation}
        V_i \otimes V_j \simeq \bigoplus_k N_{ij}^k V_k
    \end{equation}

    We call each equation of this type a \emph{fusion rule}. 

\section{Modular tensor categories}
\label{section:MTCs}
    The mathematical structure that describes anyons is that of modular tensor
    categories. 
    The definition of modular tensor categories first appeared in \cite{MS}.
    They arise in conformal field theory and from quantum groups: we will
    explain how they arise from quantum groups in Chapter \ref{chapter:MTC}.
    They have applications to topological quantum field theory: Reshetikhin and
    Turaev associated a TQFT in 2+1 dimensions to every MTC. An exposition is
    in Bakalov \& Kirillov \ref{Kirillov}.

    Essentially modular tensor categories are a special type of braided tensor
    category.  The \emph{symmetric center} $\mathbf{Z}_2(\mathcal{C})$ of a
    braided category $\mathcal{C}$ is a full subcategory with 
    \begin{equation}
        \Ob \mathbf{Z}_2(\mathcal{C}) = \left\{ x \in \mathcal{C} : \sigma_{XY} \circ \sigma_{YX} = \id\ \forall\ Y \in \mathcal{C} \right\}
    \end{equation}
    It is a measure of ``how commutative'' $\mathcal{C}$ is. In a symmetric
    category, $\mathbb{Z}_2(\mathcal{C}) = \mathcal{C}$. A modular tensor
    category is in this sense the opposite of a symmetric category:

\begin{defn}
    A modular tensor category is a semisimple ribbon category such that
    $\mathbf{Z}_2(\mathcal{C})$ is trivial
\end{defn}

These categories are called modular because each one gives rise to a projective representation of the modular group $SL(2, \mathbb{Z})$.

In any semisimple rigid braided tensor category, define

\begin{equation}
    s_{X,Y} = \tr_{X \otimes Y}(\sigma_{Y,X} \circ \sigma_{X,Y})
\end{equation}

for every simple object $X,Y$.

It is proved in \cite{mueger2001} that
\begin{theorem}
    $\mathbf{Z}_2(\mathcal{C})$ is trivial if and only if the matrix $s$ is invertible. 
\end{theorem}

This was the original definition of a modular tensor category.

\section{The category of representations of a Hopf algebra}
\label{section:RepHopfAlgebra}

Suppose $(H, \mu, \eta, \Delta, \varepsilon)$ is a Hopf algebra with antipode $S$. 

Let $\Rep_f H$ be the category of finite dimensional representations of $H$ as
a $k$-algebra.

If $A$ is an algebra and $U, V$ are $A$-modules, then $U \otimes V$ is a vector
space, but there is no natural way to impose a $A$-module structure on $U
\otimes V$. 

The comultiplication $\Delta$ on $H$ allows us to impose a $H$-module structure
on the tensor product $U \otimes V$ of two $H$-modules $U,V$ as follows.

Suppose $\Delta(h) = \sum _{i} h^{(1)}_i \otimes h^{(2)}_i$. Then we define

\begin{equation}
    h (u \otimes v) = \sum_{i} h^{(1)}_i u \otimes h^{(2)}_i v
\end{equation}

We define the tensor unit using the counit $\varepsilon$: $\one$ is the
vector space $k$, with 

\begin{equation}
    h(1) = \varepsilon(h) 1
\end{equation}

 for any $h\in H$.

So we have that $\Rep_f(H)$ is a monoidal category, with this tensor product.
Only the counit and the comultiplication are required for this definition, so
in fact the category of representations of any bialgebra is a monoidal
category.

We can use the antipode $S$ to define duals as follows:

For any module $U$, let the dual $U^*$ be the dual vector space of linear
functionals on $U$, with action

\begin{equation}
    (h\cdot \varphi)(u)  = \varphi(S(h) u)
\end{equation}

It follows that $\Rep_f(H)$ is a rigid monoidal category. This also serves as
motivation for the definition of a Hopf algebra: it is an algebra with
additional structures such that its category of representations is monoidal and
rigid. 

% maybe say that the category of reps of the group has this structure, and it
% follows that the category of reps of the group algebra k[G] has this
% structure, and the Hopf algebra stuff is a way to generalize that

\subsection{Braided structure}
\label{bialgtocategory}
In this section we'll sketch the proof of the following theorem

\begin{theorem}
    Let $(H, \mu, \eta, \Delta, \varepsilon)$ be a bialgebra. The category $\Rep_f(H)$ is braided if and only if $H$ is braided. 
\end{theorem}
\begin{proof}
    Suppose $(H, \mu, \eta, \Delta, \varepsilon)$ has a universal $R$-matrix $R$. Let $V,W$ be two $H$-modules, and $R = \sum_{i} s_i \otimes t_i$. 

    We can define a natural isomorphism $\sigma_{V,W}^R$ between $V \otimes W$ and $W \otimes V$ by

    \begin{equation}
        \sigma_{V,W}^R = \tau_{V,W}(R(v \otimes w)) = \sum_{i} t_i w \otimes s_i v
    \end{equation}
    % proof that this works: p. 178, 179, Kassel

    Conversely, let $(H, \mu, \eta, \Delta, \varepsilon)$ be a bialgebra, and suppose the category $\Rep_f(H)$ has a braiding $\sigma$. Define 

    \begin{equation}
        R = \tau_{H,H}(\sigma_{H,H}(1 \otimes 1))
    \end{equation}

    % a proof that this works can be found in Kassel, p. 318
\end{proof}


