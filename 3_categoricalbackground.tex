Throughout we will again be working over a field $k$ of characteristic 0.

\section{Monoidal Categories}
\begin{defn}
    A \emph{monoidal category} is a category $\mathcal{C}$ with 
    \begin{enumerate}
    \renewcommand{\labelenumi}{\roman{enumi})}
        \item a bifunctor $\otimes: \mathcal{C} \times \mathcal{C} \to
            \mathcal{C}$
        \item a unit object $\mathbf{1}$ and natural transformations
            \begin{align}
                \lambda_V &: \mathbf{1} \otimes V \isomto V \\
                \rho_V &: V \otimes 1 \isomto V
            \end{align}
        \item a natural transformation 

            \begin{equation}
                \alpha_{UVW} (U \otimes V) \otimes W \isomto U \otimes (V \otimes W)
            \end{equation}

            which satisfy the associativity property :

            \begin{equation}
                \xymatrix{
&  (A \otimes (B \otimes C)) \otimes D \ar[rd]^-{\alpha_{A, B \otimes C, D}} & \\
( (A \otimes B) \otimes C ) \otimes D \ar[ru]^-{\alpha_{A,B,C}\otimes 1_D}  \ar[d]^-{\alpha_{A \otimes B, C, D}} & & A \otimes ( (B \otimes C) \otimes D) \ar[d]^-{1_A \otimes \alpha_{B,C,D}} \\
                (A \otimes B) \otimes (C \otimes D) \ar[rr]^- {\alpha_{A,B,C \otimes D}} & & A \otimes (B \otimes (C \otimes D))
                }
            \end{equation}

            commutes for all objects $A,B,C,D$.

            \todo{ add the triangle condition, say that this next thing is a theorem.}

        \item if $X_1, X_2$ are two objects obtained from $V_1 \otimes V_2
            \otimes \cdots V_n$ by inserting 1s and brackets, then all
            isomorphisms $\varphi: X_1 \isomto X_2$ composed of $\alpha$'s,
            $\lambda$'s, and $\rho$'s are equal. 
        \item $\mathbf{1}$ is a simple object and $\End_\mathcal{C} \mathbf{1} = k$
    \end{enumerate}

\end{defn}

\begin{example}
    \begin{enumerate}
    \renewcommand{\labelenumi}{\roman{enumi})}
        \item The category of $k$-vector spaces $\operatorname{Vec}(k)$
        \item The category of finite dimensional representations of a group, algebra, or Lie algebra
    \end{enumerate}
\end{example}
    
\section{Braided monoidal categories}

Let $\mathcal{C}$ be a monoidal category with a natural transformation 

\begin{equation}
    \sigma_{VW} : V \otimes W \to W \otimes V
\end{equation}

\section{Rigid monoidal categories}
\todo{ write down the vector space example?}

A rigid monoidal category is a monoidal category where there is a notion of a dual. 

\begin{defn}
    Let $\mathcal{C}$ be a monoidal category, $V$ an object in $\mathcal{C}$. A
    \emph{right dual} to $V$ is an object $V^*$ with two morphisms

    \begin{align}
        e_V: V^* \otimes V \to \mathbf{1}  \\
        i_V: \mathbf{1} \to V^* \otimes V
    \end{align}
\end{defn}

such that the composition

\begin{equation}
    V \stackrel{i_v \otimes \id_V}{\xrightarrow{\hspace*{1cm}}} V \otimes V^*
    \otimes V  \stackrel{\id_V \otimes e_V}{\xrightarrow{\hspace*{1cm}}} V
\end{equation}

is equal to $\id_V$, and similarly the composition

\begin{equation}
    V^* \stackrel{id_{V^*} \otimes \id_V}{\xrightarrow{\hspace*{1cm}}} V^*
    \otimes V \otimes V^*  \stackrel{e_V \otimes
    \id_{V^*}}{\xrightarrow{\hspace*{1cm}}} V^*
\end{equation}

is equal to $\id_{V^*}$


\section{The Category of Representations of a Hopf Algebra}

Suppose $(H, \mu, \eta, \Delta, \varepsilon)$ is a Hopf algebra with antipode $S$. 

Let $\Rep_f H$ be the category of finite dimensional representations of $H$ as
a $k$-algebra.

If $A$ is an algebra and $U, V$ are $A$-modules, then $U \otimes V$ is a vector
space, but there is no natural way to impose a $A$-module structure on $U
\otimes V$. 

The comultiplication $\Delta$ on $H$ allows us to impose a $H$-module structure
on the tensor product $U \otimes V$ of two $H$-modules $U,V$ as follows.

Suppose $\Delta(h) = \sum _{i} h^{(1)}_i \otimes h^{(2)}_i$. Then we define

\begin{equation}
    h (u \otimes v) = \sum_{i} h^{(1)}_i u \otimes h^{(2)}_i v
\end{equation}

We define the tensor unit using the counit $\varepsilon$: $\mathbf{1}$ is the
vector space $k$, with 

\begin{equation}
    h(1) = \varepsilon(h) 1
\end{equation}

 for any $h\in H$.

So we have that $\Rep_f(H)$ is a monoidal category, with this tensor product.
Only the counit and the comultiplication are required for this definition, so
in fact the category of representations of any bialgebra is a monoidal
category.

We can use the antipode $S$ to define duals as follows:

For any module $U$, let the dual $U^*$ be the dual vector space of linear
functionals on $U$, with action

\begin{equation}
    (h\cdot \varphi)(u)  = \varphi(S(h) u)
\end{equation}

It follows that $\Rep_f(H)$ is a rigid monoidal category. This also serves as
motivation for the definition of a Hopf algebra: it is an algebra with
additional structures such that its category of representations is monoidal and
rigid. 

% maybe say that the category of reps of the group has this structure, and it
% follows that the category of reps of the group algebra k[G] has this
% structure, and the Hopf algebra stuff is a way to generalize that

\subsection{Braided structure}
\label{bialgtocategory}
In this section we'll sketch the proof of the following theorem

\begin{theorem}
    Let $(H, \mu, \eta, \Delta, \varepsilon)$ be a bialgebra. The category $\Rep_f(H)$ is braided if and only if $H$ is braided. 
\end{theorem}
\begin{proof}
    Suppose $(H, \mu, \eta, \Delta, \varepsilon)$ has a universal $R$-matrix $R$. Let $V,W$ be two $H$-modules, and $R = \sum_{i} s_i \otimes t_i$. 

    We can define a natural isomorphism $\sigma_{V,W}^R$ between $V \otimes W$ and $W \otimes V$ by

    \begin{equation}
        \sigma_{V,W}^R = \tau_{V,W}(R(v \otimes w)) = \sum_{i} t_i w \otimes s_i v
    \end{equation}
    % proof that this works: p. 178, 179, Kassel

    Conversely, let $(H, \mu, \eta, \Delta, \varepsilon)$ be a bialgebra, and suppose the category $\Rep_f(H)$ has a braiding $\sigma$. Define 

    \begin{equation}
        R = \tau_{H,H}(\sigma_{H,H}(1 \otimes 1))
    \end{equation}

    % a proof that this works can be found in Kassel, p. 318
\end{proof}

\section{Ribbon Categories}

In what follows it will become important to have a notion of the trace of a
morphism. The definitions that follow will allow us to define a trace in a
certain class of rigid monoidal categories called ribbon categories.

\begin{defn}

    A \emph{ribbon category} is a rigid braided tensor category with a natural
    isomorphism
    \begin{equation}
        \delta_V: V \to V^{**}
    \end{equation}

such that 
\begin{enumerate}
    \renewcommand{\labelenumi}{\roman{enumi})}

    \item $\delta_{V \otimes W} = \delta_V \otimes \delta_W$
    \item $\delta_1 = \id$
    \item $\delta_{V^*} = (\delta_V^*)^{-1}$
\end{enumerate}

\end{defn}

If $V$ is an object in a ribbon category $\mathcal{C}$ and $f$ an endomorphism
of $V$, we can define the trace of $f$ by the composition

\begin{equation}
    \xymatrix{
    \mathbf{1} \ar[r]^-{i_V} & V \otimes V^* \ar[r]^-{f \otimes \id} & V^* \otimes V \ar[r]^-{\delta_V \otimes \id} & V^{**} \otimes V^* \ar[r]^-{e_{V^*}} & 
    \mathbf{1}
    }
\end{equation}

We define the dimension of an object $V$ to be $\dim V = \tr \id_V$.

\section{Semisimple Categories}
\begin{defn}
    A category $\mathcal{C}$ is called \emph{abelian} if it satisfies the conditions:

    \begin{enumerate}
    \renewcommand{\labelenumi}{\roman{enumi})}
        \item All the hom sets $\Hom(A,B)$ are $k$-vector spaces, and the compositions
            
            \begin{equation}
                (\varphi, \psi) \mapsto \varphi \circ \psi
            \end{equation}

            are $k$-bilinear.
        \item There is a zero object $\mathbf{0} \in \Ob \mathcal{C}$ such that
            $\Hom(0,V) = \Hom(V,0) = 0$ for every object $V$
        \item Finite direct sums exist in $\mathcal{C}$
        \item Every morphism $\varphi$ has a kernel $\ker \varphi$ and a
            cokernel $\coker \varphi$. Every morphism is a composition of an
            epimorphism followed by a monomorphism. If $\ker \varphi = 0$, then
            $\varphi = \ker(\coker \varphi)$. If $\coker \varphi = 0$, then
            $\varphi = \coker(\ker \varphi)$.
    \end{enumerate}

    Examples of abelian categories include the category of $k$-vector spaces,
    the category of finite dimensional $k$-vector spaces, and the category of
    representations of a group $G$ over $k$.

\end{defn}

\begin{defn}
    An object $U$ in an abelian category is called $\emph{simple}$ if any
    injection $V \hookrightarrow U$ is either $0$ or an isomorphism.
\end{defn}

\begin{defn}
    An abelian category $\mathcal{C}$ is \emph{semisimple} if any object $V$ is isomorphic to a direct sum of simple objects

    \begin{equation}
        V \simeq \bigoplus{i} N_i V_i
    \end{equation}

    where the $V_i$ are simple objects, $N_i \in \mathbf{N}$

\end{defn}

    Suppose that $\mathcal{C}$ is a semisimple ribbon category. Let $I$ be the
    set of equivalence classes of nonzero simple objects in $\mathcal{C}$ and
    choose a representative $V_i$ for each equivalence class  $i \in I$.
    
    We can define the \emph{fusion coefficients} $N_{ij}^k \in \mathbf{N}$.

    \begin{equation}
        V_i \otimes V_j \simeq \bigoplus_k N_{ij}^k V_k
    \end{equation}

    We call each equation of this type a \emph{fusion rule}. 

    \section{Modular Tensor Categories}

The \emph{dimension} of a category $\mathcal{C}$ is defined by 

\begin{equation}
    \operatorname{dim}(\mathcal{C}) = \sum_{ x \in \Ob \mathcal{C}} (\operatorname{dim} x)^2
\end{equation}


\begin{defn}
    The \emph{symmetric center} of a braided category $\mathcal{C}$
    $\mathbf{Z}_2(\mathcal{C})$ is a full subcategory with 

    \begin{equation}
        \Ob \mathbf{Z}_2(\mathcal{C}) = \left\{ x \in \mathcal{C} : \sigma_{XY} \circ \sigma_{YX} = \id\ \forall Y \in \mathcal{C} \right\}
    \end{equation}
\end{defn}

\begin{defn}
    A modular tensor category is a semisimple ribbon category such that
    $\mathbf{Z}_2(\mathcal{C})$ is trivial (every object is isomorphic to
    $\mathbf{1}^{\oplus n}$). Equivalently, every simple object in the center
    is isomorphic to $\mathbf{1}$.  
\end{defn}

Why are these categories called modular? 
\todo{ finish this }
Define 

\begin{equation}
    s_{X,Y} = \tr_{X \otimes Y}(\sigma_{Y,X} \circ \sigma_{X,Y})
\end{equation}

for every simple object $X,Y$.

\begin{theorem}
    \todo{ find a citation for this}

    $\mathbf{Z}_2(\mathcal{C})$ is trivial if and only if the matrix $s$ is invertible. 
\end{theorem}

\begin{example}
    An elementary class of examples of modular tensor categories can be
    obtained via the quantum double construction as follows. This is a simple
    example of Drinfeld's quantum double construction which can be applied to
    any finite dimensional Hopf algebra.

    \todo{ does it need to be finite dimensional?}


    Recall the Hopf algebra $k[G]$ of any finite group $G$ from
    \ref{groupalgebra} with basis $\left\{ v_g \right\}_{g \in G}$ and

    \begin{align}
        &\text{multiplication }   & v_g v_h = v_{gh} \\
        &\text{unit }             & e \\
        &\text{comultiplication } & \Delta(v_g) = v_g \otimes v_g \\
        &\text{counit }           & \varepsilon(v_g) = 1 \\
        &\text{antipode }         & S(v_g) = v_{g^-1}
    \end{align}

    The dual Hopf algebra to $k[G]$ is the function algebra $F(G)$ with basis
    $\left\{ \delta_g : g \in G \right\}$ of functions

    \begin{equation}
        \delta_g(x) = \delta_{g,x} = \begin{cases} 1 &\text{for $g = x$} \\ 0 &\text{for $g \neq x$} \end{cases}
    \end{equation}

    \begin{align}
        &\text{multiplication} &\delta_g \delta_h = \delta_{g,h} \delta_g\\
        &\text{unit}           &\sum_{g \in G} \delta_g \\
        &\text{comultiplication} &\Delta(\delta_g) = \sum_{g_1 g_2 = g} \delta_{g_1} \otimes \delta_{g_2}\\
        &\text{counit}           &\varepsilon(\delta_g) = \delta_{e,g} \\
        &\text{antipode}         &S(\delta_g) = \delta_{g^-1}
    \end{align}


    The quantum double $D(G) of $k(G)$ can be described as follows. $D(G)$ is
    the Hopf algebra with vector space $F(G)$ \otimes_k k[G]$ and 

    \begin{align}
        &\text{multiplication} &(\delta_g\otimes v_x) (\delta_h \otimes v_y) = \delta_{gx,xh} (\delta_g \otimes v_{xy})\\
        &\text{unit}           &\sum_{g \in G} \delta_g  \otimes v_e\\
        &\text{comultiplication} &\Delta(\delta_g \otimes v_x) = \sum_{g_1 g_2 = g} (\delta_{g_1} \otimes v_x) \otimes (\delta_{g_2} \otimes v_x) \\
        &\text{counit}           &\varepsilon(\delta_g \otimes v_x) = \delta_{e,g} \\
        &\text{antipode}         &S(\delta_g \otimes v_x) = \delta_{x^{-1}g^-1 x} \otimes v_{x^{-1}}
    \end{align}


\end{example}



