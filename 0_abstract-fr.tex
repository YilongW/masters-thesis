Le calcul quantique topologique est une approche au problème d'implementation
de circuits quantique d'une façon robuste et precisé.  L'idée s'agit
d'exploiter certaines propriétés de quasiparticules, dites "anyons", pour
obtenir une implémentation du calcul quantique qui est intrinsequement
tolerante aux pannes. La structure mathématique qui décrit ces anyons est celle
des catégories modulaires.  Ces objets peuvent être construites a partir de
representations de certaines algèbres, appelées groupes quantiques.  Dans ce
mémoire, nous donnerons une exposition des catégories modulaires, des groupes
quantiques et du lien qu'ils partagent avec le calcul quantique. Le mémoire ne
devrait requérir qu'une connaissance de base en algèbre et en théorie des
categories. L'espoir étant de donner un model concret pour les informaticiens
de la théorie de catégories obtenus à partir de groupes quantiques.  L'emphase
sera sur le point de vu algèbrique et catégorique plutôt que celui physique.
