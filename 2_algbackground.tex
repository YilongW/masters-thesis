The setting for topological quantum computing is a quotient of the category of
representations of a deformation of a semisimple Lie algebra. In this chapter
we define and discuss some relevant results about the required algebraic
structures. Everything in this section takes place over a field of
characteristic 0.

\section{Quantum algebra}

In the definitions and results involving quantum groups we will frequently
encounter so-called `quantum integers'. In this section we will provide some
definitions and motivation.

\begin{equation}
    [n]_q = \frac{q^n - q^{-n}}{q - q^{-1}} = q^{n-1} + q^{n-3} + \cdots + q^{-n+3} + q^{-n+1}
\end{equation}

\begin{equation}
    [n]!_q = [n]_q [n-1]_q \cdots [1]_q
\end{equation}

$q^{2d}=1$ if and only if $[d]_q = 0$, so $[n]_q \neq 0$ for every nonzero
integer when $q$ is not a root of unity. When $q$ is an $\ell^\text{th}$ root of unity, $[n]_q = 0$ when $\ell$ divides $n$.


\section{Representations of algebras}

Throughout we will need some basic definitions and results about the
representation theory of associative algebras. An associative algebra $A$ over a field $k$ is a
$k$-vector space which also has the structure of a ring. Some important examples of $k$-algebras are: 

\begin{enumerate}
    \item The $n \times n$ matrices over $k$
    \item The group algebra of a finite group (defined in \ref{groupalgebra})
    \item The polynomial algebra $k[x]$
    \item The universal enveloping algebra of a Lie algebra (see \ref{UnivEnvAlg})
\end{enumerate}

A module for an algebra $A$ is the same as a representation of the algebra, and
we will use the terms interchangeably. 

\begin{defn}
    An $A$-\emph{module} for an algebra $A$ is a vector space $M$ with a binary
    operation $A \times M \to M$ written $r\cdot n$ or $rn$ such that for $r,s \in A$ and $m,n \in M$:

    \begin{enumerate}
        \item $r(n + m) = rn + rm$
        \item $(rs)n = r(sn)$
        \item $1n = n$
        \item $(r+s)n = rn + sn$
    \end{enumerate}
\end{defn}

\begin{defn}
    A subspace $N \subset M$ of a module $M$ is a \emph{submodule} if $an \in N$ for every $a \in A, n\in N$.
\end{defn}

\begin{defn}
    A submodule $N$ of a module $M$ is called \emph{maximal} if there is no submodule $N'$ of $M$ such that $N \subsetneq N' \subsetneq M$.
\end{defn}

\begin{defn}
    We say that a module is \emph{simple} or \emph{irreducible} if it has no nontrivial submodules.
\end{defn}

\section{Lie Algebras}

The examples of quantum groups that we are interested in are obtained as
deformations of semisimple Lie algebras. We will give some basic definitions of
Lie algebras, ideals, simplicity, and semisimplicity, and discuss briefly the
classification of semisimple Lie algebras. Much more detail can be found in
\cite{Humphreys1973}

Throughout this section $\mathfrak{g}$ will denote a Lie algebra.

A \emph{Lie algebra} is a vector space $\mathfrak{g}$ with a bilinear operation
$\left[ \cdot, \cdot \right]$  $\mathfrak{g} \times \mathfrak{g} \to
\mathfrak{g}$ such that for any $x,y,z\in \mathfrak{g}$,

\begin{itemize}
    \item $\left[ x,y \right] = -\left[ y,x \right]$
    \item $\left[ x, \left[ y,z \right] \right] + \left[ y, \left[ z,x \right] \right] + \left[ z, \left[ x,y \right] \right] = 0$ (the Jacobi identity)
\end{itemize}

A basic example of a Lie algebra is $\sll(2)$: the algebra of $2 \times 2$
matrices with trace zero. $\sll(2)$ is generated by the matrices 

\begin{equation}
    X = \begin{pmatrix} 0 & 1 \\ 0 & 0 \end{pmatrix},
    Y = \begin{pmatrix} 0 & 0 \\ 1 & 0 \end{pmatrix}, 
    H = \begin{pmatrix} 1 & 0 \\ 0 &-1 \end{pmatrix}
\end{equation}

In general any finite dimensional Lie algebra over a
field of characteristic zero can be realized as a vector space of matrices such
that 

\[ \left[ X,Y \right] = XY - YX\] 
for any $X,Y$ (Ado's theorem), so the bracket $[X,Y]$ can safely be thought of
as the commutator $XY - YX$. 


The subject of general Lie algebras is fascinating, but we will only be
interested here in semisimple Lie algebras, which are a well-characterized
class of Lie algebras. 

\begin{defn}
    A subspace $\mathfrak{h} \subset \mathfrak{g}$ of a Lie algebra is a
    \emph{Lie subalgebra} if $\mathfrak{h}$ is closed under the Lie bracket. 

    An \emph{ideal} $I$ of $\mathfrak{g}$ is a subalgebra such that $[\mathfrak{g}, I] \subset I$
\end{defn}

\begin{defn}
    A Lie algebra is \emph{simple} if it has no proper ideals and is not
    commutative. 
\end{defn}

$\sll(2)$ is the lowest-dimensional simple Lie algebra. In fact, it is the
basic example of a simple Lie algebra: the study of its representation theory
is a key factor in the study of the representation theory of all simple Lie
algebras.
%todo: write this better?

\begin{defn}
    A Lie algebra is \emph{semisimple} if it is the direct sum of simple Lie algebras.
\end{defn}

\section{Representations of $\sll(2)$}

A \emph{representation} of a Lie algebra $\mathfrak{g}$ is a vector space $V$
together with an action of $\mathfrak{g}$ on $V$ such that 

\begin{align*}
    \left[ x,y \right] v &= x(yv) - y(xv) \\
    (x+y)v &= xv + yv \\
    (ax)v &= a(xv)
\end{align*}

for all $x,y \in \mathfrak{g}, v \in V, a \in \mathbb{C}$

A representation is called \emph{irreducible} or \emph{simple} if it has no
nontrivial subspaces invariant under the action of $\mathfrak{g}$.

The representations of $\sll(2)$ can be classified as follows: 

For each integer $n \geq 0$, there is a unique (up to isomorphism)
simple representation of $\sll(2)$ of dimension $n+1$. This representation has
basis
$v_0, \ldots, v_{n}$ such that

% from Kassel, p. 101
\begin{align*}
    &H v_i = (n - 2i) v_i& \\
    &Y v_i = \begin{cases} 
                (i+1)v_{i+1}& \text{ for $i < m$} \\
                0& \text{ for $i = m$} \\
            \end{cases} \\
    &X v_i = \begin{cases} 
                (n-i+1)v_{i-1}& \text{ for $i > 0$} \\
                0& \text{ for $i = 0$} \\
            \end{cases}
\end{align*}



\section{Hopf algebras}
\subsection{Algebras and Coalgebras}
For a field $k$, a \emph{$k$-algebra} $A$ is a $k$-vector space with an
associative bilinear mapping $A \times A \to A$ which has an identity element
$1 \in A$ such that $1\cdot x = x\cdot 1 = x$ for any $x \in A$.

Put in categorical terms, a $k$-algebra is given by a triple $(A, \mu, \eta)$,
where $A$ is a vector space, and $\mu: A \otimes A \to A$ and $\eta: k \to A$
are linear maps satisfying the axioms:

Associativity:
The diagram
\begin{equation}
\xymatrix{
A \otimes A \otimes A \ar[d]^-{\id \otimes \mu} \ar[r]^-{\mu \otimes \id} & A \otimes A \ar[d]^-{\mu}\\
 A \otimes A \ar[r]^-\mu & A 
}
\end{equation}
commutes.

Unit: 

The diagram

\begin{equation}
    \xymatrix{
    k \otimes A \ar[r]^-{\eta \otimes \id} \ar[rd]_\simeq & A \otimes A \ar[d]^-\mu & A \otimes k \ar[l]_{\id \otimes \eta} \ar[ld]^-{\simeq} \\
    & A &
    }
\end{equation}
commutes.

An algebra is called \emph{commutative} if $x \cdot y = y \cdot x$ for any
$x,y\in A$. In other terms, it needs to satisfy the commutativity axiom:

The triangle 

\begin{equation}
    \xymatrix{
    A \otimes A \ar[rr]^-{\tau_{A,A}} \ar[rd]_\mu & & A \otimes A \ar[ld]^-\mu \\
    & A &
    }
\end{equation}

Given two algebras $(A_1, \mu_1, \eta_1)$ and $(A_2, \mu_2, \eta_2)$, a linear
map $f: A_1 \to A_2$ is called a \emph{morphism of algebras} or a
\emph{algebra homomorphism} if $f(\mu_1(a,b)) = \mu_2(f(a), f(b))$ for any
$a,b\in A_1$ and $f(\eta_1(1)) = \eta_2(1)$. In other words, the following two
diagrams need to commute:

\begin{equation}
    \xymatrix{
    1 \ar[r]^-{\eta_1} \ar[rd]_{\eta_2} & A_1 \ar[d]^-f \\
    & A_2
    }
\end{equation}
\begin{equation}
    \xymatrix{
    A_1 \ar[r]^-f & A_2 \\
    A_1 \otimes A_1 \ar[u]^-{\mu_1} \ar[r]_-{f\otimes f} & A_2 \otimes A_2 \ar[u]^-{\mu_2}
    }
\end{equation}


We can obtain the definition of a coalgebra by reversing all the arrows as follows:

\begin{defn}
    A \emph{coalgebra} is a triple $(C, \Delta, \varepsilon)$ where $C$ is a
    vector space, and $\Delta: C \to C \otimes C$, $\varepsilon: C \to k$ are
    linear maps satisfying the axioms:

Associativity:
The diagram
\begin{equation}
\xymatrix{
C \otimes C \otimes C   & C \otimes C \ar[l]_-{\Delta \otimes \id}\\
 C \otimes C\ar[u]^-{\id \otimes \Delta}  & C \ar[l]_-\Delta \ar[u]_-{\Delta}
}
\end{equation}
commutes.

Unit: 

The diagram

\begin{equation}
    \xymatrix{
    k \otimes C  & C \otimes C \ar[l]_-{\varepsilon \otimes \id} \ar[r]^-{\id \otimes \varepsilon} & C \otimes k   \\
    & C \ar[u]_-\Delta \ar[lu]^-\simeq \ar[ru]_-{\simeq}&
    }
\end{equation}
commutes.
\end{defn}
A coalgebra is called \emph{cocommutative} if the triangle 

\begin{equation}
    \xymatrix{
    A \otimes A   & & A \ar[ll]^-{\tau_{A,A}} \otimes A  \\
    & \ar[lu]^-\Delta A \ar[ru]_-\Delta&
    }
\end{equation}

commutes.

A linear map $f: C_1 \to C_2$ between two coalgebras $(C_1, \Delta_1,
\varepsilon_1)$, $(C_2, \Delta_2, \varepsilon_2)$ is a coalgebra homomorphism
if the following two diagrams commute: 


\begin{equation}
    \xymatrix{
    1 \ar[r]^-{\eta_1} \ar[rd]_{\eta_2} & A_1 \ar[d]^-f \\
    & A_2
    }
\end{equation}
\begin{equation}
    \xymatrix{
    A_1 \ar[r]^-f & A_2 \\
    A_1 \otimes A_1 \ar[u]^-{\mu_1} \ar[r]_-{f\otimes f} & A_2 \otimes A_2 \ar[u]^-{\mu_2}
    }
\end{equation}



\subsection{Bialgebras}

Suppose $H$ is a vector space which has both an algebra structure $(H, \mu,
\eta)$ and a coalgebra structure $(H, \Delta, \varepsilon)$. We call this a
\emph{bialgebra} if the two structures are compatible in the following sense:

\begin{defn}
    A vector space with an algebra and coalgebra structure is called a
    \emph{bialgebra} if one of the following two equivalent conditions holds:

    \begin{enumerate}
        \item The maps $\mu$ and $\eta$ are morphisms of coalgebras
        \item The maps $\Delta$ and $\varepsilon$ are morphisms of algebras
    \end{enumerate}
\end{defn}

A morphism of bialgebras is a map which is both a morphism of algebras and a
morphism of coalgebras.
\subsection{Hopf Algebras}

% todo: introduce Sweedler notation??

\begin{defn}
    Let $(H, \mu, \eta, \Delta, \varepsilon)$ be a bialgebra. An endomorphism
    $S$ of $H$ is called a \emph{antipode} for the bialgebra if the following
    two diagrams commute:

    \begin{equation}
        \xymatrix{
        H \ar[r]^-{\Delta} \ar[rrd]_-{\varepsilon}& H \otimes H \ar[rr]^-{S \otimes \id} & & H \otimes H \ar[r]^-{\mu} & H \\
        & & k \ar[rru]_-{\eta}& &
        }
    \end{equation}

    \begin{equation}
        \xymatrix{
        H \ar[r]^-{\Delta} \ar[rrd]_-{\varepsilon}& H \otimes H \ar[rr]^-{\id \otimes S} & & H \otimes H \ar[r]^-{\mu} & H \\
        & & k \ar[rru]_-{\eta}& &
        }
    \end{equation}
\end{defn}

\begin{defn}
    A \emph{Hopf algebra} is a bialgebra with an antipode. A morphism of Hopf
    algebras is a morphism between the bialgebras which commutes with the
    antipodes. 
\end{defn}

\begin{example}
    \label{groupalgebra}
    One important example of a Hopf algebra is the Hopf algebra obtained from
    any finite group. 

    Let $G$ be a finite group. The Hopf algebra $k[G]$ is the algebra with
    basis $\left\{ v_g: g \in G \right\}$, multiplication $v_g v_h = v_{gh}$,
    and unit $v_1$. 

    We can define a coalgebra structure on $k[G]$ by $\Delta(v_g) = v_g \otimes v_g$
    and $\varepsilon(v_g) = 1$ for any $g \in G$.

    The antipode $S$ is given by $S(v_g) = v_{g^{-1}}$ for any $g \in G$.

    Note that the group algebra is not commutative if the underlying group is
    not commutative, but is always cocommutative. The examples of Hopf
    algebras we will be interested in later are neither commutative nor
    cocommutative. 
\end{example}

In general the antipode map can be thought of as an analog to the inverse map
in a group. If a Hopf algebra has an antipode, it is unique, and $S^2 = \id$.

\begin{example}
    \label{UnivEnvAlg}
    Given a Lie algebra $\mathfrak{g}$, we can define an associative algebra
    called the \emph{universal enveloping algebra} of $\mathfrak{g}$. The
    universal enveloping algebra of any Lie algebra is a Hopf algebra.

    The \emph{tensor algebra} $T(V)$ of any $k$-vector space $V$ is an associative
    algebra defined by

    \begin{equation}
        T(V) = k \oplus \bigoplus_{n=1}^\infty \underbrace{(V \otimes \cdots \otimes V)}_{\text{$n$ times}}
    \end{equation}

    with multiplication $v \cdot w = v \otimes w$.

    %todo: this doesn't feel well explained :(

    If we have a Lie algebra $\mathfrak{g}$, then we can define an ideal
    $I(\mathfrak{g})$ of the tensor algebra $T(\mathfrak{g})$ generated by all
    elements of the form $(xy - yx) - \left[ x,y \right]$ for $x,y \in
    \mathfrak{g}$.

    We define the universal enveloping algebra to be 

    \begin{equation}
        U(\mathfrak{g}) = T(\mathfrak{g}) / I(\mathfrak{g})
    \end{equation}

    We can put a Hopf algebra structure on the enveloping algebra
    $U(\mathfrak{g})$ for any Lie algebra $\mathfrak{g}$, with $\Delta,
    \varepsilon, S$ defined by: 
    \begin{align}
        \Delta(x) &= x \otimes 1 + 1 \otimes x \\
        \varepsilon(x) &= 0 \\
        S(x) &= -x  
    \end{align}

    for $x \in \mathfrak{g}$.

\end{example}


