The setting for topological quantum computing is a quotient of the category of
representations of a deformation of a semisimple Lie algebra. In this chapter
we define and discuss some relevant results about the required algebraic
structures. Everything in this section takes place over a field $k$ of
characteristic 0.

\section{Quantum algebra}

In the definitions and results involving quantum groups we will frequently
encounter so-called `quantum integers'. In this section we will provide some
definitions and motivation.
\todo{provide the motivation}

\begin{equation}
    [n]_q = \frac{q^n - q^{-n}}{q - q^{-1}} = q^{n-1} + q^{n-3} + \cdots + q^{-n+3} + q^{-n+1}
\end{equation}

\begin{equation}
    [n]!_q = [n]_q [n-1]_q \cdots [1]_q
\end{equation}

$q^{2d}=1$ if and only if $[d]_q = 0$, so $[n]_q \neq 0$ for every nonzero
integer when $q$ is not a root of unity. When $q$ is an $\ell^\text{th}$ root
of unity, $[n]_q = 0$ when $\ell$ divides $n$.


\section{Representations of algebras}

We will model anyons by representations of certain algebras called quantum
groups. We therefore give a few key definitions about algebras and their
representations here.  

An associative algebra $A$ over a field $k$ is a $k$-vector space which also
has the structure of a ring. Some important examples of $k$-algebras are: 

\begin{enumerate}
    \item $k$ itself
    \item The $n \times n$ matrices over $k$, $M_n(k)$
    \item The group algebra of a finite group (defined in \ref{example:groupalgebra})
    \item The algebra $k[x]$ of polynomials in $x$ over $k$
    \item The universal enveloping algebra of a Lie algebra (see \ref{example:UEA})
\end{enumerate}

A module for an algebra $A$ is the same as a representation of the algebra, and
we will use the terms interchangeably. 

\begin{defn}
    An $A$-\emph{module} for an algebra $A$ is a vector space $M$ with a binary
    operation $A \times M \to M$ written $r\cdot n$ or $rn$ such that for $r,s
    \in A$ and $m,n \in M$:

    \begin{enumerate}
        \item $r(n + m) = rn + rm$
        \item $(rs)n = r(sn)$
        \item $1n = n$
        \item $(r+s)n = rn + sn$
    \end{enumerate}
\end{defn}

\begin{defn}
    A subspace $N \subset M$ of a module $M$ is a \emph{submodule} if $an \in
    N$ for every $a \in A, n\in N$.
\end{defn}

\begin{defn}
    A submodule $N$ of a module $M$ is called \emph{maximal} if there is no
    submodule $N'$ of $M$ such that $N \subsetneq N' \subsetneq M$.
\end{defn}

\begin{defn}
    We say that a module is \emph{simple} or \emph{irreducible} if it has no
    nontrivial submodules.
\end{defn}

\section{Lie Algebras}

The examples of quantum groups that we are interested in are obtained as
deformations of semisimple Lie algebras. We will give some basic definitions of
Lie algebras, ideals, simplicity, and semisimplicity, and discuss briefly the
classification of semisimple Lie algebras.  For more information about Lie
algebras, \cite{Hall2003} is an excellent reference for complex Lie algebras,
and \cite{Humphreys1973} for the more general case of Lie algebras over a
field of arbitrary characteristic. Here we will work over $\mathbb{C}$.


Throughout this section $\mathfrak{g}$ will denote a Lie algebra.

A \emph{Lie algebra} is a vector space $\mathfrak{g}$ with a bilinear operation
$\left[ \cdot, \cdot \right]$  $\mathfrak{g} \times \mathfrak{g} \to
\mathfrak{g}$ such that for any $x,y,z\in \mathfrak{g}$,

\begin{itemize}
    \item $\left[ x,y \right] = -\left[ y,x \right]$
    \item $\left[ x, \left[ y,z \right] \right] + \left[ y, \left[ z,x \right] \right] + \left[ z, \left[ x,y \right] \right] = 0$ (the Jacobi identity)
\end{itemize}

A basic example of a Lie algebra is $\sll(2)$: the algebra of $2 \times 2$
matrices with trace zero. $\sll(2)$ is generated by the matrices 

\begin{equation}
    X = \begin{pmatrix} 0 & 1 \\ 0 & 0 \end{pmatrix},
    Y = \begin{pmatrix} 0 & 0 \\ 1 & 0 \end{pmatrix}, 
    H = \begin{pmatrix} 1 & 0 \\ 0 &-1 \end{pmatrix}
\end{equation}

In general Ado's theorem says that any finite dimensional Lie algebra over a
field of characteristic zero can be realized as a vector space of matrices such
that 

\[ \left[ X,Y \right] = XY - YX\] 
for any $X,Y$, so the bracket $[X,Y]$ can safely be thought of
as the commutator $XY - YX$. 


We will only be interested here in semisimple Lie algebras, which are a
well-characterized class of Lie algebras. They are defined as follows.

\begin{defn}
    A subspace $\mathfrak{h} \subset \mathfrak{g}$ of a Lie algebra is a
    \emph{Lie subalgebra} if $\mathfrak{h}$ is closed under the Lie bracket. 

    An \emph{ideal} $I$ of $\mathfrak{g}$ is a subalgebra such that
    $[\mathfrak{g}, I] \subset I$
\end{defn}

\begin{defn}
    A Lie algebra is \emph{simple} if it has no proper ideals and is not
    commutative. 
\end{defn}

$\sll(2)$ is the lowest-dimensional simple Lie algebra. In fact, it is the
basic example of a simple Lie algebra: its representation theory is used in a
crucial way in the study of the representation theory of all semisimple Lie
algebras.

\begin{defn}
    A Lie algebra is \emph{semisimple} if it is the direct sum of simple Lie
    algebras.
\end{defn}

\subsection{Representations of $\sll(2)$}
We model anyons by representations of a deformation of $\sll(2)$, and
understanding the representations of $\sll(2)$ is an important first step. We
present the basic facts about the representation theory of $\sll(2)$ here
without proof.

A \emph{representation} of a Lie algebra $\mathfrak{g}$ is a vector space $V$
together with an action of $\mathfrak{g}$ on $V$ such that 

\begin{align*}
    \left[ x,y \right] v &= x(yv) - y(xv) \\
    (x+y)v &= xv + yv \\
    (ax)v &= a(xv)
\end{align*}

for all $x,y \in \mathfrak{g}, v \in V, a \in \mathbb{C}$

A representation is called \emph{irreducible} or \emph{simple} if it has no
nontrivial subspaces invariant under the action of $\mathfrak{g}$.

In $\sll(2)$,
\begin{itemize}
    \item Every representation decomposes as a direct sum of irreducible
        representations (in other words, is completely reduble)
    \item The eigenvectors of $H$ form a basis for every irreducible
        representation. Their eigenvalues are called \emph{weights} and and are
        all integers.
    \item The irreducible representations are classified by their highest
    weights $n \geq 0$: the representation $V(n)$ with highest weight $n$ has
    basis $\left\{ v_0, \ldots, v_n \right\}$ such that 
    % from Kassel, p. 101
\begin{align*}
    &H v_i = (n - 2i) v_i& \\
    &Y v_i = \begin{cases} 
                (i+1)v_{i+1}& \text{ for $i < m$} \\
                0& \text{ for $i = m$} \\
            \end{cases} \\
    &X v_i = \begin{cases} 
                (n-i+1)v_{i-1}& \text{ for $i > 0$} \\
                0& \text{ for $i = 0$} \\
            \end{cases}
\end{align*}
\end{itemize}




\subsection{Representations of Semisimple Lie Algebras}

Here we will describe some of the representation theory of semisimple Lie
algebras, in particular defining roots and weights and inner products of roots.
We will need this theory when discussing the representation theory of the
quantum groups in chapter \ref{chap:uqsl2}. 
In a general semisimple Lie algebra, the \emph{Cartan subalgebra}
$\mathfrak{h}$ will take the place of the matrix $H$: the basis vectors for
irreducible representations of $\frakg$ will be simultaneous eigenvectors of
every element of $\mathfrak{h}$ instead of eigenvectors of $H$.

\begin{defn}
    If $\frakg$ is a complex semisimple Lie algebra, then a \emph{Cartan
        subalgebra} of $\frakg$ is a complex subspace $\mathfrak{h}$ of
        $\frakg$ such that: 
    \begin{enumerate}
        \renewcommand{\labelenumi}{\roman{enumi})}
        \item $\left[ H_1, H_2 \right] = 0$ for all $H_1, H_2 \in \mathfrak{h}$
        \item For all $X \in \frakg$, if $\left[ H,X \right] = 0$, for all $H
            \in \mathfrak{h}$, then $X \in \mathfrak{h}$
        \item $\ad_H$ is diagonalizable for all $H \in \mathfrak{h}$
    \end{enumerate}
\end{defn}

\begin{defn}
A \emph{root} of $\frakg$ is a nonzero linear functional $\alpha$ on $\mathfrak{h}$ such that there exists a nonzero element $X \in \frakg$ with 

\begin{equation}
\left[ H,X \right] = \alpha(H)X
\end{equation}
for all $H \in \mathfrak{g}$
\end{defn}

\section{Hopf algebras}

Models of topological quantum computation come from certain algebras called
quantum groups. These quantum groups are somewhat deceptively named in that
they are not groups. However, they are Hopf algebras, which  are a
generalization of the group algebra of a group: they have a map called the
antipode which generalizes the group inverse. Basically a Hopf algebra is an
algebra which has a compatible coalgebra structure and an antipode map which
``behaves like'' an inverse in a suitable way. If an algebra is a Hopf algebra,
then its category of representations has additional structure, which will
be crucial for us in our discussion of topological quantum computing. This is
explained in \ref{section:RepHopfAlgebra}. 

In this section, we define algebras and coalgebras, then bialgebras which are
simultaneously algebras and coalgebras, then Hopf algebras which are bialgebras
with an antipode map.

\subsection{Algebras and Coalgebras}
We said earlier that for a field $k$, a \emph{$k$-algebra} $A$ is a $k$-vector
space with an
associative bilinear mapping $A \times A \to A$ which has an identity element
$1 \in A$ such that $1\cdot x = x\cdot 1 = x$ for any $x \in A$.

Restated in categorical terms, a $k$-algebra is given by a triple $(A, \mu, \eta)$,
where $A$ is a vector space, and the multiplication map $\mu: A \otimes A \to A$ and the unit map $\eta: k \to A$
are linear maps. For $A$ to be an algebra, the following two diagrams need to commute:

Associativity axiom:
\begin{equation}
\xymatrix{
A \otimes A \otimes A \ar[d]^-{\id \otimes \mu} \ar[r]^-{\mu \otimes \id} & A \otimes A \ar[d]^-{\mu}\\
 A \otimes A \ar[r]^-\mu & A 
}
\end{equation}

Unit axiom: 

\begin{equation}
    \xymatrix{
    k \otimes A \ar[r]^-{\eta \otimes \id} \ar[rd]_\simeq & A \otimes A \ar[d]^-\mu & A \otimes k \ar[l]_{\id \otimes \eta} \ar[ld]^-{\simeq} \\
    & A &
    }
\end{equation}

An algebra is called \emph{commutative} if $x \cdot y = y \cdot x$ for any
$x,y\in A$. In other terms, it needs to satisfy the commutativity axiom:

The triangle 

\begin{equation}
    \xymatrix{
    A \otimes A \ar[rr]^-{\tau_{A,A}} \ar[rd]_\mu & & A \otimes A \ar[ld]^-\mu \\
    & A &
    }
\end{equation}

Given two algebras $(A_1, \mu_1, \eta_1)$ and $(A_2, \mu_2, \eta_2)$, a linear
map $f: A_1 \to A_2$ is called a \emph{morphism of algebras} or an
\emph{algebra homomorphism} if $f(\mu_1(a,b)) = \mu_2(f(a), f(b))$ for any
$a,b\in A_1$ and $f(\eta_1(1)) = \eta_2(1)$. In other words, the following two
diagrams need to commute:

\begin{equation}
    \xymatrix{
    1 \ar[r]^-{\eta_1} \ar[rd]_{\eta_2} & A_1 \ar[d]^-f \\
    & A_2
    }
\end{equation}
\begin{equation}
    \xymatrix{
    A_1 \ar[r]^-f & A_2 \\
    A_1 \otimes A_1 \ar[u]^-{\mu_1} \ar[r]_-{f\otimes f} & A_2 \otimes A_2 \ar[u]^-{\mu_2}
    }
\end{equation}


We can obtain the definition of a coalgebra by reversing all the arrows in the
diagrams above as follows:

\begin{defn}
    A \emph{coalgebra} is a triple $(C, \Delta, \varepsilon)$ where $C$ is a
    vector space, and $\Delta: C \to C \otimes C$, $\varepsilon: C \to k$ are
    linear maps such that the following two diagrams commute:

Associativity axiom:
\begin{equation}
\xymatrix{
C \otimes C \otimes C   & C \otimes C \ar[l]_-{\Delta \otimes \id}\\
 C \otimes C\ar[u]^-{\id \otimes \Delta}  & C \ar[l]_-\Delta \ar[u]_-{\Delta}
}
\end{equation}

Counit axiom:


\begin{equation}
    \xymatrix{
    k \otimes C  & C \otimes C \ar[l]_-{\varepsilon \otimes \id} \ar[r]^-{\id \otimes \varepsilon} & C \otimes k   \\
    & C \ar[u]_-\Delta \ar[lu]^-\simeq \ar[ru]_-{\simeq}&
    }
\end{equation}
\end{defn}
A coalgebra is called \emph{cocommutative} if the following triangle commutes:

\begin{equation}
    \xymatrix{
    A \otimes A   & & A \ar[ll]^-{\tau_{A,A}} \otimes A  \\
    & \ar[lu]^-\Delta A \ar[ru]_-\Delta&
    }
\end{equation}


A linear map $f: C_1 \to C_2$ between two coalgebras $(C_1, \Delta_1,
\varepsilon_1)$, $(C_2, \Delta_2, \varepsilon_2)$ is a coalgebra homomorphism
if the following two diagrams commute: 


\begin{equation}
    \xymatrix{
    1 \ar[r]^-{\eta_1} \ar[rd]_{\eta_2} & A_1 \ar[d]^-f \\
    & A_2
    }
\end{equation}
\begin{equation}
    \xymatrix{
    A_1 \ar[r]^-f & A_2 \\
    A_1 \otimes A_1 \ar[u]^-{\mu_1} \ar[r]_-{f\otimes f} & A_2 \otimes A_2 \ar[u]^-{\mu_2}
    }
\end{equation}



\subsection{Bialgebras}

Suppose $H$ is a vector space which has both an algebra structure $(H, \mu,
\eta)$ and a coalgebra structure $(H, \Delta, \varepsilon)$. We call this a
\emph{bialgebra} if the two structures are compatible in the following sense:

\begin{defn}
    A vector space with an algebra and coalgebra structure is called a
    \emph{bialgebra} if one of the following two equivalent conditions holds:

    \begin{enumerate}
        \item The maps $\mu$ and $\eta$ are morphisms of coalgebras
        \item The maps $\Delta$ and $\varepsilon$ are morphisms of algebras
    \end{enumerate}
\end{defn}

A morphism of bialgebras is a map which is both a morphism of algebras and a
morphism of coalgebras.
\subsection{Hopf Algebras}
As we said earlier, a Hopf algebra is a bialgebra with an antipode map:

\begin{defn}
    Let $(H, \mu, \eta, \Delta, \varepsilon)$ be a bialgebra. An endomorphism
    $S$ of $H$ is called a \emph{antipode} for the bialgebra if the following
    two diagrams commute:

    \begin{equation}
        \xymatrix{
        H \ar[r]^-{\Delta} \ar[rrd]_-{\varepsilon}& H \otimes H \ar[rr]^-{S \otimes \id} & & H \otimes H \ar[r]^-{\mu} & H \\
        & & k \ar[rru]_-{\eta}& &
        }
    \end{equation}

    \begin{equation}
        \xymatrix{
        H \ar[r]^-{\Delta} \ar[rrd]_-{\varepsilon}& H \otimes H \ar[rr]^-{\id \otimes S} & & H \otimes H \ar[r]^-{\mu} & H \\
        & & k \ar[rru]_-{\eta}& &
        }
    \end{equation}
\end{defn}

\begin{defn}
    A \emph{Hopf algebra} is a bialgebra with an antipode. A morphism of Hopf
    algebras is a morphism between the bialgebras which commutes with the
    antipodes. 
\end{defn}

\begin{example}
    \label{example:groupalgebra}
    As mentioned earlier, the group algebra $k[G]$ of any finite group is a
    Hopf algebra. $k[G]$ has basis $\left\{ g: g \in G \right\}$ and
    multiplication given by the group multiplication. Define

\begin{align}
\Delta(g) &= g \otimes g \\
\varepsilon(g) &= 1 \\
S(g) &= g^{-1}
\end{align}

    It is easy but useful to check that this gives a Hopf algebra structure on $k[G]$.

    Note that the group algebra is not commutative if the underlying group is
    not commutative, but is always cocommutative. The examples of Hopf algebras
    we will be interested in later (quantum groups) are neither commutative nor
    cocommutative. 
\end{example}

In general the antipode map can be thought of as an analogue to the inverse map
in a group. If a bialgebra has an antipode, it is unique, and $S^2 = \id$.

This next example will be quite important to us: the quantum groups we will
study are deformations of universal enveloping algebras of semisimple Lie
algebras.

\begin{example}
    \label{example:UEA}
    Given a Lie algebra $\mathfrak{g}$, we can define an associative algebra
    called the \emph{universal enveloping algebra} of $\mathfrak{g}$. The
    universal enveloping algebra of any Lie algebra is a Hopf algebra.

    The \emph{tensor algebra} $T(V)$ of any $k$-vector space $V$ is an associative
    algebra defined by

    \begin{equation}
        T(V) = k \oplus \bigoplus_{n=1}^\infty \underbrace{(V \otimes \cdots \otimes V)}_{\text{$n$ times}}
    \end{equation}

    with multiplication $v \cdot w = v \otimes w$.

%    \todo{this doesn't feel well explained :(} 

    If we have a Lie algebra $\mathfrak{g}$, then we can define an ideal
    $I(\mathfrak{g})$ of the tensor algebra $T(\mathfrak{g})$ generated by all
    elements of the form $(xy - yx) - \left[ x,y \right]$ for $x,y \in
    \mathfrak{g}$.

    We define the universal enveloping algebra to be 

    \begin{equation}
        U(\mathfrak{g}) = T(\mathfrak{g}) / I(\mathfrak{g})
    \end{equation}

    We can put a Hopf algebra structure on the enveloping algebra
    $U(\mathfrak{g})$ for any Lie algebra $\mathfrak{g}$, with $\Delta,
    \varepsilon, S$ defined by: 
    \begin{align}
        \Delta(x) &= x \otimes 1 + 1 \otimes x \\
        \varepsilon(x) &= 0 \\
        S(x) &= -x  
    \end{align}

    for $x \in \mathfrak{g}$.

\end{example}

\subsection{Braided Hopf algebras and the $R$-matrix}
\label{subsection:R-matrix}

Both of the examples of Hopf algebras above are cocommutative. As mentioned
earlier, quantum groups are not cocommutative. However, bialgebras that are not
cocommutative can be \emph{braided}. This additional structure on a bialgebra
will allow us to make its category of representations braided.

Recall that a bialgebra is \emph{cocommutative} if for any $x$

\begin{equation}
    \tau_{H,H}\circ \Delta(x)  = \Delta(x)
\end{equation}

\begin{defn}
    A bialgebra $(H,\mu,\eta, \Delta, \varepsilon)$ is
    \emph{quasi-cocommutative} if there exists an invertible element $R \in H
    \otimes H$ such that for any $x$,

    \begin{equation}
        \tau_{H,H}\circ \Delta(x)  = R \Delta(x) R^{-1}
    \end{equation}
\end{defn}

Such an element $R$ is called a \emph{universal $R$-matrix} for the bialgebra. 

\begin{defn}
    A quasi-cocommutative bialgebra $(H,\mu,\eta, \Delta, \varepsilon, R)$ is
    \emph{braided} if $R$ satisfies the conditions

    \begin{align}
        (\Delta \otimes \id_H)(R) = R_{13} R_{23} \\
        (\id_H \otimes \Delta)(R) = R_{13} R_{12} 
    \end{align}

    where $R_{ij} \in H \otimes H \otimes H$, $R_{ij} = \sum_{i} y_i^{(1)} \otimes y_{i}^{(2)} \otimes y_i^{(2)}$

    \begin{equation}
        y_i^{(k)} = \begin{cases} s_i & \text{ if k = i} \\
                                  t_i & \text{ if $k = j$} \\
                                  1   & \text{ else}
                    \end{cases}
    \end{equation}
    For example $R_{13} = \sum_{i} s_i \otimes 1 \otimes t_i$.

\end{defn}

It can be shown that $R$ satisfies the Yang-Baxter equation

\begin{equation}
    R_{12} R_{13} R_{23} = R_{23} R_{13} R_{12}
\end{equation}

We will see in \ref{bialgtocategory} how this leads to a braiding on the
category of representations.
