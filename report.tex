\documentclass[]{article}
\usepackage{amsmath}
\usepackage{amsthm}
\usepackage{amsfonts}
\usepackage{xypic}

\newtheorem{theorem}{Theorem}[subsection]
\newtheorem{remark}[theorem]{Remark}
\newtheorem{defn}[theorem]{Definition}
\newtheorem{lemma}[theorem]{Lemma}
\newtheorem{example}[theorem]{Example}

\newcommand{\isomto}{\overset{\sim}{\rightarrow}}
\newcommand{\tr}{\operatorname{tr}}
\newcommand{\id}{\operatorname{id}}
\newcommand{\Hom}{\operatorname{Hom}}
\newcommand{\End}{\operatorname{End}}

\numberwithin{equation}{subsection}

\begin{document}
\tableofcontents
\section{Algebraic Background}
\subsection{Modules}
\subsection{Hopf algebras}
\subsubsection{Algebras and Coalgebras}
For a field $k$, a \emph{$k$-algebra} $A$ is a $k$-vector space with an
associative bilinear mapping $A \times A \to A$ which has an identity element
$1 \in A$ such that $1\cdot x = x\cdot 1 = x$ for any $x \in A$.

Put in categorical terms, a $k$-algebra is given by a triple $(A, \mu, \eta)$,
where $A$ is a vector space, and $\mu: A \otimes A \to A$ and $\eta: k \to A$
are linear maps satisfying the axioms:

Associativity:
The diagram
\begin{equation}
\xymatrix{
A \otimes A \otimes A \ar[d]^{\id \otimes \mu} \ar[r]^{\mu \otimes \id} & A \otimes A \ar[d]^{\mu}\\
 A \otimes A \ar[r]^\mu & A 
}
\end{equation}
commutes.

Unit: 

The diagram

\begin{equation}
    \xymatrix{
    k \otimes A \ar[r]^{\eta \otimes \id} \ar[rd]^\simeq & A \otimes A \ar[d]^\mu & A \otimes k \ar[l]_{\id \otimes \eta} \ar[ld]^{\simeq} \\
    & A &
    }
\end{equation}
commutes.

An algebra is called \emph{commutative} if $x \cdot y = y \cdot x$ for any $x,y\in A$. In other terms, it needs to satisfy the commutativity axiom:

The triangle 

\begin{equation}
    \xymatrix{
    A \otimes A \ar[rr]^{\tau_{A,A}} \ar[rd]_\mu & & A \otimes A \ar[ld]^\mu \\
    & A &
    }
\end{equation}


We can obtain the definition of a coalgebra by reversing all the arrows as follows:

\begin{defn}
    A \emph{coalgebra} is a triple $(C, \Delta, \varepsilon)$ where $C$ is a
    vector space, and $\Delta: C \to C \otimes C$, $\varepsilon: C \to k$ are
    linear maps satisfying the axioms:

Associativity:
The diagram
\begin{equation}
\xymatrix{
C \otimes C \otimes C   & C \otimes C \ar[l]_{\Delta \otimes \id}\\
 C \otimes C\ar[u]^{\id \otimes \Delta}  & C \ar[l]_\Delta \ar[u]_{\Delta}
}
\end{equation}
commutes.

Unit: 

The diagram

\begin{equation}
    \xymatrix{
    k \otimes C  & C \otimes C \ar[l]_{\varepsilon \otimes \id} \ar[r]^{\id \otimes \varepsilon} & C \otimes k   \\
    & C \ar[u]_\Delta \ar[lu]^\simeq \ar[ru]_{\simeq}&
    }
\end{equation}
commutes.
\end{defn}
A coalgebra is called \emph{cocommutative} if the triangle 

\begin{equation}
    \xymatrix{
    A \otimes A   & & A \ar[ll]^{\tau_{A,A}} \otimes A  \\
    & \ar[lu]^\Delta A \ar[ru]_\Delta&
    }
\end{equation}

commutes.

%todo: define a morphism of bialgebras

\subsubsection{Bialgebras}

Suppose $H$ is a vector space which has both an algebra structure $(H, \mu,
\eta)$ and a coalgebra structure $(H, \Delta, \varepsilon)$. We call this a
\emph{bialgebra} if the two structures are compatible:

\begin{defn}
    A vector space with an algebra and coalgebra structure is called a
    \emph{bialgebra} if one of the following two equivalent conditions holds:

    \begin{enumerate}
        \item The maps $\mu$ and $\eta$ are morphisms of coalgebras
        \item The maps $\Delta$ and $\varepsilon$ are morphisms of algebras
    \end{enumerate}
\end{defn}

\subsubsection{Hopf Algebras}

% todo: introduce Sweedler notation??

\begin{defn}
    Let $(H, \mu, \eta, \Delta, \varepsilon)$ be a bialgebra. An endomophisms
    $S$ of $H$ is called a \emph{antipode} for the bialgebra if the following
    two diagrams commute:

    \begin{equation}
        \xymatrix{
        H \ar[r]^{\Delta} \ar[rrd]_{\varepsilon}& H \otimes H \ar[rr]^{S \otimes \id} & & H \otimes H \ar[r]^{\mu} & H \\
        & & k \ar[rru]_{\eta}& &
        }
    \end{equation}

    \begin{equation}
        \xymatrix{
        H \ar[r]^{\Delta} \ar[rrd]_{\varepsilon}& H \otimes H \ar[rr]^{\id \otimes S} & & H \otimes H \ar[r]^{\mu} & H \\
        & & k \ar[rru]_{\eta}& &
        }
    \end{equation}
\end{defn}

\begin{defn}
    A \emph{Hopf algebra} is a bialgebra with an antipode. A morphism of Hopf
    algebras is a morphism between the bialgebras which commutes with the
    antipodes. 
\end{defn}

\begin{example}
    One motivating example of a Hopf algebra is the Hopf algebra obtained from
    any finite group. 

    Let $G$ be a finite group. The Hopf algebra $k[G]$ is the algebra with
    basis $\left\{ v_g: g \in G \right\}$, multiplication $v_g v_h = v_{gh}$,
    and unit the group identity. 

    We can define a coalgebra structure on $k[G]$ by $\Delta(v_g) = v_g \otimes v_g$
    and $\varepsilon(v_g) = 0$ for $g \neq e$, $\varepsilon(v_e) = 1$.

    The antipode $S$ is given by $S(v_g) = v_{g^{-1}}$ for any $g \in G$.
\end{example}

In general the antipode can be thought of as an analog to the inverse in a
group. If a Hopf algebra has an antipode, it is unique, and $S^2 = \id$.
\subsection{Representations}

\section{Categorical Background}
\subsection{Monoidal Categories}
% todo: talk about abelian categories, simple objects, the endomorphism rings
% being vector spaces
\begin{defn}
    A \emph{monoidal category} is a category $\mathcal{C}$ with 
    \begin{enumerate}
            %todo: fix the enumerate labels
    \renewcommand{\labelenumi}{(\roman{enumi})}
        \item a bifunctor $\otimes: \mathcal{C} \times \mathcal{C} \to
            \mathcal{C}$
        \item a unit object $\mathbf{1}$ and natural transformations
            \begin{equation}
                \lambda_V : \mathbf{1} \otimes V \isomto V \\
                \rho_V : V \otimes 1 \isomto V
            \end{equation}
        \item a natural transformation 
            \begin{equation}
                \alpha_{UVW} (U \otimes V) \otimes W \isomto U \otimes (V \otimes W)
            \end{equation}

            which satisfy the associativity property 

        \item if $X_1, X_2$ are two objects obtained from $V_1 \otimes V_2 \otimes \cdots V_n$ by inserting 1s and brackets, then all isomorphisms $\varphi: X_1 \isomto X_2$ composed of $\alpha$'s, $\lambda$'s, and $\rho$'s are equal. 
        \item $\mathbf{1}$ is a simple object and $\End_\mathcal{C} \mathbf{1} = k$
    \end{enumerate}

\end{defn}

\begin{example}
    \begin{enumerate}
    \renewcommand{\labelenumi}{\roman{enumi}}
        \item The category of vector spaces over a field $\operatorname{Vec}(k)$
        \item the category of finite dimensional representations of a group, algebra, or Lie algebra
    \end{enumerate}
\end{example}
    
\subsection{Braided monoidal categories}

Let $\mathcal{C}$ be a monoidal category with a natural transformation 

\begin{equation}
    \sigma_{VW} : V \otimes W \to W \otimes V
\end{equation}

% todo: copy this stuff from B & K. blah.
\subsection{Rigid monoidal categories}
% todo: write down the vector space example
A rigid monoidal category is a category where there is a notion of a dual. 

\begin{defn}
    Let $\mathcal{C}$ be a monoidal category, $V$ an object in $\mathcal{C}$. A \emph{right dual} to $V$ is an object $V^*$ with two morphisms

    \begin{equation}
        e_V: V^* \otimes V \to \mathbf{1} \\
        i_V: \mathbf{1} \to V^* \otimes V
    \end{equation}
\end{defn}

such that the composition

\begin{equation}
    V \stackrel{i_v \otimes \id_V}{\xrightarrow{\hspace*{1cm}}} V \otimes V^*
    \otimes V  \stackrel{\id_V \otimes e_V}{\xrightarrow{\hspace*{1cm}}} V
\end{equation}

is equal to $\id_V$, and similarly the composition

\begin{equation}
    V^* \stackrel{id_{V^*} \otimes \id_V}{\xrightarrow{\hspace*{1cm}}} V^*
    \otimes V \otimes V^*  \stackrel{e_V \otimes
    \id_{V^*}}{\xrightarrow{\hspace*{1cm}}} V^*
\end{equation}

is equal to $\id_{V^*}$

\subsection{The Category of Representation of a Hopf Algebra}

Suppose $(H, \mu, \eta, \Delta, \varepsilon)$ is a Hopf algebra with antipode $S$. 
\section{The Representation Theory of $U_q(\mathfrak{sl}(2))$}
\subsection{Lie Algebras}

The examples of quantum groups that we are interested in are obtained as deformations of simple Lie algebras. 

A \emph{Lie algebra} is a vector space with a bilinear operation $\left[ \cdot, \cdot \right]$ such that 

\begin{itemize}
    \item $\left[ x,y \right] = -\left[ y,x \right]$
    \item $\left[ x, \left[ y,z \right] \right] + \left[ y, \left[ z,x \right] \right] + \left[ z, \left[ x,y \right] \right] = 0$ (the Jacobi identity)
\end{itemize}

% find a reference for the following
Any Lie algebra can be realized as a vector space of matrices such that 

\[ \left[ X,Y \right] = XY - YX\] 
for any $X,Y$

A basic example of a Lie algebra is $\mathfrak{sl}(2)$: the algebra of $2 \times 2$ matrices with trace zero. $\mathfrak{sl}(2)$ is generated by the matrices 
\begin{equation}
    X = \begin{pmatrix} 0 & 1 \\ 0 & 0 \end{pmatrix},
    Y = \begin{pmatrix} 0 & 0 \\ 1 & 0 \end{pmatrix}, 
    H = \begin{pmatrix} 1 & 0 \\ 0 &-1 \end{pmatrix}
\end{equation}

It is the lowest-dimensional simple nontrivial Lie algebra. 
%% define a simple Lie algebra

\subsection{Representations of $\mathfrak{sl}(2)$}

A \emph{representation} of a Lie algebra $\mathfrak{g}$ is a vector space $V$ together with an action of $\mathfrak{g}$ on $V$ such that 

\begin{align*}
    \left[ x,y \right] v &= x(yv) - y(xv) \\
    (x+y)v &= xv + yv \\
    (ax)v &= a(xv)
\end{align*}

for all $x,y \in \mathfrak{g}, v \in V, a \in \mathbb{C}$

The representations of $\mathfrak{sl}(2)$ can be classified as follows: 

For each integer $n \geq 0$, there is a unique (up to isomorphism) representation of $\mathfrak{sl}(2)$ of dimension $n+1$. This representation has basis $v_0, \ldots, v_{n}$ such that
% from Kassel, p. 101
\begin{align*}
    &H v_i = (n - 2i) v_i& \\
    &Y v_i = \begin{cases} 
                (i+1)v_{i+1}& \text{ for $i < m$} \\
                0& \text{ for $i = m$} \\
            \end{cases} \\
    &X v_i = \begin{cases} 
                (n-i+1)v_{i-1}& \text{ for $i > 0$} \\
                0& \text{ for $i = 0$} \\
            \end{cases}
\end{align*}

\subsection{The quantized universal enveloping algebra $U_q(\mathfrak{sl}(2))$}
% Kassel, p. 121
We first introduce some notation. For any integer $n$, let 

\begin{equation}
    [n]_q = \frac{q^n - q^{-n}}{q - q^{-1}} = q^{n-1} + q^{n-3} + \cdots + q^{-n+3} + q^{-n+1}
\end{equation}

\begin{remark}
$q^{2d}=1$ if and only if $[d]_q = 0$, so $[n]_q \neq 0$ for every nonzero integer when $q$ is not a root of unity. 
\end{remark}

%% Jantzen, p. 9
Let $q$ be a complex number such that $q \neq 0$, $q^2 \neq 1$.  Then $U_q(\mathfrak{sl}(2))$ is the associative algebra with generators $E,F,K, K^{-1}$ and relations 

\begin{align}
    KK^{-1} &= 1 = K^{-1}K \\
    KEK^{-1} &= q^2 E \\
    KFK^{-1} &= q^{-2} F \\
    [E,F] &= \frac{K - K^{-1}}{q - q^{-1}}
\end{align}

%% this presentation taken verbatim from Kassel, p. 125

$U_q(\mathfrak{sl}(2))$ is isomorphic to the algebra $U'_q(\mathfrak{sl}(2))$ with generators $E,F,K,K^{-1},L$ and relations:
%todo: fix the numbering here
\begin{equation}
    \begin{gathered}
    KK^{-1} = 1 = K^{-1}K \\
    KEK^{-1} = q^2 E \\
    KFK^{-1} = q^{-2} F \\
    [E,F] = L, (q - q^{-1})L = K-K^{-1} \\
    [L,E] = q(EK + K^{-1}E), [L,F] = -q^{-1}(FK + K^{-1}F) \\
    \end{gathered}
\end{equation}

% todo: do I need to explain quotients by ideals? Do I need to define algebras? Help.
% todo: what proofs should I put for things? Ask prakash.


For $q=1$, we can define a homomorphism $\varphi: U'_1(\mathfrak{sl}(2)) \to
U(\mathfrak{sl}(2))$ by sending $E$ to $X$, $F$ to $Y$, and $L$ to $H$. 
\begin{lemma}
This homomorphism is surjective and has kernel generated by $(K-1)$, so
$U'_1(\mathfrak{sl}(2)) \stackrel{\sim}{=} U(\mathfrak{sl}(2))$
\end{lemma}
%todo: should I explain why the representation theory looks this way? 
%todo: include this picture from Kassel, p. 129 along with a similar picture for sl(2)
% from Kassel, p. 123
There is a unique algebra automorphism of $U_q(\mathfrak{sl}(2))$ such that 

\begin{equation}
    \omega(E) = F, \omega(F) = E, \omega(K) = K^{-1}
\end{equation}

This is called the \emph{Cartan automorphism}.

\subsection{Representations of $U_q(\mathfrak{sl}(2))$, $q$ a root of unity}
% Jantzen, 2.11 - 2.13. June 2010 in Clairefontaine book

%% Explain this explicitly
The representation theory of the associative algebra $U(\mathfrak{sl}(2))$ can
be described in exactly the same way as that of the Lie algebra
$\mathfrak{sl(2)}$. 

If $q$ is not a root of unity, then the representations of $U_q(\mathfrak{sl}(2))$ can be classified as follows: 

For each $n\geq 0$, there are 2 simple $U_q(\mathfrak{sl}(2))$-modules of dimension $n+1$: 
% todo: fix formatting here
$L_+(n)$ with basis $v_0, \ldots, v_n$ \\
$L_-(n)$ with basis $v'_0, \ldots, v'_n$

such that: 
% todo: find a better way to motivate q-notation. (maybe that paper that
% prakash showed you?)
\begin{align*}
    &K v_i = q^{n-2i} v_i  &
    &K v'_i = -q^{n-2i} v'_i \\
    &F v_i = \begin{cases} [i+1]_qv_{i+1}& \text{ if $i < n$} \\ 0& \text{ if $i = n$} \end{cases} &
    &F v'_i = \begin{cases} [i+1]_qv'_{i+1}& \text{ if $i < n$} \\ 0& \text{ if $i = n$} \end{cases} \\
    &E v_i = \begin{cases} 
                     [n-i+1]_qv_{i-1}& \text{ if $i > 0$} \\ 
                    0& \text{ if $i = 0$} 
             \end{cases} &
    &E v'_i = \begin{cases} 
                    -[n-i+1]_q v'_{i-1}& \text{ if $i > 0$} \\ 
                    0& \text{ if $i = 0$} 
             \end{cases}
\end{align*}

% todo: why are there 2 representations here instead of 1? what extra freedom
% allows that to happen?

Suppose $\ell$ is odd, and $q$ is an $\ell^{\text{th}}$ root of unity. 

If $0 \leq n \leq \ell - 1$, the $(n+1)$-dimensional
$U_q(\mathfrak{sl}(2))$-modules are as described above. %todo: name them 
There are no simple $U_q(\mathfrak{sl}(2))$-modules with dimension $\geq
\ell+1$. It remains to classify the $\ell$-dimensional
$U_q(\mathfrak{sl}(2))$-modules.

There are $3$ infinite classes of simple $\ell$-dimensional representations of  $U_q(\mathfrak{sl}(2))$:
\begin{enumerate}
        % todo: say somewhere that the results and notation are taken from
        % Jantzen
        \item For any $b,\lambda \in \mathbf{C}, \lambda \neq 0$, define
            $Z_b(\lambda)$ to be the module with basis $v_0, \ldots, v_{\ell -
            1}$ with action of $U$ given by: 
\begin{align*}
    &K v_i = q^{-2i} \lambda v_i \\
    &F v_i = \begin{cases} v_{i+1}& \text{ if $i < n$} \\  b v_0& \text{ if $i = n$} \end{cases}  \\
    &E v_i = \begin{cases} 
        [i]_q \frac{\lambda q^{1-i} - \lambda^{-1} q^{i-1}}{q - q^{-1}} v_{i-1}& \text{ if $i > 0$} \\ 
                    0& \text{ if $i = 0$} 
             \end{cases} 
\end{align*}

If $b \neq 0$ or $\lambda^{2\ell}\neq 1$ then $Z_b(\lambda)$ is simple.
$Z_0(\pm q^k)$ is simple if and only if $k = \ell - 1$. 

\begin{remark}
    $U_q(\mathfrak{sl}(2))$ is not semisimple: the modules $Z_0(\pm q_k)$ are not semisimple for $0 \leq k < \ell - 1$.
\end{remark}
% todo: include some representation theory background. Give a more coherent
% definition of a module. define semisimple. define simple (equiv. irreducible)
% from Jantzen
\item For any $U_q(\mathfrak{sl}(2))$-module $N$, define $^\omega N$ to be
    equal to $N$ as a vector space, and where $u$ acts on $^\omega N$ as
    $\omega(u)$ acts on $N$.

    The $^\omega Z_b(\lambda)$ are another class of modules.
\item Let $a,b,\lambda \in \mathbf{C}$, $a,b\neq 0$
    % todo: say something about them not all being the same.
\begin{align*}
    &K v_i = q^{-2i} \lambda v_i \\
    &F v_i = \begin{cases} 
                v_{i+1}& \text{ if $i < n$} \\  
                b v_0& \text{ if $i = n$} 
             \end{cases}  \\
    &E v_i = \begin{cases} 
                \left(ab + [i]_q \frac{(\lambda q^{1-i} - \lambda^{-1} q^{i-1})}{q - q^{-1}}\right) v_{i-1}& \text{ if $i > 0$} \\ 
                a v_{\ell - 1}& \text{ if $i = 0$} 
             \end{cases} 
\end{align*}
\end{enumerate}
\subsection{A Hopf Algebra Structure on $U_q(\mathfrak{sl}(2))$}

We can define a Hopf algebra stucture on $U_q(\mathfrak{sl}(2))$ as follows:

%todo: fix formatting
% from Kassel, p. 140
\begin{align}
    &\Delta(E) = 1 \otimes E + E \otimes K, \Delta(F) = K^{-1} \otimes F + F \otimes 1 \\
    &\Delta(K) = K \otimes K, \Delta(K^{-1}) = K^{-1} \otimes K^{-1}\\ 
    &\varepsilon(E) = \varepsilon(F) = 0, \varepsilon(K) = \varepsilon(K^{-1}) = 1
\end{align}

and antipode defined by 

\begin{align}
    & S(E) = -EK^{-1}, S(F) = -KF \\
    & S(K) = K^{-1}, S(K^{-1}) = K 
\end{align}
% todo: talk about how you get a tensor, etc. structure on the category from a
% Hopf algebra structure.
It is straightforward to check that this defines a Hopf algebra structure. 

\subsection{The quotient category}
% from Bakalov & Kirillov, section 3.3
We will denote the category of representations of $U_q(\mathfrak{sl}(2))$ at a
$k^\text{th}$ root of unity $q$ by $\mathcal{C}(\mathfrak{sl}(2), k)$. As
discussed previously, this category is fairly complicated, and in particular
not semisimple. Our goal is to find an appropriate semisimple part of this
category.

The first step is to restrict our attention to the subcategory of \emph{tilting modules}. 

% todo: define maximal submodule?
\begin{defn}
    A \emph{composition series} for a module $M$ is a sequence of submodules 

    \begin{equation}
        \left\{ 0 \right\} = J_0 \subset \cdots \subset J_n = M
    \end{equation}

    such each $J_k$ is a maximal submodule of $J_{k+1}$
\end{defn}
\begin{defn}
    A $U_q(\mathfrak{sl}(2))$-module $T$ is called \emph{tilting} if both $T$ and $T^*$ have composition series with factors $V_n$.
    % TODO: define the V_n
\end{defn}

To make this subcategory into a modular tensor categories, we will to quotient
this category by the modules with quantum dimension zero. 

\begin{defn}
    A tilting module $T$ is \emph{negligible} if $\tr_q(f) = 0$ for all $f: T \to T$.
\end{defn}
\begin{lemma}
    %% todo: make this less general
    $T$ is negligible if and only if $T = \bigoplus _{\lambda \notin \mathcal{C}} n_\lambda T_\lambda$
\end{lemma}

\begin{defn}
    A morphism $f: T_1 \to T_2$ is \emph{negligible} if $\tr_q(fg) = 0$ for any $g: T_2 \to T_1$
\end{defn}

\begin{defn}
    Define the category $\mathcal{C}^\text{int}$ be the category with objects tilting modules and morphisms 

    \begin{equation*}
        \Hom(V,W) = \Hom_T(V,W) / \text{negligible morphisms}
    \end{equation*}
\end{defn}

The category $\mathcal{C}^\text{int}$ has the following properties:
\begin{enumerate}
    \item An object $T$ is negligible if and only if it is isomorphic to 0 in $\mathcal{C}^\text{int}$
    \item  $\mathcal{C}^\text{int}$ is a ribbon category
    \item Any object $T$ in $\mathcal{C}^\text{int}$ is isomorphic to % todo: fix
    \item $\mathcal{C}^\text{int}$ is a semisimple abelian category. 
    \item $\dim_{\mathcal{C}^\text{int}} T > 0$ for every $T \not\simeq 0$
\end{enumerate}

$\mathcal{C}^\text{int}$ is in fact a modular tensor category. A proof of this
can be found in Bakalov \& Kirillov. %todo: make this a real reference
 

% Bakalov & Kirillov
\end{document}
