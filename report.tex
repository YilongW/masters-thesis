\documentclass[]{article}
\usepackage{amsmath}
\usepackage{amsthm}
\usepackage{amsfonts}
\newtheorem{remark}{Remark}
\begin{document}
\tableofcontents
\section{The Representation Theory of $U_q(\mathfrak{sl}(2))$}
\subsection{Lie Algebras}

The examples of quantum groups that we are interested in are obtained as deformations of simple Lie algebras. 

A \emph{Lie algebra} is a vector space with a bilinear operation $\left[ \cdot, \cdot \right]$ such that 

\begin{itemize}
    \item $\left[ x,y \right] = -\left[ y,x \right]$
    \item $\left[ x, \left[ y,z \right] \right] + \left[ y, \left[ z,x \right] \right] + \left[ z, \left[ x,y \right] \right] = 0$ (the Jacobi identity)
\end{itemize}

% find a reference for the following
Any Lie algebra can be realized as a vector space of matrices such that 

\[ \left[ X,Y \right] = XY - YX\] 
for any $X,Y$

A basic example of a Lie algebra is $\mathfrak{sl}(2)$: the algebra of $2 \times 2$ matrices with trace zero. $\mathfrak{sl}(2)$ is generated by the matrices 
\begin{equation}
    X = \begin{pmatrix} 0 & 1 \\ 0 & 0 \end{pmatrix},
    Y = \begin{pmatrix} 0 & 0 \\ 1 & 0 \end{pmatrix}, 
    H = \begin{pmatrix} 1 & 0 \\ 0 &-1 \end{pmatrix}
\end{equation}

It is the lowest-dimensional simple nontrivial Lie algebra. 
%% define a simple Lie algebra

\subsection{Representations of $\mathfrak{sl}(2)$}

A \emph{representation} of a Lie algebra $\mathfrak{g}$ is a vector space $V$ together with an action of $\mathfrak{g}$ on $V$ such that 

\begin{align*}
    \left[ x,y \right] v &= x(yv) - y(xv) \\
    (x+y)v &= xv + yv \\
    (ax)v &= a(xv)
\end{align*}

for all $x,y \in \mathfrak{g}, v \in V, a \in \mathbb{C}$

The representations of $\mathfrak{sl}(2)$ can be classified as follows: 

For each integer $n \geq 0$, there is a unique (up to isomorphism) representation of $\mathfrak{sl}(2)$ of dimension $n+1$. This representation has basis $v_0, \ldots, v_{n}$ such that
% from Kassel, p. 101
\begin{align*}
    &H v_i = (n - 2i) v_i& \\
    &Y v_i = \begin{cases} 
                (i+1)v_{i+1}& \text{ for $i < m$} \\
                0& \text{ for $i = m$} \\
            \end{cases} \\
    &X v_i = \begin{cases} 
                (n-i+1)v_{i-1}& \text{ for $i > 0$} \\
                0& \text{ for $i = 0$} \\
            \end{cases}
\end{align*}

\subsection{The quantized universal enveloping algebra $U_q(\mathfrak{sl}(2))$}
% Kassel, p. 121
We first introduce some notation. For any integer $n$, let 

\begin{equation}
    [n] = \frac{q^n - q^{-n}}{q - q^{-1}} = q^{n-1} + q^{n-3} + \cdots + q^{-n+3} + q^{-n+1}
\end{equation}

\begin{remark}
$q^{2d}=1$ if and only if $[d]_q = 0$, so $[n] \neq 0$ for every nonzero integer when $q$ is not a root of unity. 
\end{remark}

%% Jantzen, p. 9
Let $q$ be a complex number such that $q \neq 0$, $q^2 \neq 1$.  Then $U_q(\mathfrak{sl}(2))$ is the associative algebra with generators $E,F,K, K^{-1}$ and relations 

\begin{align}
    KK^{-1} &= 1 = K^{-1}K \\
    KEK^{-1} &= q^2 E \\
    KFK^{-1} &= q^{-2} F \\
    [E,F] &= \frac{K - K^{-1}}{q - q^{-1}}
\end{align}

%% this presentation taken verbatim from Kassel, p. 125

$U_q(\mathfrak{sl}(2))$ is isomorphic to the algebra $U'_q(\mathfrak{sl}(2))$ with generators $E,F,K,K^{-1},L$ and relations:
%todo: fix the numbering here
\begin{equation}
    \begin{gathered}
    KK^{-1} = 1 = K^{-1}K \\
    KEK^{-1} = q^2 E \\
    KFK^{-1} = q^{-2} F \\
    [E,F] = L, (q - q^{-1})L = K-K^{-1} \\
    [L,E] = q(EK + K^{-1}E), [L,F] = -q^{-1}(FK + K^{-1}F) \\
    \end{gathered}
\end{equation}

% todo: do I need to explain quotients by ideals? Do I need to define algebras? Help.
% todo: what proofs should I put for things? Ask prakash.


For $q=1$, we can define a homomorphism $\varphi: U'_1(\mathfrak{sl}(2)) \to
U(\mathfrak{sl}(2))$ by sending $E$ to $X$, $F$ to $Y$, and $L$ to $H$. This
homomorphism is surjective and has kernel generated by $(K-1)$, so
$U'_1(\mathfrak{sl}(2)) \stackrel{\sim}{=} U(\mathfrak{sl}(2))$
%todo: put this in a lemma. Help?
%todo: should I explain why the representation theory looks this way? 
%todo: include this picture from Kassel, p. 129 along with a similar picture for sl(2)
% from Kassel, p. 123
There is a unique algebra automorphism of $U_q(\mathfrak{sl}(2))$ such that 

\begin{equation}
    \omega(E) = F, \omega(F) = E, \omega(K) = K^{-1}
\end{equation}

This is called the \emph{Cartan automorphism}.

\subsection{Representations of $U_q(\mathfrak{sl}(2))$, $q$ a root of unity}
% Jantzen, 2.11 - 2.13. June 2010 in Clairefontaine book

%% Explain this explicitly
The representation theory of the associative algebra $U(\mathfrak{sl}(2))$ can
be described in exactly the same way as that of the Lie algebra
$\mathfrak{sl(2)}$. 

If $q$ is not a root of unity, then the representations of $U_q(\mathfrak{sl}(2))$ can be classified as follows: 

For each $n\geq 0$, there are 2 simple $U_q(\mathfrak{sl}(2))$-modules of dimension $n+1$: 
% todo: fix formatting here
$L_+(n)$ with basis $v_0, \ldots, v_n$ \\
$L_-(n)$ with basis $v'_0, \ldots, v'_n$

such that: 
% todo: introduce q-notation earlier on. find a good fucking way to motivate
% it. (maybe that paper that prakash showed you?)
\begin{align*}
    &K v_i = q^{n-2i} v_i  &
    &K v'_i = -q^{n-2i} v'_i \\
    &F v_i = \begin{cases} [i+1]_qv_{i+1}& \text{ if $i < n$} \\ 0& \text{ if $i = n$} \end{cases} &
    &F v'_i = \begin{cases} [i+1]_qv'_{i+1}& \text{ if $i < n$} \\ 0& \text{ if $i = n$} \end{cases} \\
    &E v_i = \begin{cases} 
                     [n-i+1]_qv_{i-1}& \text{ if $i > 0$} \\ 
                    0& \text{ if $i = 0$} 
             \end{cases} &
    &E v'_i = \begin{cases} 
                    -[n-i+1]_q v'_{i-1}& \text{ if $i > 0$} \\ 
                    0& \text{ if $i = 0$} 
             \end{cases}
\end{align*}

% todo: why are there 2 representations here instead of 1? what extra freedom
% allows that to happen?

Suppose $\ell$ is odd, and $q$ is an $\ell^{\text{th}}$ root of unity. 

If $0 \leq n \leq \ell - 1$, the $n+1$-dimensional
$U_q(\mathfrak{sl}(2))$-modules are as described above. %todo: name them 
There are no simple $U_q(\mathfrak{sl}(2))$-modules with dimension $\geq
\ell+1$. It remains to classify the $\ell$-dimensional
$U_q(\mathfrak{sl}(2))$-modules.

There are $3$ infinite classes of simple $\ell$-dimensional representations of  $U_q(\mathfrak{sl}(2))$:
% todo: is it really appropriate to use an enumerate environment here? Maybe.
\begin{enumerate}
        % todo: say somewhere that the results and notation are taken from Jantzen
        \item For any $b,\lambda \in \mathbf{C}, \lambda \neq 0$, define
            $Z_b(\lambda)$ to be the module with basis $v_0, \ldots, v_{\ell -
            1}$ with action of $U$ given by: 
\begin{align*}
    &K v_i = q^{-2i} \lambda v_i \\
    &F v_i = \begin{cases} v_{i+1}& \text{ if $i < n$} \\  b v_0& \text{ if $i = n$} \end{cases}  \\
    &E v_i = \begin{cases} 
        [i]_q \frac{\lambda q^{1-i} - \lambda^{-1} q^{i-1}}{q - q^{-1}} v_{i-1}& \text{ if $i > 0$} \\ 
                    0& \text{ if $i = 0$} 
             \end{cases} 
\end{align*}

If $b \neq 0$ or $\lambda^{2\ell}\neq 1$ then $Z_b(\lambda)$ is simple.
$Z_0(\pm q^k)$ is simple if and only if $k = \ell - 1$. 

\begin{remark}
    $U_q(\mathfrak{sl}(2))$ is not semisimple: the modules $Z_0(\pm q_k)$ are not semisimple for $0 \leq k < \ell - 1$.
\end{remark}
% todo: include some representation theory background. Give a more coherent
% definition of a module. define semisimple. define simple (equiv. irreducible)
% from Jantzen
\item For any $U_q(\mathfrak{sl}(2))$-module $N$, define $^\omega N$ to be
    equal to $N$ as a vector space, and where $u$ acts on $^\omega N$ as
    $\omega(u)$ acts on $N$.

    The $^\omega Z_b(\lambda)$ are another class of modules.
\item Let $a,b,\lambda \in \mathbf{C}$, $a,b\neq 0$
    % todo: say something about them not all being the same.
\begin{align*}
    &K v_i = q^{-2i} \lambda v_i \\
    &F v_i = \begin{cases} 
                v_{i+1}& \text{ if $i < n$} \\  
                b v_0& \text{ if $i = n$} 
             \end{cases}  \\
    &E v_i = \begin{cases} 
                \left(ab + [i]_q \frac{(\lambda q^{1-i} - \lambda^{-1} q^{i-1})}{q - q^{-1}}\right) v_{i-1}& \text{ if $i > 0$} \\ 
                a v_{\ell - 1}& \text{ if $i = 0$} 
             \end{cases} 
\end{align*}
\end{enumerate}
\subsection{A Hopf Algebra Structure on $U_q(\mathfrak{sl}(2))$}
%todo: define a Hopf algebra

We can define a Hopf algebra stucture on $U_q(\mathfrak{sl}(2))$ as follows:

%todo: fix formatting
% from Kassel, p. 140
\begin{align}
    &\Delta(E) = 1 \otimes E + E \otimes K, \Delta(F) = K^{-1} \otimes F + F \otimes 1 \\
    &\Delta(K) = K \otimes K, \Delta(K^{-1}) = K^{-1} \otimes K^{-1}\\ 
    &\varepsilon(E) = \varepsilon(F) = 0, \varepsilon(K) = \varepsilon(K^{-1}) = 1
\end{align}

and antipode defined by 

\begin{align}
    & S(E) = -EK^{-1}, S(F) = -KF \\
    & S(K) = K^{-1}, S(K^{-1}) = K 
\end{align}
% todo: talk aobut how you get a tensor, etc. structure on the category from a Hopf algebra structure.
It is straightforward to check that this defines a Hopf algebra structure. 

\subsection{The quotient category}
\end{document}
