\documentclass[]{article}
\usepackage{amsmath}
\usepackage{amsthm}
\usepackage{amsfonts}
\usepackage{xypic}

\newtheorem{theorem}{Theorem}[subsection]
\newtheorem{remark}[theorem]{Remark}
\newtheorem{defn}[theorem]{Definition}
\newtheorem{lemma}[theorem]{Lemma}
\newtheorem{example}[theorem]{Example}

\newcommand{\isomto}{\overset{\sim}{\rightarrow}}
\newcommand{\tr}{\operatorname{tr}}
\newcommand{\id}{\operatorname{id}}
\newcommand{\Hom}{\operatorname{Hom}}
\newcommand{\Rep}{\operatorname{Rep}}
\newcommand{\coker}{\operatorname{coker}}
\newcommand{\End}{\operatorname{End}}
\newcommand{\Ob}{\operatorname{Ob}}

\newcommand{\sll}{\mathfrak{sl}}

\numberwithin{equation}{subsection}

\begin{document}
\tableofcontents
\section{Algebraic Background}
\subsection{Modules}

    %todo: actually write this section
In  this section we will be working over a ring $R$ with a unit $1$. 
\begin{defn}
    A $R$-\emph{module} over a ring $R$ is an abelian group $N$ with a binary
    operation $R \times N \to N$ written $r\cdot n$ or $rn$ such that for $r,s \in R$ and $m,n \in N$:

    \begin{enumerate}
        \item r(n + m) = rn + rm
        \item (rs)n = r(sn)
        \item 1n = n
        \item (r+s)n = rn + s
    \end{enumerate}
\end{defn}

\subsection{Hopf algebras}
\subsubsection{Algebras and Coalgebras}
For a field $k$, a \emph{$k$-algebra} $A$ is a $k$-vector space with an
associative bilinear mapping $A \times A \to A$ which has an identity element
$1 \in A$ such that $1\cdot x = x\cdot 1 = x$ for any $x \in A$.

Put in categorical terms, a $k$-algebra is given by a triple $(A, \mu, \eta)$,
where $A$ is a vector space, and $\mu: A \otimes A \to A$ and $\eta: k \to A$
are linear maps satisfying the axioms:

Associativity:
The diagram
\begin{equation}
\xymatrix{
A \otimes A \otimes A \ar[d]^{\id \otimes \mu} \ar[r]^{\mu \otimes \id} & A \otimes A \ar[d]^{\mu}\\
 A \otimes A \ar[r]^\mu & A 
}
\end{equation}
commutes.

Unit: 

The diagram

\begin{equation}
    \xymatrix{
    k \otimes A \ar[r]^{\eta \otimes \id} \ar[rd]_\simeq & A \otimes A \ar[d]^\mu & A \otimes k \ar[l]_{\id \otimes \eta} \ar[ld]^{\simeq} \\
    & A &
    }
\end{equation}
commutes.

An algebra is called \emph{commutative} if $x \cdot y = y \cdot x$ for any $x,y\in A$. In other terms, it needs to satisfy the commutativity axiom:

The triangle 

\begin{equation}
    \xymatrix{
    A \otimes A \ar[rr]^{\tau_{A,A}} \ar[rd]_\mu & & A \otimes A \ar[ld]^\mu \\
    & A &
    }
\end{equation}

Given two algebras $(A_1, \mu_1, \eta_1)$ and $(A_2, \mu_2, \eta_2)$, a linear
map $f: A_1 \to A_2$ is called a \emph{morphism of algebras} or a
\emph{homomorphism} if $f(\mu_1(a,b)) = \mu_2(f(a), f(b))$ for any $a,b\in A_1$ and $f(\eta_1(1)) = \eta_2(1)$.


We can obtain the definition of a coalgebra by reversing all the arrows as follows:

\begin{defn}
    A \emph{coalgebra} is a triple $(C, \Delta, \varepsilon)$ where $C$ is a
    vector space, and $\Delta: C \to C \otimes C$, $\varepsilon: C \to k$ are
    linear maps satisfying the axioms:

Associativity:
The diagram
\begin{equation}
\xymatrix{
C \otimes C \otimes C   & C \otimes C \ar[l]_{\Delta \otimes \id}\\
 C \otimes C\ar[u]^{\id \otimes \Delta}  & C \ar[l]_\Delta \ar[u]_{\Delta}
}
\end{equation}
commutes.

Unit: 

The diagram

\begin{equation}
    \xymatrix{
    k \otimes C  & C \otimes C \ar[l]_{\varepsilon \otimes \id} \ar[r]^{\id \otimes \varepsilon} & C \otimes k   \\
    & C \ar[u]_\Delta \ar[lu]^\simeq \ar[ru]_{\simeq}&
    }
\end{equation}
commutes.
\end{defn}
A coalgebra is called \emph{cocommutative} if the triangle 

\begin{equation}
    \xymatrix{
    A \otimes A   & & A \ar[ll]^{\tau_{A,A}} \otimes A  \\
    & \ar[lu]^\Delta A \ar[ru]_\Delta&
    }
\end{equation}

commutes.


\subsubsection{Bialgebras}

Suppose $H$ is a vector space which has both an algebra structure $(H, \mu,
\eta)$ and a coalgebra structure $(H, \Delta, \varepsilon)$. We call this a
\emph{bialgebra} if the two structures are compatible:

\begin{defn}
    A vector space with an algebra and coalgebra structure is called a
    \emph{bialgebra} if one of the following two equivalent conditions holds:

    \begin{enumerate}
        \item The maps $\mu$ and $\eta$ are morphisms of coalgebras
        \item The maps $\Delta$ and $\varepsilon$ are morphisms of algebras
    \end{enumerate}
\end{defn}

A morphism of bialgebras is a map which is both a morphism of algebras an a
morphism of coalgebras.
\subsubsection{Hopf Algebras}

% todo: introduce Sweedler notation??

\begin{defn}
    Let $(H, \mu, \eta, \Delta, \varepsilon)$ be a bialgebra. An endomorphism
    $S$ of $H$ is called a \emph{antipode} for the bialgebra if the following
    two diagrams commute:

    \begin{equation}
        \xymatrix{
        H \ar[r]^{\Delta} \ar[rrd]_{\varepsilon}& H \otimes H \ar[rr]^{S \otimes \id} & & H \otimes H \ar[r]^{\mu} & H \\
        & & k \ar[rru]_{\eta}& &
        }
    \end{equation}

    \begin{equation}
        \xymatrix{
        H \ar[r]^{\Delta} \ar[rrd]_{\varepsilon}& H \otimes H \ar[rr]^{\id \otimes S} & & H \otimes H \ar[r]^{\mu} & H \\
        & & k \ar[rru]_{\eta}& &
        }
    \end{equation}
\end{defn}

\begin{defn}
    A \emph{Hopf algebra} is a bialgebra with an antipode. A morphism of Hopf
    algebras is a morphism between the bialgebras which commutes with the
    antipodes. 
\end{defn}

\begin{example}
    \label{groupalgebra}
    One important example of a Hopf algebra is the Hopf algebra obtained from
    any finite group. 

    Let $G$ be a finite group. The Hopf algebra $k[G]$ is the algebra with
    basis $\left\{ v_g: g \in G \right\}$, multiplication $v_g v_h = v_{gh}$,
    and unit the group identity. 

    We can define a coalgebra structure on $k[G]$ by $\Delta(v_g) = v_g \otimes v_g$
    and $\varepsilon(v_g) = 1$ for any $g \in G$.

    The antipode $S$ is given by $S(v_g) = v_{g^{-1}}$ for any $g \in G$.

    Note that the group algebra is not commutative if the underlying group is
    not commutative, but it is always cocommutative. The examples of Hopf
    algebras we will be interested in later are neither commutative nor
    cocommutative. 
\end{example}

In general the antipode can be thought of as an analog to the inverse in a
group. If a Hopf algebra has an antipode, it is unique, and $S^2 = \id$.

\subsection{Lie Algebras}

The examples of quantum groups that we are interested in are obtained as deformations of simple Lie algebras. 

A \emph{Lie algebra} is a vector space with a bilinear operation $\left[ \cdot, \cdot \right]$ such that 

\begin{itemize}
    \item $\left[ x,y \right] = -\left[ y,x \right]$
    \item $\left[ x, \left[ y,z \right] \right] + \left[ y, \left[ z,x \right] \right] + \left[ z, \left[ x,y \right] \right] = 0$ (the Jacobi identity)
\end{itemize}

By Ado's theorem, any finite dimensional Lie algebra over a field of characteristic zero can be realized as a vector space of matrices such that 

\[ \left[ X,Y \right] = XY - YX\] 
for any $X,Y$

A basic example of a Lie algebra is $\sll(2)$: the algebra of $2 \times 2$ matrices with trace zero. $\sll(2)$ is generated by the matrices 
\begin{equation}
    X = \begin{pmatrix} 0 & 1 \\ 0 & 0 \end{pmatrix},
    Y = \begin{pmatrix} 0 & 0 \\ 1 & 0 \end{pmatrix}, 
    H = \begin{pmatrix} 1 & 0 \\ 0 &-1 \end{pmatrix}
\end{equation}

It is the lowest-dimensional simple nontrivial Lie algebra. 
%% define a simple Lie algebra

\subsection{Representations of $\sll(2)$}

A \emph{representation} of a Lie algebra $\mathfrak{g}$ is a vector space $V$ together with an action of $\mathfrak{g}$ on $V$ such that 

\begin{align*}
    \left[ x,y \right] v &= x(yv) - y(xv) \\
    (x+y)v &= xv + yv \\
    (ax)v &= a(xv)
\end{align*}

for all $x,y \in \mathfrak{g}, v \in V, a \in \mathbb{C}$

The representations of $\sll(2)$ can be classified as follows: 

For each integer $n \geq 0$, there is a unique (up to isomorphism) representation of $\sll(2)$ of dimension $n+1$. This representation has basis $v_0, \ldots, v_{n}$ such that
% from Kassel, p. 101
\begin{align*}
    &H v_i = (n - 2i) v_i& \\
    &Y v_i = \begin{cases} 
                (i+1)v_{i+1}& \text{ for $i < m$} \\
                0& \text{ for $i = m$} \\
            \end{cases} \\
    &X v_i = \begin{cases} 
                (n-i+1)v_{i-1}& \text{ for $i > 0$} \\
                0& \text{ for $i = 0$} \\
            \end{cases}
\end{align*}


\subsection{Representations}

\section{Categorical Background}
Throughout we will be working over a field $k$ of characterisic 0.
\subsection{Abelian and Monoidal Categories}


\begin{defn}
    A category $\mathcal{C}$ is called \emph{abelian} if it satisfies the conditions:

    \begin{enumerate}
    \renewcommand{\labelenumi}{\roman{enumi})}
        \item All the hom sets $\Hom(A,B)$ are $k$-vector spaces, and the compositions
            
            \begin{equation}
                (\varphi, \psi) \mapsto \varphi \circ \psi
            \end{equation}

            are $k$-bilinear.
        \item There is a zero object $\mathbf{0} \in \Ob \mathcal{C}$ such that $\Hom(0,V) = \Hom(V,0) = 0$ for every object $V$
        \item Finite direct sums exist in $\mathcal{C}$
        \item Every morphism $\varphi$ has a kernel $\ker \varphi$ and a
            cokernel $\coker \varphi$. Every morphism is a composition of an
            epimorphism followed by a monomorphism. If $\ker \varphi = 0$, then
            $\varphi = \ker(\coker \varphi)$. If $\coker \varphi = 0$, then
            $\varphi = \coker(\ker \varphi)$.
    \end{enumerate}

    Examples of abelian categories include the category of $k$-vector spaces,
    the category of finite dimensional $k$-vector spaces, and the category of
    representations of a group $G$ over $k$.

    \begin{defn}
        An object $U$ in an abelian category is called $\emph{simple}$ if any
        injection $V \hookrightarrow U$ is either $0$ or an isomorphism.
    \end{defn}
\end{defn}
% todo: simple objects
\begin{defn}
    A \emph{monoidal category} is a category $\mathcal{C}$ with 
    \begin{enumerate}
            %todo: fix the enumerate labels
    \renewcommand{\labelenumi}{\roman{enumi})}
        \item a bifunctor $\otimes: \mathcal{C} \times \mathcal{C} \to
            \mathcal{C}$
        \item a unit object $\mathbf{1}$ and natural transformations
            \begin{equation}
                \lambda_V : \mathbf{1} \otimes V \isomto V \\
                \rho_V : V \otimes 1 \isomto V
            \end{equation}
        \item a natural transformation 

            \begin{equation}
                \alpha_{UVW} (U \otimes V) \otimes W \isomto U \otimes (V \otimes W)
            \end{equation}

            which satisfy the associativity property 

        \item if $X_1, X_2$ are two objects obtained from $V_1 \otimes V_2 \otimes \cdots V_n$ by inserting 1s and brackets, then all isomorphisms $\varphi: X_1 \isomto X_2$ composed of $\alpha$'s, $\lambda$'s, and $\rho$'s are equal. 
        \item $\mathbf{1}$ is a simple object and $\End_\mathcal{C} \mathbf{1} = k$
    \end{enumerate}

\end{defn}

\begin{example}
    \begin{enumerate}
    \renewcommand{\labelenumi}{\roman{enumi})}
        \item The category of vector spaces over a field $\operatorname{Vec}(k)$
        \item the category of finite dimensional representations of a group, algebra, or Lie algebra
    \end{enumerate}
\end{example}
    
\subsection{Braided monoidal categories}

Let $\mathcal{C}$ be a monoidal category with a natural transformation 

\begin{equation}
    \sigma_{VW} : V \otimes W \to W \otimes V
\end{equation}

% todo: make your grammar wrt natural transformations right

\subsection{Rigid monoidal categories}
% todo: write down the vector space example
A rigid monoidal category is a monoidal category where there is a notion of a dual. 

\begin{defn}
    Let $\mathcal{C}$ be a monoidal category, $V$ an object in $\mathcal{C}$. A \emph{right dual} to $V$ is an object $V^*$ with two morphisms

    \begin{align}
        e_V: V^* \otimes V \to \mathbf{1}  \\
        i_V: \mathbf{1} \to V^* \otimes V
    \end{align}
\end{defn}

such that the composition

\begin{equation}
    V \stackrel{i_v \otimes \id_V}{\xrightarrow{\hspace*{1cm}}} V \otimes V^*
    \otimes V  \stackrel{\id_V \otimes e_V}{\xrightarrow{\hspace*{1cm}}} V
\end{equation}

is equal to $\id_V$, and similarly the composition

\begin{equation}
    V^* \stackrel{id_{V^*} \otimes \id_V}{\xrightarrow{\hspace*{1cm}}} V^*
    \otimes V \otimes V^*  \stackrel{e_V \otimes
    \id_{V^*}}{\xrightarrow{\hspace*{1cm}}} V^*
\end{equation}

is equal to $\id_{V^*}$


\subsection{The Category of Representations of a Hopf Algebra}

Suppose $(H, \mu, \eta, \Delta, \varepsilon)$ is a Hopf algebra with antipode $S$. 

Let $\Rep_f H$ be the category of finite dimensional representations of $H$ as a $k$-algebra.

If $A$ is an algebra and $U, V$ are $A$-modules, then $U \otimes V$ is a vector
space, but there is no natural way to impose a $A$-module structure on $U
\otimes V$. 

The comultiplication $\Delta$ on $H$ allows us to impose a $H$-module structure
on the tensor product $U \otimes V$ of two $H$-modules $U,V$ as follows.

Suppose $\Delta(h) = \sum _{i} h^{(1)}_i \otimes h^{(2)}_i$. Then we define
% todo: define Sweedler notation??

\begin{equation}
    h (u \otimes v) = \sum_{i} h^{(1)}_i u \otimes h^{(2)}_i v
\end{equation}

We define the tensor unit using the counit $\varepsilon$: $\mathbf{1}$ is the vector space $k$, with 

\begin{equation}
    h(1) = \varepsilon(h) 1
\end{equation}

 for any $h\in H$.

So we have that $\Rep_f(H)$ is a monoidal category, with this tensor product.
Only the counit and the comultiplication are required for this definition, so
in fact the category of representations of any bialgebra is a monoidal
category.

We can use the antipode $S$ to define duals as follows:

For any module $U$, let the dual $U^*$ be the dual vector space of linear functionals on $U$, with action
\begin{equation}
    (h\cdot \varphi)(u)  = \varphi(S(h) u)
\end{equation}

It follows that $\Rep_f(H)$ is a rigid monoidal category. This also serves as
motivation for the definition of a Hopf algebra: it is an algebra with
additional structures such that its category of representations is monoidal and
rigid. 

% maybe say that the category of reps of the group has this structure, and it
% follows that the category of reps of the group algebra k[G] has this
% structure, and the Hopf algebra stuff is a way to generalize that




\subsection{Ribbon Categories}

In what follows it will become important to have a notion of the trace of a
morphism. The definitions that follow will allow us to define a trace in a
certain class of rigid monoidal categories called ribbon categories.

\begin{defn}

    A \emph{ribbon category} is a rigid braided tensor category with a natural isomorphism
    \begin{equation}
        \delta_V: V \to V^{**}
    \end{equation}

such that 
\begin{enumerate}
    \renewcommand{\labelenumi}{\roman{enumi})}

    \item $\delta_{V \otimes W} = \delta_V \otimes \delta_W$
    \item $\delta_1 = \id$
    \item $\delta_{V^*} = (\delta_V^*)^{-1}$
\end{enumerate}

\end{defn}

If $V$ is an object in a ribbon category $\mathcal{C}$ and $f$ an endomorphism of $V$, we can define the trace of $f$ by the composition

\begin{equation}
    %todo: why are my arrow labels not in the middle?
    \xymatrix{
    \mathbf{1} \ar[r]^{i_V} & V \otimes V^* \ar[r]^{f \otimes \id} & V^* \otimes V \ar[r]^{\delta_V \otimes \id} & V^{**} \otimes V^* \ar[r]^{e_{V^*}} & 
    \mathbf{1}
    }
\end{equation}

We define the dimension of an object $V$ to be $\dim V = \tr \id_V$.

\subsection{Semisimple Categories}
\begin{defn}
    An abelian category $\mathcal{C}$ is \emph{semisimple} if any object $V$ is isomorphic to a direct sum of simple objects

    \begin{equation}
        V \simeq \bigoplus{i} N_i V_i
    \end{equation}

    where the $V_i$ are simple objects, $N_i \in \mathbf{N}$

\end{defn}

    Suppose that $\mathcal{C}$ is a semisimple ribbon category. Let $I$ be the
    set of equivalence classes of nonzero simple objects in $\mathcal{C}$ and
    choose a representive $V_i$ for each equivalence class  $i \in I$.
    
    We can define the \emph{fusion coefficients} $N_{ij}^k \in \mathbf{N}$.

    \begin{equation}
        V_i \otimes V_j \simeq \bigoplus_k N_{ij}^k V_k
    \end{equation}

    We call each equation of this type a \emph{fusion rule}. 

    \subsection{Modular Tensor Categories}

The \emph{dimension} of a category $\mathcal{C}$ is defined by 

\begin{equation}
    \operatorname{dim}(\mathcal{C}) = \sum_{ x \in \Ob \mathcal{C}} (\operatorname{dim} x)^2
\end{equation}


\begin{defn}
    The \emph{symmetric center} of a braided category $\mathcal{C}$ $\mathbf{Z}_2(\mathcal{C})$ is a full subcategory with 

    \begin{equation}
        \Ob \mathbf{Z}_2(\mathcal{C}) = \left\{ x \in \mathcal{C} : \sigma_{XY} \circ \sigma_{YX} = \id\ \forall Y \in \mathcal{C} \right\}
    \end{equation}
\end{defn}

% todo: define fusion category
% todo: put some more stuff in here from Bakalov & Kirillov
\begin{defn}
    A modular tensor category is a braided fusion category such that $\mathbf{Z}_2(\mathcal{C})$ is trivial (every object is isomorphic to $\mathbf{1}^{\oplus n}$). Equivalently, every simple object in the center is isomorphic to $\mathbf{1}$.
\end{defn}

Why are these categories called modular? 
%todo: finish this 
Define 

\begin{equation}
    % todo: make this a picture??
    s_{X,Y} = \tr_{X \otimes Y}(\sigma_{Y,X} \circ \sigma_{X,Y})
\end{equation}

for every simple object $X,Y$.

\begin{theorem}
    % todo: find a citation for this
    $\mathbf{Z}_2(\mathcal{C})$ is trivial if and only if the matrix $s$ is invertible. 
\end{theorem}

\begin{example}
    An elementary class of examples of modular tensor categories can be obtained via the quantum double construction as follows. 

    % todo: write something about this being a simpler example of something
    % more general that Drinfel'd did

    Recall the Hopf algebra $k[G]$ of any finite group $G$ from \ref{groupalgebra} with basis $\left\{ v_g \right\}_{g \in G}$ and

    % todo: fix aligning here
    \begin{align}
        &\text{multiplication} &v_g v_h = v_{gh} \\
        &\text{unit}           &e \\
        &\text{comultiplication} &\Delta(v_g) = v_g \otimes v_g \\
        &\text{counit}           &\varepsilon(v_g) = 1 \\
        &\text{antipode}         &S(v_g) = v_{g^-1}
    \end{align}

    The dual Hopf algebra to $k[G]$ is the function algebra $F(G)$ with basis $\left\{ \delta_g : g \in G \right\}$ of functions

    \begin{equation}
        \delta_g(x) = \delta_{g,x} = \begin{cases} 1 &\text{for $g = x$} \\ 0 &\text{for $g \neq x$} \end{cases}
    \end{equation}

    \begin{align}
        &\text{multiplication} &\delta_g \delta_h = \delta_{g,h} \delta_g\\
        &\text{unit}           &\sum_{g \in G} \delta_g \\
        &\text{comultiplication} &\Delta(\delta_g) = \sum_{g_1 g_2 = g} \delta_{g_1} \otimes \delta_{g_2}\\
        &\text{counit}           &\varepsilon(\delta_g) = \delta_{e,g} \\
        &\text{antipode}         &S(\delta_g) = \delta_{g^-1}
    \end{align}


    The quantum double $D(G)$ of $k(G)$ can be described as follows. $D(G)$ is the Hopf algebra with vector space $F(G) \otimes_k k[G]$ and 

    \begin{align}
        &\text{multiplication} &(\delta_g\otimes v_x) (\delta_h \otimes v_y) = \delta_{gx,xh} (\delta_g \otimes v_{xy})\\
        &\text{unit}           &\sum_{g \in G} \delta_g  \otimes v_e\\
        &\text{comultiplication} &\Delta(\delta_g \otimes v_x) = \sum_{g_1 g_2 = g} (\delta_{g_1} \otimes v_x) \otimes (\delta_{g_2} \otimes v_x) \\
        &\text{counit}           &\varepsilon(\delta_g \otimes v_x) = \delta_{e,g} \\
        &\text{antipode}         &S(\delta_g \otimes v_x) = \delta_{x^{-1}g^-1 x} \otimes v_{x^{-1}}
    \end{align}


\end{example}


\section{The Representation Theory of $U_q(\sll(2))$}
\subsection{The quantized universal enveloping algebra $U_q(\sll(2))$}
% Kassel, p. 121
We first introduce some notation. For any integer $n$, let 

\begin{equation}
    [n]_q = \frac{q^n - q^{-n}}{q - q^{-1}} = q^{n-1} + q^{n-3} + \cdots + q^{-n+3} + q^{-n+1}
\end{equation}

\begin{remark}
$q^{2d}=1$ if and only if $[d]_q = 0$, so $[n]_q \neq 0$ for every nonzero integer when $q$ is not a root of unity. 
\end{remark}

%% Jantzen, p. 9
Let $q$ be a complex number such that $q \neq 0$, $q^2 \neq 1$.  Then $U_q(\sll(2))$ is the associative algebra with generators $E,F,K, K^{-1}$ and relations 

\begin{align}
    KK^{-1} &= 1 = K^{-1}K \\
    KEK^{-1} &= q^2 E \\
    KFK^{-1} &= q^{-2} F \\
    [E,F] &= \frac{K - K^{-1}}{q - q^{-1}}
\end{align}

%% this presentation taken verbatim from Kassel, p. 125

$U_q(\sll(2))$ is isomorphic to the algebra $U'_q(\sll(2))$ with generators $E,F,K,K^{-1},L$ and relations:
%todo: fix the numbering here
\begin{equation}
    \begin{gathered}
    KK^{-1} = 1 = K^{-1}K \\
    KEK^{-1} = q^2 E \\
    KFK^{-1} = q^{-2} F \\
    [E,F] = L, (q - q^{-1})L = K-K^{-1} \\
    [L,E] = q(EK + K^{-1}E), [L,F] = -q^{-1}(FK + K^{-1}F) \\
    \end{gathered}
\end{equation}

% todo: do I need to explain quotients by ideals? Do I need to define algebras? Help.


For $q=1$, we can define a homomorphism $\varphi: U'_1(\sll(2)) \to
U(\sll(2))$ by sending $E$ to $X$, $F$ to $Y$, $K$ to 1, and $L$ to $H$. 
\begin{lemma}
This homomorphism is surjective and has kernel generated by $(K-1)$, so
$U'_1(\sll(2)) \stackrel{\sim}{=} U(\sll(2))$
\end{lemma}
%todo: should I explain why the representation theory looks this way? 
% from Kassel, p. 123
There is a unique algebra automorphism of $U_q(\sll(2))$ such that 

\begin{equation}
    \omega(E) = F, \omega(F) = E, \omega(K) = K^{-1}
\end{equation}

This is called the \emph{Cartan automorphism}.

\subsection{A Hopf Algebra Structure on $U_q(\sll(2))$}

We can define a Hopf algebra structure on $U_q(\sll(2))$ as follows:

%todo: fix formatting
% from Kassel, p. 140
\begin{align}
    &\Delta(E) = 1 \otimes E + E \otimes K, \Delta(F) = K^{-1} \otimes F + F \otimes 1 \\
    &\Delta(K) = K \otimes K, \Delta(K^{-1}) = K^{-1} \otimes K^{-1}\\ 
    &\varepsilon(E) = \varepsilon(F) = 0, \varepsilon(K) = \varepsilon(K^{-1}) = 1
\end{align}

and antipode defined by 

\begin{align}
    & S(E) = -EK^{-1}, S(F) = -KF \\
    & S(K) = K^{-1}, S(K^{-1}) = K 
\end{align}
It is straightforward to check that this defines a Hopf algebra structure. 

\subsection{Representations of $U_q(\sll(2))$, $q$ not a root of unity}

When $q$ is not a root of unity, the representation theory of $U_q(\sll(2))$ bears a striking resemblance to that of $\sll(2)$.

Every representation of $U_q(\sll(2))$ is semisimple, and the irreducible representations are classified as follows:



\subsection{Representations of $U_q(\sll(2))$, $q$ a root of unity}
% Jantzen, 2.11 - 2.13. June 2010 in Clairefontaine book

%% Explain this explicitly
The representation theory of the associative algebra $U(\sll(2))$ can
be described in exactly the same way as that of the Lie algebra
$\sll(2)$. 

%todo: all the content is here I think, but structure it better 
If $q$ is not a root of unity, then there are no simple $U_q(\sll(2))$-modules
with dimension $\geq \ell+1$. 

The irreducible representations of
$U_q(\sll(2))$ can be classified as follows: 

For each $n\geq 0$, there are 2 simple $U_q(\sll(2))$-modules of dimension $n+1$: 
% todo: fix formatting here
$V_+(n)$ with basis $v_0, \ldots, v_n$ \\
$V_-(n)$ with basis $v'_0, \ldots, v'_n$

such that: 
% todo: find a better way to motivate q-notation. (maybe that paper that
% prakash showed you?)
\begin{align*}
    &K v_i = q^{n-2i} v_i  &
    &K v'_i = -q^{n-2i} v'_i \\
    &F v_i = \begin{cases} [i+1]_qv_{i+1}& \text{ if $i < n$} \\ 0& \text{ if $i = n$} \end{cases} &
    &F v'_i = \begin{cases} [i+1]_qv'_{i+1}& \text{ if $i < n$} \\ 0& \text{ if $i = n$} \end{cases} \\
    &E v_i = \begin{cases} 
                     [n-i+1]_qv_{i-1}& \text{ if $i > 0$} \\ 
                    0& \text{ if $i = 0$} 
             \end{cases} &
    &E v'_i = \begin{cases} 
                    -[n-i+1]_q v'_{i-1}& \text{ if $i > 0$} \\ 
                    0& \text{ if $i = 0$} 
             \end{cases}
\end{align*}

% todo: why are there 2 representations here instead of 1? what extra freedom
% allows that to happen?

Suppose $\ell$ is odd, and $q$ is an $\ell^{\text{th}}$ root of unity. 

If $0 \leq n \leq \ell - 1$, the $(n+1)$-dimensional
$U_q(\sll(2))$-modules are the $V_{\pm}(n)$ described above. 

It remains to classify the $\ell$-dimensional $U_q(\sll(2))$-modules.
There are $3$ infinite classes of simple $\ell$-dimensional representations of  $U_q(\sll(2))$:
\begin{enumerate}
        % todo: say somewhere that the results and notation are taken from
        % Jantzen
        \item For any $b,\lambda \in \mathbf{C}, \lambda \neq 0$, define
            $Z_b(\lambda)$ to be the module with basis $v_0, \ldots, v_{\ell -
            1}$ with action of $U$ given by: 
\begin{align*}
    &K v_i = q^{-2i} \lambda v_i \\
    &F v_i = \begin{cases} v_{i+1}& \text{ if $i < n$} \\  b v_0& \text{ if $i = n$} \end{cases}  \\
    &E v_i = \begin{cases} 
        [i]_q \frac{\lambda q^{1-i} - \lambda^{-1} q^{i-1}}{q - q^{-1}} v_{i-1}& \text{ if $i > 0$} \\ 
                    0& \text{ if $i = 0$} 
             \end{cases} 
\end{align*}

If $b \neq 0$ or $\lambda^{2\ell}\neq 1$ then $Z_b(\lambda)$ is simple.
$Z_0(\pm q^k)$ is simple if and only if $k = \ell - 1$. 

\begin{remark}
    $U_q(\sll(2))$ is not semisimple: the modules $Z_0(\pm q_k)$ are not semisimple for $0 \leq k < \ell - 1$.
\end{remark}
% todo: include some representation theory background. Give a more coherent
% definition of a module. define semisimple. define simple (equiv. irreducible)
% from Jantzen
\item For any $U_q(\sll(2))$-module $N$, define $^\omega N$ to be
    equal to $N$ as a vector space, and where $u$ acts on $^\omega N$ as
    $\omega(u)$ acts on $N$.

    The $^\omega Z_b(\lambda)$ are another class of modules.
\item Let $a,b,\lambda \in \mathbf{C}$, $a,b\neq 0$
    % todo: say something about them not all being the same.
\begin{align*}
    &K v_i = q^{-2i} \lambda v_i \\
    &F v_i = \begin{cases} 
                v_{i+1}& \text{ if $i < n$} \\  
                b v_0& \text{ if $i = n$} 
             \end{cases}  \\
    &E v_i = \begin{cases} 
                \left(ab + [i]_q \frac{(\lambda q^{1-i} - \lambda^{-1} q^{i-1})}{q - q^{-1}}\right) v_{i-1}& \text{ if $i > 0$} \\ 
                a v_{\ell - 1}& \text{ if $i = 0$} 
             \end{cases} 
\end{align*}
\end{enumerate}


\subsection{Representations of $U_q(\mathfrak{g})$}
    \subsubsection{$U_q(\mathfrak{g})$}
    \subsubsection{Representation theory of $U_q(\mathfrak{g})$}


\section{Modular Tensor Categories from $U_q(\mathfrak{g})$}
\subsection{Tilting modules}
\subsection{Negligible morphisms}
\subsection{Construction of the MTC}
\subsection{Fusion rules}
% from Bakalov & Kirillov, section 3.3
We will denote the category of representations of $U_q(\sll(2))$ at a
$k^\text{th}$ root of unity $q$ by $\mathcal{C}(\sll(2), k)$. 
This is a rigid monoidal category because of the Hopf algebra structure on
$U_q(\sll(2))$ described above. However, there are infinitely many
simple objects and further the category is not semisimple. Our goal is to find
an appropriate semisimple part of this category to obtain a modular tensor
category.

The first step is to restrict our attention to the subcategory of \emph{tilting modules}. 

% todo: define maximal submodule?
\begin{defn}
    A \emph{composition series} for a module $M$ is a sequence of submodules 

    \begin{equation}
        \left\{ 0 \right\} = J_0 \subset \cdots \subset J_n = M
    \end{equation}

    such each $J_k$ is a maximal submodule of $J_{k+1}$
\end{defn}
\begin{defn}
    A $U_q(\sll(2))$-module $T$ is called \emph{tilting} if both $T$ and $T^*$ have composition series with factors $V_n$.
    % TODO: define the V_n
\end{defn}

To make this subcategory into a modular tensor categories, we will to quotient
this category by the modules with quantum dimension zero. 

\begin{defn}
    A tilting module $T$ is \emph{negligible} if $\tr_q(f) = 0$ for all $f: T \to T$.
\end{defn}
\begin{lemma}
    %% todo: make this less general
    $T$ is negligible if and only if $T = \bigoplus _{\lambda \notin \mathcal{C}} n_\lambda T_\lambda$
\end{lemma}

\begin{defn}
    A morphism $f: T_1 \to T_2$ is \emph{negligible} if $\tr_q(fg) = 0$ for any $g: T_2 \to T_1$
\end{defn}

\begin{defn}
    Define the category $\mathcal{C}^\text{int}$ be the category with objects tilting modules and morphisms 

    \begin{equation*}
        \Hom(V,W) = \Hom_T(V,W) / \text{negligible morphisms}
    \end{equation*}
\end{defn}

The category $\mathcal{C}^\text{int}$ has the following properties:
\begin{enumerate}
    \item An object $T$ is negligible if and only if it is isomorphic to 0 in $\mathcal{C}^\text{int}$
    \item  $\mathcal{C}^\text{int}$ is a ribbon category
    \item Any object $T$ in $\mathcal{C}^\text{int}$ is isomorphic to % todo: fix
    \item $\mathcal{C}^\text{int}$ is a semisimple abelian category. 
    \item $\dim_{\mathcal{C}^\text{int}} T > 0$ for every $T \not\simeq 0$
\end{enumerate}

$\mathcal{C}^\text{int}$ is in fact a modular tensor category. A proof of this
can be found in Bakalov \& Kirillov. %todo: make this a real reference
 

% Bakalov & Kirillov

\section{From MTC to TQC}
\subsection{Overview of TQC}
\subsection{The Fibonacci anyon}
\subsection{Universality}
\end{document}
